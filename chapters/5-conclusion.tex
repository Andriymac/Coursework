\anonsection{Висновок}{}

Використовувати \LaTeX  \ для написання наукових праць, переважно з математики та фізики, є стандартом. \LaTeX \  надає широкий вибір зі свого інстументарію для створення документів різних рівнів складності. Також варто пам'ятати про високий рівень поліграфіної якості для друку. 

Головною відмінністю \LaTeX \ від систем WYSIWYG(What you see is what you get - те що ви бачите, те й отримуєте) є те що створивши наперед шаблон(стильовий файл), можна не задумуватися над форматуванням тексту. \LaTeX \  все зробить за Вас. Він відноситься до типу WHSIWYM(What you see is what you mean - те що ви бачите є тим що ви задумали). 

В ході виконання курсової роботи було оглянуто історію \LaTeX \ і TeX взагалі: \LaTeX \ походить від TeX, який розробив Дональд Кнут у 80-х роках для своєї книги "Мистецтво програмувати". А \LaTeX \ розробив Леслі Лампорт у союзі з Пітером Гордоном в Аддісоні-Веслі, просуваючи його як "TeX для мас".

В другому розділі розглянуто деякі основні команди для роботи з \LaTeX. Як зробити текст жирним, курсивом, вставити зображення, таблицю і т.д. 

В третьому розділі було створено шаблон для курсових робіт та у 4 розділі його порівняння з іншими шаблонами. Його особливістю є те що використовується титульний аркуш згідно з нормами ТНТУ ім. І. Пулюя, також зроблено сторінку для завдання та календарного плану. З використанням пакету \textit{eskdx} в документі є рамки --- одна з основних вимог оформлення.

Створений шаблон не претендує на звання найкращого, ще є багато моментів, які слід доробити. Проте вже зараз його можна використовувати для написання курсових робіт. Згодом він буде тільки поліпшуватися та можливо стане стандартним шаблоном для нашого університету.