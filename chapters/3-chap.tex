\newSection{Розробка шаблону}{sec:sec3}{}

Почати роботу з виготовлення шаблону потрібно з вимог до завдання, а саме: потрібно розробити шаблон для курсових робіт з дотриманням ДСТУ 3008:2015. Ось деякі плавила оформлення:

\begin{itemize}
	\item Аркуші формату А4 (210х297 мм).
	\item Шрифт 14 розміру з 1,5 інтервалом.
	\item  віддаль між рядками повинна бути однакова і рівна 8-10 мм.
	\item  відстань між заголовками підрозділів або пунктів і подальшим або попереднім текстом 15-20 мм;
	\item  відстань між назвою розділу і назвою підрозділу або пункту 18-22 мм;
	\item  абзацний відступ повинен бути однаковим впродовж усього тексту записки і дорівнювати 10-15 мм.
\end{itemize}

Ще одною умовою є додавання рамок до роботи. І останній пункт, додатковий - зверстати титульну сторінку.

\subsection{Написання преамбули}

\subsubsection{Клас документа та кодування}

Для звичайної курсової роботи, без рамок, можна було б скористатися класом документа \textit{article} або \textit{extarticle}, проте довелося б реалізовувати рамки вручну. На просторах інтернету можна знайти колекцію пакетів \textit{eskdx}, який надає нам такий функціонал. Колекція пакетів надає 3 класи: \textit{eskdtext}(для текстової документації), \textit{eskdgraph}(для креслення схем) і \textit{eskdtab}(для документів, разбитих на графи). Для нашого випадку підходить eskdtext, його і використаємо.

\newpage

Ось як буде виглядати підключення класу:

\begin{lstlisting}
		\documentclass[14pt,ukrainian,utf8, simple, pointsection,
		floatsection ]{eskdtext} 
\end{lstlisting}

В додаткових параметрах вказано 14 розмір шрифту, українську мову(вибір тільки з 2, інша - російська), кодування, simple - відображати тільки основні графи, останні два параметри вказують на нумерування пунктів та фігур в межах секцій.

Для роботи з кирилицею потрібно підключити мовний пакет. На вибір є 2 основних: \textit{babel} та \textit{poliglossia}, скористаємося другим. Перевага другого пакету в тому що кириличні символи кодуються правильно, навідміну від babel, який робить заміну \textit{і} на латиницю. Це допоможе при проходженні роботи на антиплагіат.

Підключаємо його та ще пару пакетів для кодування кирилиці:

\begin{lstlisting}
		\usepackage{fontspec}
		\usepackage{xecyr}
		\usepackage{polyglossia}
		\setmainlanguage{ukrainian}
		\usepackage{xunicode, xltxtra}
\end{lstlisting}


\subsubsection{Специфічні налаштування класу}


Пакети ESKDX надають багато налаштувань, їх ми будемо зберігати в окремому файлі з назвою \textit{ESKDXconfig.tex}. Він є частиною преамбули тож і підключатиметься там. В цьому файлі зберігатимуться попередні команди і також для налаштування шрифтів.

\begin{lstlisting}
		\defaultfontfeatures{Ligatures=TeX}
		\setmainfont{Times New Roman} 
		\newfontfamily\cyrillicfont{Times New Roman}
		\setotherlanguage{english}
		\setmonofont{FreeMono}
\end{lstlisting}

Вказавши деякі команди можна створити титульну сторінку, проте її формати не підходить, тому ми будемо писати свою титулку в наступному підрозділі. Зараз тільки вкажемо деякі необхідні команди для задання інформації на рамках.
\begin{lstlisting}
			\ESKDsignature{ КПКН 20.055.014.000 ПЗ
			\ESKDcolumnIX{{\small ТНТУ, ФІС, КН СН-31}}
			\ESKDtitle{\ESKDfontIII Створення шаблону для курсової в Latex}
			\ESKDauthor{ **** А. В. }
			\ESKDchecker{ **** О. Б. }
			
\end{lstlisting}

Останнє в цьому файлі це налаштування стилів відображення секцій:

\begin{lstlisting}
			\ESKDsectAlign{section}{Center}
			\ESKDsectStyle{section}{\normalsize \bfseries \uppercase}
			\ESKDsectStyle{subsection}{\normalsize \bfseries}
			\ESKDsectSkip{section}{0pt}{0.8cm}
			\ESKDsectSkip{subsection}{0.8cm}{0.5cm}
			\ESKDsectSkip{subsubsection}{0.5cm}{0.1pt}
\end{lstlisting}

Секції центруємо, робимо 14 шрифтом, жирний та все у верхньому регістрі. Підсекції - те саме, тільки у звичайному регісті. Також налаштовано відступи між секціями.

Для цього файлу - це все, повний обсяг буде наведено в додатках.

\subsubsection{Створення титульної сторінки}

Налаштування титульної сторінки будуть знаходитися в окремому файлі \textit{title.tex}.

Спершу налаштуємо колонтитули, оскільки це найлегше що можна зробити зараз:
\begin{lstlisting}
		\usepackage{fancyhdr} % Колонтитули
		\pagestyle{fancy}
		
		\fancypagestyle{firststyle}{
		\renewcommand{\headrulewidth}{0pt}
		\fancyfoot{}
		\cfoot{Тернопіль 2020}
		}
		\renewcommand{\headrulewidth}{0pt}
		\fancyfoot{}
		\fancyhead{}
\end{lstlisting}

Використали пакет \textit{fancyhdr} і створили стиль колонтитула для сторінки в якому по центрі внизу записали необхідні дані. 

Створимо нову команду для титульної сторінки, яка буде приймати аргументи для їх встановлення на сторінку: найменування вищого навчального закладу, кафедра, назва роботи, тема, дисципліна, і т.д. Також буде створено ще 2 команди для завдання та календарного плану. Команди можуть приймати до 9 параметрів, в нашому випадку їх є більше, тому скористаємося пакетом \textit{keyval}, який дозволить використовувати опційні параметри, які не обмежуються кількістю.
Оголошення змінних та задання стандартних значень:

\begin{lstlisting}
			\define@key{titlee}{university}{\def\tl@university{#1}}
			\define@key{titlee}{katedra}{\def\tl@katedra{#1}}
			\define@key{titlee}{type}{\def\tl@type{#1}}
			\define@key{titlee}{discipline}{\def\tl@discipline{#1}}
			\define@key{titlee}{thema}{\def\tl@thema{#1}}
			\define@key{titlee}{kurs}{\def\tl@kurs{#1}}
			\define@key{titlee}{group}{\def\tl@group{#1}}
			\define@key{titlee}{specialty}{\def\tl@specialty{#1}}
			\define@key{titlee}{author}{\def\tl@author{#1}}
			\define@key{titlee}{posada}{\def\tl@posada{#1}}
			\define@key{titlee}{kerivnyk}{\def\tl@kerivnyk{#1}}
			\define@key{titlee}{pidpys}{\def\tl@pidpys{#1}}
			
			% zavdannia
			\define@key{titlee}{semestr}{\def\tl@semestr{#1}}
			\define@key{titlee}{date}{\def\tl@date{#1}}
			\define@key{titlee}{fulldate}{\def\tl@fulldate{#1}}
			%kalendar
			\define@key{titlee}{enddate}{\def\tl@enddate{#1}}
			\define@key{titlee}{sources}{\def\tl@sources{#1}}
			\define@key{titlee}{zapyska}{\def\tl@zapyska{#1}}
			\define@key{titlee}{graphika}{\def\tl@graphika{#1}}
			
			\setkeys{titlee}{university= University ,katedra = Kафедра,
			type= Курсова робота ,thema= {Тема \\ \ } , discipline = Предмет,
			kurs=№, group=ke-4, specialty=122 - CS, author=\qquad \qquad
			\qquad \qquad,posada=,kerivnyk=Teacher, semestr=4, date=,
			sources=sources, graphika=,enddate=,fulldate=, zapyska={\
			\quad \\ \ \\ \ \\ \ \\ \ \\ \ }, pidpys=}%			
\end{lstlisting}


Створюємо нові команди, всі будуть приймати один опційний аргумент, який потім в \verb|\setkeys{}|встановиться у відповідні змінні. Для використання змінних потрібно щоб вони перебували в групі, це оголошується \verb|\begingroup%|:

\begin{lstlisting}
			\newcommand{\setzavdannia}[1][]{
			\begingroup%
			\setkeys{titlee}{#1}% Set new keys
			...content
			\endgroup%
			}
\end{lstlisting}

Далі проведено опис основних моментів створення титульної сторінки, для детальнішого огляду див. додатки.

Починаємо зверху, де треба вказати навч. заклад та кафедру:

\begin{lstlisting}
		\centering
		Міністерство освіти і науки України\\
		\tl@university
		\hrule
		{\scriptsize (повне найменування вищого навчального закладу)}
		\hspace{0.2cm}
		Кафедра \tl@katedra
		\hrule
		{\scriptsize (повна назва кафедри)}
\end{lstlisting}

Команда \verb|\centering| відцентрує текст. У 2 рядку встановлюємо на параметр на його місце та малюємо горизонтальну лінію на всю ширину - \verb|\hrule|. \verb|\scriptsize| зменшує шрифт на дуже маленький.

Результат:

\addimghere{3}{1}{Верхня частина титулки}{}

Наступним створюємо тип та назву роботи:

\begin{lstlisting}
			\vspace{3cm}
			\begin{center}
			\textbf{ \large \tl@type}
			\end{center}
			
			з  \textbf{<<\tl@discipline>>}
			\hrule
			
			{\scriptsize (назва дисципліни)}
			\hspace{0.2cm}
			\setlength{\unitlength}{1cm}
			
			\begin{picture}(0,0)
				\put(-9,-1.85){\line(1,0){18}}
				\put(-9,-0.75){\line(1,0){18}}
				\put(-9,-1.3){\line(1,0){18}}
			\end{picture}
			
			на тему : \textbf{<<\tl@thema>>}
			}			
					
\end{lstlisting}

Команди \verb|\vspace{}, \hspace{}| роблять відступ вертикально та горизонтально на вказану відстань. Середовище \verb|\begin{picture}(0,0)| Робить область для створення простих графіків, фігур. В дужках вказано її розміри, в нас нема розміру для зручності, оскільки будуть малюватися лінії серед тексту. 

Команда \textit{put} ставить певну фігуру за вказаними координатами, її можна використовувати тільки в середовищі \textit{picture}. \textit{Line} - фігура лінії, в дужках осі, в параметрі довжина.

Ось результат:

\addimghere{4}{1}{Назва роботи, тема}{}

Перейдемо до останньої частини, де вказується інформація про автора та керівника. 

\begin{lstlisting}
			\vspace{5cm}
			\hfill
			\begin{minipage}{0.5\linewidth}
				\begin{tabular}{lp{0.1\linewidth}ccp{0.145\linewidth}}
					Студента & \centering \tl@kurs & курсу, & групи & \tl@group
					\\
					\cline{2-2} \cline{5-5}
				\end{tabular}
				\vspace{0.1cm}
				\begin{tabular}{lc}
					спеціальності & \tl@specialty \\
					\hline
					\tl@author    & \tl@pidpys    \\
					\hline
				\end{tabular}
				\vspace{-0.8cm}\hspace{0.3cm}	{ \centering\scriptsize (прізвище 
				та ініціали)  \hspace{2cm}(підпис студента)} \\
				
				\begin{tabular}{p{0.3\textwidth}p{0.6\textwidth}}
					Керівник: & \tl@posada \\
					\cline{2-2}
					\multicolumn{2}{c}{\tl@kerivnyk}\\
					\hline
				\end{tabular}
				\centering
				{\scriptsize (посада, вчене звання, науковий ступінь, прізвище
				та ініціали) }
			\end{minipage}
\end{lstlisting}

\addimghere{5}{1}{Остання частина титульної сторінки}{}


Поєднання \textit{begin\{minipage\}} та \textit{hfill} робить окрему область сторінки на половину її ширини та розташовує її справа. Для організації розмітки використовується таблиця, командами \verb|\cline{2-2} \cline{5-5}|, можна підкреслити необхідні стовпці в рядку.

Поєднуючи таблиці та малювання ліній, було зроблено необхідні лінії для 3 сторінок: титулка, завдання. календарний план.

\subsection{Завершення преамбули}

Для роботи із зображеннями підключаємо пакет \textit{graphicx}, вказуємо їхнє розташування:

\begin{lstlisting}
			\usepackage{graphicx} % Вставка картинок 
			\graphicspath{{images/}}
\end{lstlisting}

Напишемо пару нових команд для вставлення картинок. Перша команда вставляє одне зображення в зручному для Latex місці, приймає 4 команди: назва, ширина, підпис, позначка. Друга команда робить все те саме, проте вставляє картинку так як вона є в тексті. Остання команда вставляє 2 зображення поруч з одним підписом до них.

\begin{lstlisting}
			\newcommand{\addimg}[4]{ % add one img
				\begin{figure}
					\centering
					\includegraphics[width=#2\linewidth]{#1}
					\caption{#3} \label{#4}
				\end{figure}
			}
			\newcommand{\addimghere}[4]{ % add img here
				\begin{figure}[H]
					\centering
					\includegraphics[width=#2\linewidth]{#1}
					\caption{#3} \label{#4}
				\end{figure}
			}
			\newcommand{\addtwoimghere}[4]{ % two img side by side
				\begin{figure}[H]
					\centering
					\begin{subfigure}[t]{0.45\textwidth}
						\includegraphics[width=\textwidth]{#1}
					\end{subfigure}
					\begin{subfigure}[t]{0.45\textwidth}
						\centering
						\includegraphics[width=\textwidth]{#2}
						
					\end{subfigure}
					\caption{#3}\label{#4}
				
				\end{figure}
			}
\end{lstlisting}

Створимо ще дві функції для роботи із секціями. Команда \textit{newSection} додає нову секція з нової сторінки, також є позначення для посилання на неї, останнє - це застосування розширеної рамки і вказанням того самого розділу. Друга команда схожа, тільки секції будуть без нумерації.
\begin{lstlisting}
			\newcommand{\newSection}[3]{
				\newpage
				\section{\uppercase{#1}}
				\label{#2}
				\ESKDcolumnI{#3#1}
				\ESKDthisStyle{formII}
			}
			
			\newcommand{\anonsection}[2]{
				\newpage
				\phantomsection
				\addcontentsline{toc}{section}{\uppercase{#1}}
				\section*{\uppercase{#1}}
				\ESKDcolumnI{\uppercase{#1}}
				\label{#2}
				\ESKDthisStyle{formII}
			}
\end{lstlisting}

Один дуже важливий нюанс. Пакет ESKDX не надає можливості зміни даних в рамках в тілі документа, як це зробили ми, це призведе до помилки. Один з варіантів вирішення є написання власного стилю рамки, проте це дуже важко і потребує поглиблених знань пакету. Другий варіант - зміна пари рядків коду в початкових файлах пакету. Під час компіляції можна побачити що викликається команда збереження рамки в преамбулі та застосовується без змін далі, тому треба замість застосування - заново намалювати. Отже у пакетному файлі \textit{eskdstamp.sty} потрібно знайти 2 форму рамок(\verb|\ESKD@stamp@ii@box|) та скопіювати код створення рамки в місце де застосовується збережена рамка. 

Налаштуємо підписи до малюнків та таблиць:

\begin{lstlisting}
			\RequirePackage{caption}
			\DeclareCaptionLabelSeparator{defffis}{ -- } % Розділювач
			\captionsetup[figure]{justification=centering, labelsep=defffis,
			 format=plain} % Підпис малюнка по центру
			\captionsetup[table]{justification=raggedleft, labelsep=defffis, 
			format=plain, singlelinecheck=false} % Підпис таблиці справа
			\addto\captionsukrainian{\renewcommand{\figurename}{Рисунок}} 
			% Ім' я фігури
\end{lstlisting}

Оформимо відображення початкового коду, використовуючи пакет \textit{listings}:

\begin{lstlisting}
			\usepackage{listings}
			\lstset{
				basicstyle=\small\ttfamily,
				breaklines=true,
				tabsize=2,                  
				extendedchars=\true,
				keepspaces=true,
				literate={--}{{-{}-}}2,     
				literate={---}{{-{}-{}-}}3, 
				texcl=true, }
\end{lstlisting}

Останнє це оформлення секцій у вступі, зробимо так щоб відображалися крапки від секції до номера сторінки і щоб усе було нежирним:

\begin{lstlisting}
			\makeatletter
				%format sections in tableofcontents
				\renewcommand{\l@section}
					{\@dottedtocline{1}{0em}{1.25em}}
				\renewcommand{\l@subsection}
					{\@dottedtocline{2}{1.25em}{1.75em}}
				\renewcommand{\l@subsubsection}
					{\@dottedtocline{3}{2.75em}{2.6em}}
			\makeatother
\end{lstlisting}

Преамубла буде в окремому файлі \textit{preambule.tex}. Для вставлення інших файлів використовуєтсья команда \textit{input} або \textit{include}. Враховуючи титульну сторінку і конфіг класу, остаточний вигляд преамбули:

\begin{lstlisting}
			\documentclass[14pt,ukrainian,utf8, simple, 
			pointsection,floatsection ]{eskdtext} 
			
			\usepackage[figure,table]{totalcount} % counting figures, tables
\usepackage{eskdtotal} % total figures, tables...

\makeatletter
%format sections in tableofcontents
\renewcommand{\l@section}{\@dottedtocline{1}{0em}{1.25em}}
\renewcommand{\l@subsection}{\@dottedtocline{2}{1.25em}{1.75em}}
\renewcommand{\l@subsubsection}{\@dottedtocline{3}{2.75em}{2.6em}}
\makeatother

\usepackage{hyperref}
\hypersetup{
	colorlinks=false,
	hidelinks,
	bookmarks=true,
	,unicode=true
	,pdfcreator={XeLaTeX}
	,pdfa=true}

\renewcommand{\baselinestretch}{1.5} % Полуторный межстрочный интервал
\usepackage{graphicx}
\graphicspath{{images/}}

\usepackage[none]{hyphenat} % без переносів
\sloppy

% Формат подрисуночных надписей
\RequirePackage{caption}
\DeclareCaptionLabelSeparator{defffis}{ -- } % Разделитель
\captionsetup[figure]{justification=centering,
labelsep=defffis, format=plain} % Подпись рисунка по центру
\captionsetup[table]{justification=raggedleft, labelsep=defffis, 
format=plain, singlelinecheck=false} % Подпись таблицы справа
\addto\captionsukrainian{\renewcommand{\figurename}{Рисунок}}

\usepackage{float}
\usepackage{wrapfig}
\usepackage{subcaption}
\usepackage{array,tabularx,tabulary,booktabs}
\newcommand{\newSection}[3]{
	\newpage
	\section{\uppercase{#1}}
	\label{#2}
	\ESKDcolumnI{#3#1}
	\ESKDthisStyle{formII}
}
\newcommand{\anonsection}[2]{
	\newpage
	\phantomsection
	\addcontentsline{toc}{section}{\uppercase{#1}}
	\section*{\uppercase{#1}}
	\ESKDcolumnI{\uppercase{#1}}
	\label{#2}
	\ESKDthisStyle{formII}
}
% Списки
\usepackage{enumitem}
\setlist[enumerate,itemize]{leftmargin=1.5cm} % Отступы в списках
\setlist{nosep} % no separations

\usepackage{listings} % Оформление исходного кода
\lstset{
basicstyle=\small\ttfamily, % Размер и тип шрифта
breaklines=true,            % Перенос строк
tabsize=2,                  % Размер табуляции
extendedchars=\true,
keepspaces=true,
%frame=single,               % Рамка
literate={--}{{-{}-}}2,     % Корректно отображать двойной дефис
literate={---}{{-{}-{}-}}3,  % Корректно отображать тройной дефис
texcl=true,
}
\newcommand{\addimg}[4]{ 
	\begin{figure}
		\centering
		\includegraphics[width=#2\linewidth]{#1}
		\caption{#3} \label{#4}
	\end{figure}
}
\newcommand{\addimghere}[4]{ 
	\begin{figure}[H]
		\centering
		\includegraphics[width=#2\linewidth]{#1}
		\caption{#3} \label{#4}
	\end{figure}
}
\newcommand{\addtwoimghere}[4]{
	\begin{figure}[H]
		\centering
		\begin{subfigure}[t]{0.45\textwidth}
			\includegraphics[width=\textwidth]{#1}

			%\subcaption{}\label{sub:2a}
		\end{subfigure}
		\begin{subfigure}[t]{0.45\textwidth}
			\centering
			\includegraphics[width=\textwidth]{#2}
			%	\subcaption{}\label{sub:2b}
		\end{subfigure}
		\caption{#3}\label{#4}

	\end{figure}
}



			
			\usepackage{keyval}% http://ctan.org/pkg/keyval
\usepackage{lmodern} % font-size

\usepackage{multirow}
\newcolumntype{C}[1]{>{\centering\arraybackslash}m{#1}}
%\usepackage{enumerate}

\usepackage{fancyhdr} % Колонтитулы
\pagestyle{fancy}

\fancypagestyle{firststyle}{
	\renewcommand{\headrulewidth}{0pt}
	\fancyfoot{}
	\cfoot{Тернопіль 2020}
}
\renewcommand{\headrulewidth}{0pt}
\fancyfoot{}
\fancyhead{}

\makeatletter

\define@key{titlee}{university}{\def\tl@university{#1}}
\define@key{titlee}{katedra}{\def\tl@katedra{#1}}
\define@key{titlee}{type}{\def\tl@type{#1}}
\define@key{titlee}{discipline}{\def\tl@discipline{#1}}
\define@key{titlee}{thema}{\def\tl@thema{#1}}
\define@key{titlee}{kurs}{\def\tl@kurs{#1}}
\define@key{titlee}{group}{\def\tl@group{#1}}
\define@key{titlee}{specialty}{\def\tl@specialty{#1}}
\define@key{titlee}{author}{\def\tl@author{#1}}
\define@key{titlee}{posada}{\def\tl@posada{#1}}
\define@key{titlee}{kerivnyk}{\def\tl@kerivnyk{#1}}
\define@key{titlee}{pidpys}{\def\tl@pidpys{#1}}

% zavdannia
\define@key{titlee}{semestr}{\def\tl@semestr{#1}}
\define@key{titlee}{date}{\def\tl@date{#1}}
\define@key{titlee}{fulldate}{\def\tl@fulldate{#1}}
%kalendar
\define@key{titlee}{enddate}{\def\tl@enddate{#1}}
\define@key{titlee}{sources}{\def\tl@sources{#1}}
\define@key{titlee}{zapyska}{\def\tl@zapyska{#1}}
\define@key{titlee}{graphika}{\def\tl@graphika{#1}}

\setkeys{titlee}{university= University ,katedra= Kафедра,
type= Курсова робота ,thema= {Тема \\ \ } , discipline=Предмет, kurs=№, 
group=ke-4, specialty=122 - CS, author=\qquad \qquad \qquad
\qquad,posada=,kerivnyk=Teacher, semestr=4, date=,
sources=sources, graphika=,enddate=,fulldate=, zapyska={\ \quad \\ \ \\
\ \\ \ \\ \ \\ \ },pidpys=}%


\newcommand{\setzavdannia}[1][]{
	\begingroup%
	\setkeys{titlee}{#1}% Set new keys

	{\linespread{1}\selectfont
		\begin{center}
			Міністерство освіти і науки України\\
			\tl@university
		\end{center}
		\raggedright
		\setlength{\unitlength}{1cm}
		\begin{picture}(0,0)
			\put(2.45,-0.66){\line(1,0){15.05}}
			\put(2.7,-1.26){\line(1,0){14.8}}
			\put(0,-1.88){\line(1,0){17.5}}
			\put(3.15,-2.54){\line(1,0){14.35}}
		\end{picture}

		Кафедра \hspace{1cm} \tl@katedra

		Дисципліна \hspace{1cm} \tl@discipline

		Спеціальність \hspace{1cm} \tl@specialty

		\hspace{-0.35cm}
		\begin{tabular}{lp{0.05\linewidth}cC{0.1\linewidth}
		cC{0.05\textwidth}}

			Курс & \centering \tl@kurs & Група & \tl@group & Семестр 
			& \tl@semestr \\
			\cline{2-2} \cline{4-4} \cline{6-6}
		\end{tabular}
		\vspace{1.5cm}
		\begin{center}
			ЗАВДАННЯ\\
			на курсову роботу\\
			Студента\\
			\underline{\tl@author}\\
			{\small (прізвище, ім’я, по батькові)}
		\end{center}

		%\setlength{\unitlength}{1cm}
		\begin{picture}(0,0)
			\put(4.1,-0.66){\line(1,0){13.8}}
			\put(0.75,-1.26){\line(1,0){17.15}}

			\put(0.75,-1.88){\line(1,0){17.15}}
			\put(10,-3.1){\line(1,0){7.9}}
			\put(6.1,-3.7){\line(1,0){11.8}}
			\put(0.75,-4.36){\line(1,0){17.15}}
			\put(3.4,-5.55){\line(1,0){14.5}}
			\put(0.75,-6.2){\line(1,0){17.15}}
			\put(0.75,-6.84){\line(1,0){17.15}}
			\put(0.75,-7.46){\line(1,0){17.15}}
			\put(0.75,-8.01){\line(1,0){17.15}}
			\put(0.75,-8.68){\line(1,0){17.15}}
			\put(0.75,-9.3){\line(1,0){17.15}}

		\end{picture}
		\begin{enumerate}[label={\arabic*.},leftmargin=1cm]
			\item Тема роботи: \tl@thema

			      \vspace{0.66cm}

			\item Строк здачі студентом закінченої роботи \quad  \tl@enddate
			\item Вихідні дані до роботи: \tl@sources
			      \vspace{0.66cm}
			\item Зміст розрахунково - пояснювальної записки (перелік питань, 
			які підлягають розробці): \tl@zapyska

			      \begin{picture}(0,0)

				      \put(-0.2,-1.28){\line(1,0){17.15}}
				      \put(4.75,-1.88){\line(1,0){12.15}}

			      \end{picture}
			\item  Перелік графічного матеріалу (із точним зазначенням
			 обов’язкових креслень): \tl@graphika

			\item Дата видачі завдання: \quad \tl@date
		\end{enumerate}
	}
	\endgroup%
}
\newcommand{\settitle}[2][]{%

	\begingroup%
	\setkeys{titlee}{#1}% Set new keys
	{\linespread{1}\selectfont
		\centering
		Міністерство освіти і науки України\\
	
		\tl@university
		\hrule
	
		{\scriptsize (повне найменування вищого навчального закладу)}

		\hspace{0.2cm}
		Кафедра \tl@katedra
		\hrule
	
		{\scriptsize (повна назва кафедри)}

		\vspace{3cm}
		\begin{center}
			\textbf{ \large \tl@type}
		\end{center}

		з  \textbf{<<\tl@discipline>>}

		\hrule
		%	\vspace{0.2cm}
		{\scriptsize (назва дисципліни)}
		\hspace{0.2cm}


		\setlength{\unitlength}{1cm}
		\begin{picture}(0,0)
			\put(-9,-1.85){\line(1,0){18}}
			\put(-9,-0.75){\line(1,0){18}}
			\put(-9,-1.3){\line(1,0){18}}

		\end{picture}


		на тему : \textbf{<<\tl@thema>>}

	}
	{\linespread{1.2}\selectfont
		\vspace{5cm}
		\hfill
		\begin{minipage}{0.5\linewidth}


			\begin{tabular}{lp{0.1\linewidth}ccp{0.145\linewidth}}
				Студента & \centering \tl@kurs & курсу, & групи & \tl@group \\
				\cline{2-2} \cline{5-5}
			\end{tabular}

			\vspace{0.1cm}
			\begin{tabular}{lc}
				спеціальності & \tl@specialty \\
				\hline
				\tl@author    & \tl@pidpys    \\
				\hline
			\end{tabular}
			\vspace{-0.8cm}\hspace{0.3cm}	{ \centering\scriptsize (прізвище
			та ініціали)  \hspace{2cm}(підпис студента)} \\

			\begin{tabular}{p{0.3\textwidth}p{0.6\textwidth}}
				Керівник: & \tl@posada \\
				\cline{2-2}
				\multicolumn{2}{c}{\tl@kerivnyk}\\
				\hline
			\end{tabular}
			\centering
			{\scriptsize (посада, вчене звання, науковий ступінь, прізвище
			та ініціали) }

			\setlength{\unitlength}{1cm}
			\begin{picture}(0,0)
				\put(5.5, 0){\line(1,0){3.45}}
				\put(7.35, -0.7){\line(1,0){1.6}}
				\put(3, -0.7){\line(1,0){1.6}}
				\put(-2.3, -1.65){{\fontsize{10}{10} {\selectfont Члени комісії
				:}}}
			\end{picture}
			\raggedright
			{\fontsize
			{10}{10} \selectfont Оцінка за національною шкалою:}

			\begin{tabular}{lp{0.15\textwidth}lp{0.15\textwidth}}
				\fontsize
				{10}{10} \selectfont Кількість балів: &   & \fontsize{10}{10}
			 \selectfont Оцінка ECTS &
			\end{tabular}

			%\hspace{-2.2cm}\vspace{0.2cm}
			\hspace{0.2cm}
			\begin{tabular}{p{0.26\textwidth}p{0.001\textwidth}
			C{0.56\textwidth}}
				  &   & \tl@kerivnyk \\
				\cline{1-1} \cline{3-3}
			\end{tabular}
			\vspace{-0.89cm}	\centering

			{\scriptsize \hspace{-0.7cm}	(підпис)  \hspace{2.4cm}   (прізвище
			 та ініціали) }

			\hspace{0.2cm}
			\begin{tabular}{p{0.26\textwidth}p{0.001\textwidth}
			C{0.56\textwidth}}
				  &   &   \\
				\cline{1-1} \cline{3-3}
			\end{tabular}
			\vspace{-0.89cm}	\centering

			{\scriptsize \hspace{-0.7cm}	(підпис)  \hspace{2.4cm}   (прізвище
			 та ініціали) }

			\hspace{0.2cm}
			\begin{tabular}{p{0.26\textwidth}p{0.001\textwidth}C
			{0.56\textwidth}}
				  &   &   \\
				\cline{1-1} \cline{3-3}
			\end{tabular}
			\vspace{-0.89cm}\centering

			{\scriptsize \hspace{-0.7cm}	(підпис)  \hspace{2.4cm}   (прізвище
			та ініціали) }
		\end{minipage}

	}


	\thispagestyle{firststyle} % колонтитул

	\endgroup%

}

\newcommand{\setkalendar}[2][]{
	\newgeometry{right=3cm,left=1cm}

	\begingroup%
	\setkeys{titlee}{#1}% Set new keys
	{\linespread{1}\selectfont
		\begin{center}
			\textbf{Календарний план}
		\end{center}
		\centering
		\begin{tabulary}{1.0\textwidth}{|C{1cm}|L|C{3cm}|C{2.5cm}|}
			\hline
			\fontsize{10}{10} \selectfont № п/п &
			 \fontsize{10}{10} \selectfont Назва етапів курсового проекту
			( роботи )& \fontsize{10}{10} \selectfont Строк виконання етапів
			проекту (роботи) & \fontsize{10}{10} \selectfont Примітки \\
			\hline
			#2
			& & & \\ \hline
			& & & \\ \hline
			& & & \\ \hline
			& & & \\ \hline
			& & & \\ \hline
			& & & \\ \hline


		\end{tabulary}

		\vspace{1cm}\hspace{-0.5cm}
		\raggedright

		\begin{tabular}{p{2cm}p{5cm}p{2cm}C{5cm}}
			Студент &   &   & \tl@author \\
			\cline{2-2} \cline{4-4}
		\end{tabular}
		\centering

		{\scriptsize \hspace{3.5cm}	(підпис)  \hspace{5.4cm}   (прізвище, 
		ім’я, по батькові) }
		\begin{tabular}{p{2cm}p{5cm}p{2cm}C{5cm}}
			Керівник &   &   & \tl@kerivnyk \\
			\cline{2-2} \cline{4-4}
		\end{tabular}
		\centering

		{\scriptsize \hspace{3.5cm}	(підпис)  \hspace{5.4cm}   (прізвище,
		ім’я, по батькові) }

		\hspace{0.5cm}
		\raggedright
		\vspace{1cm}
		\begin{tabular}{c C{5cm} }
			Дата & \tl@fulldate \\
			\cline{2-2}
		\end{tabular}
	}
	\endgroup%
	\restoregeometry
}

\makeatother





			
			\usepackage{fontspec}
\usepackage{xecyr}
\usepackage{polyglossia}
\setmainlanguage{ukrainian}
\usepackage{xunicode, xltxtra}
\usepackage{cmap}	
\defaultfontfeatures{Ligatures=TeX}

\setmainfont{Times New Roman} 
\newfontfamily\cyrillicfont{Times New Roman}
\setotherlanguage{english}
\setmonofont{FreeMono}

%% Название документа
\ESKDtitle{\ESKDfontIII Створення шаблону для курсової в Latex}
\ESKDauthor{ ****** А.В. }
\ESKDchecker{****** О.Б. }

\ESKDsignature{ КПКН 20.055.014.000 ПЗ }
\ESKDcolumnIX{{\small ТНТУ, ФІС, КН ****}}

\renewcommand{\ESKDcolumnVIIname}{\ESKDfontII Аркуш}

\ESKDsectAlign{section}{Center}
\ESKDsectStyle{section}{\normalsize \bfseries \uppercase}
\ESKDsectStyle{subsection}{\normalsize \bfseries}
\ESKDsectSkip{section}{0pt}{0.8cm}
\ESKDsectSkip{subsection}{0.8cm}{0.5cm}
\ESKDsectSkip{subsubsection}{0.5cm}{0.1pt}

\end{lstlisting}

\subsection{Тіло документа}

В першу чергу необхідно застосувати створені команди для титульної сторінки і завдання з календарним планом, передаючи необхідні параметри:

\addimghere{6}{1}{Використання команд для перших 3 сторінок}{}

Команда \verb|\ESKDthisStyle{empty}| робить стиль сторінки без рамки. Для зручності ці команди буде винесено в окремий файл \textit{titlepage.tex}. Також розділи і все інше буде окремо:

\begin{lstlisting}
			\begin{document}
			
			\renewcommand{\ESKDcolumnVIIname}{\ESKDfontIII Аркуш}
			
\ESKDthisStyle{empty}
	
	\settitle[
	thema= Створення шаблону для курсової в Latex ,
	university=Тернопільський національний технічний університет імені Івана Пулюя,
	katedra=комп'ютерних наук,
	type=КУРСОВА РОБОТА,
	discipline={Комп’ютерні системи обробки текстової, графічної та мультимедійної
	інформації},
	kurs=3,
	group=****,
	specialty=122 Комп'ютерні науки,
	author=***********,
	kerivnyk=***********,
	]
	
	
	\newpage
	\ESKDthisStyle{empty}
	
	\setzavdannia[
	university=Тернопільський національний технічний університет імені Івана Пулюя,
	katedra= комп'ютерних наук,
	kurs= 3,
	discipline ={Комп’ютерні системи обробки текстової, графічної та мультимедійної інформації},
	specialty=  122 Комп'ютерні науки, 
	group= ****, 
	semestr= 6,
	author=******** Андрія Володимировича,
	thema= Створення шаблону для курсової в Latex \\ \ \\ \ \\ ,
	sources=,
	zapyska={Створення шаблону для курсової в Latex, Реферат, Зміст, Вступ, Розділ 1.Latex.Історія.Версії, Розділ 2. Інструментарій Latex. Розділ 3. Розробка шаблону для курсової, Розділ 4.
		Порівняння шаблону з іншими аналогами. Висновок, Список літературних
		джерел, Додатки \\ \ \\ \ \\ \  },
	graphika={Презентація – ХХ слайдів у форматі .PPTX, x додатки},
	date=01.05.2020,
	enddate=22.06.2020
	]

	\newpage
	\ESKDthisStyle{empty}
	\setkalendar[
		author=****** А.В.,
		kerivnyk=******* О.Б.,
		fulldate=1 травня 2020р.
	]{
	1 & Дата видачі індивідуального завдання & 01.05.2020 & Виконано \\ \hline
	2 & Підготовка до виконання курсової & 04.05.2020 & Виконано \\ \hline
	3&Формування структури курсової&08.04.2020&Виконано \\ \hline
	4&Збір інформації для про Latex&11.05.2020&Виконано \\ \hline
	5&Написання 1 розділу&13.05.2020&Виконано \\ \hline
	6&Написання 2 розділу& 16.05.2020&Виконано \\ \hline
	7&Створоення шаблону & 20.06.2020 & Виконано \\ \hline
	8&Написання 3 розділу&25.05.2020&Виконано \\ \hline
	9&Пошук інших шаблонів і порівняння&28.05.2020&Виконано \\ \hline
	10& Написання 4 розділу & 01.06.2020& Виконано \\ \hline
	11&Висновок курсової&05.06.2020&Виконано \\ \hline
	12&Захист курсової роботи& 22.06.2020 & \\ \hline
	  & & & \\ \hline
	    & & & \\ \hline
	      & & & \\ \hline
	        & & & \\ \hline
	               
	}

			\ESKDthisStyle{empty}

\begin{center}
	\uppercase{\textbf{Реферат}}
\end{center}
	
	Курсова робота // Створення шаблону для курсової в Latex// ******* Андрій Володимирович // Тернопільський національний
	технічний
	університет
	імені
	Івана
	Пулюя,
	факультет
	комп’ютерно-інформаційних систем та програмної інженерії, кафедра комп’ютерних наук,
	група **** // Тернопіль, 2020, сторінок \pageref{LastPage}, рисунків \totalfigures{} , джерел \ESKDtotal{bibitem} , таблиць \totaltables{}
	креслень 0, додатків \ESKDtotal{appendix}.
	
	Ключові слова: Latex, шаблон, шаблон для Latex, мова розмітки даних, Tex, типографія.
	
	В даній курсовій роботі було проведено опис систему для видавництва документів  Latex, та на його основі було створено шаблон для курсових робіт. Після створення здійснено порівняння з іншими шаблонами.
\newpage
			\ESKDthisStyle{formII}
			
			\tableofcontents
			\anonsection{Вступ}{sec:intro}
\ESKDthisStyle{formII}

Написання та оформлення текстів у високій якості потребує спеціального програмного забезпечення(ПЗ). Особливо важливо мати можливість використовувати системи набору тексту для написання документів, книг, статтей та інших форм видавничої справи в науковій галузі. Через необхідність написання складних текстів з красивим оформленням для наукових праць -- ця тема є актуальною для мене, оскільки, навчаючись в університеті потрібно вести звітність про виконання завдань, і найкращий спосіб вирішення цієї проблеми - це розробити шаблон для робіт, і сконцентруватися на самому завданні.

Багато ПЗ можна привести для прикладу, но ми зосередимося на наборі макророзширень системи комп'ютерної верстки \textbf{Latex}, яка розглядається в цій курсовій.

В ході наступних розділів буде розглянуто історію створення Latex, його основні можливості. Далі буде проводитись розробка шаблону для курсових робіт. Маючи такий шаблон, його можна буде легко модифікувати для використання в лабораторних роботах. В останньому розділі буде порівняння готового шаблону з іншими шаблонами, знайденими в мережі інтернет.

			\newSection{Історія Latex}{sec:seccc}{\ESKDfontV}


Говорячи про Latex потрібно вказати що він є, надбудовою над TEX - оригінальною системою текстового препроцесора. Все що можна зробити в Latex можна і в оригінальні системі TEX. Програмне забезпечення для набору тексту TEX було розроблено Дональдом Е. Кнутом в кінці 1970-х. Він був випущений з ліцензією з відкритим кодом і став стандартом наукового видавництва. Тепер TEX використовується для набору та публікації більшої частини світової інформації наукової літератури з фізики та математики.

Однією з найважливіших причин, по якій люди використовують LATEX, є те, що він відокремлює зміст документа від стилю. Це означає, що після написання вмісту вашого документа ми можемо легко змінити його вигляд. Так само ви можете створити один стиль документа, який можна використовувати для стандартизації зовнішнього вигляду безлічі різних документів. Це дозволяє науковим журналам створювати шаблони для публікацій. Ці шаблони мають заздалегідь зроблений макет, що означає, що потрібно додавати лише вміст.

Основний вплив для широкого розповсюдження здійснив
Леслі Лампорт у союзі з Пітером Гордоном в Аддісоні-Веслі,
версія 2.09 від приблизно середини 80-х років, яка походить від  системи TEX Дональда Кнута, яка досить швидко поширилася поза спільнотою північноамериканських математиків, які підтримували  розвиток TEX від його створення як один із "особистих інструментів продуктивності" Дона, створеного просто щоб забезпечити швидке завершення та типографічну якість його книги <<Мистецтво комп'ютерного програмування>> \cite{Knuth} Менш прямий, але, ймовірно, ширший вплив випливає з того, що вона є першою широко використовуваною мовою для опису логічної структури широкого кола документів таким \ чином \ її впровадження філософії логічного \ проектування, \ яке \newpage \noindent використовується Brain Reid in Scribe \cite{Reid}: "під час написання документа ви повинні
переймайтеся його логічним змістом, а не його візуальним оформленням."

Тоді Latex по-різному описувався як "TEX для мас" та "Написання, звільнене від негнучкого керування форматом ". Не зовсім
зрозуміло, чи все це було зроблено навмисне Леслі як особливість дизайну, але, безумовно, він не очікував що згодом, здійснить такий широкий вплив. Поширеність Latex була, навіть у кінці 1980-х років, більшою порівняно з більшістю некомерційного програмного забезпечення того часу. Хороші новини швидко поширювалися і 1994 року Леслі міг написати «Latex зараз надзвичайно популярний у науковій та академічній спільнотах, і він широко використовується у індустрії." Але цей рівень повсюдності все-таки був мізерним порівняно з сьогоднішнім днем, коли він став для багатьох професіоналів, невід'ємним інструментом, присутність якого є дуже важливою.

\subsection{Складові Latex(TeX)}

Робота з Latex пов'язана з так званими <<рівнями>> \cite{levels}:
\begin{enumerate}[label={\arabic*.},leftmargin=1.45cm]
	\item \textbf{Дистрибутиви}: MiKTeX, TeX Live,… Це великі колекції програмного забезпечення, пов'язаного з TeX, для завантаження та встановлення. Коли хтось каже: «Мені потрібно встановити TeX на свою машину», він зазвичай шукає дистрибутив.
	\item \textbf{Редактори}: Emacs, vim, TeXworks, TeXShop, TeXnicCenter, WinEdt,… ці редактори - це те, що ви використовуєте для створення файлу документа. Деякі (наприклад, TeXShop) присвячені спеціально TeX, інші (наприклад, Emacs) можуть використовуватися для редагування будь-яких файлів. Документи TeX не залежать від будь-якого конкретного редактора; сама програма набору TeX не включає редактор.
	\item \textbf{Компілятори}: TeX, pdfTeX, XeTeX, LuaTeX,… це виконувані бінарні файли, які реалізують різні варіанти TeX. Наприклад, pdfTeX реалізує прямий вихід у форматі PDF (якого немає в оригінальному TeX Кнута), LuaTeX забезпечує доступ до багатьох внутрішніх систем через вбудовану мову Lua і т.д. Коли хтось каже: "TeX не може знайти мої шрифти", вони зазвичай мають на увазі компілятор.
	\item \textbf{Формати}: LaTeX, звичайний TeX,… Це мови на основі TeX, якими фактично пишуть документи. Коли хтось каже, що "TeX дає мені невідому помилку", вони зазвичай мають на увазі формат. (До речі, "LaTeX" вже багато років означає "LaTeX2e".)
	\item \textbf{Пакети}: geometry, lm, ... Це доповнення до основної системи TeX, розроблені незалежно, надаючи додаткові функції набору тексту, шрифти, документацію тощо. Пакет може або не може працювати з будь-яким заданим форматом та/або компілятором; наприклад, багато є розроблено спеціально для LaTeX, але є і багато для інших. Сайти CTAN надають доступ до переважної більшості пакетів у світі TeX; CTAN, як правило, є джерелом, яке використовується дистрибутивами.
\end{enumerate}	

	Особливу увагу треба звернути на компілятори, в цій курсовій я використовую \textit{Xelatex}. Його особливістю є використання системний шрифтів, на відміну від Latex, який має свої вбудовані. Для оформлення документів в Україні згідно з ДСТУ 3008:2015 потрібно використовувати шрифт \textit{Times new Roman}, що і дозволяє зробити Xelatex. 
	
	\subsection{TeX і Latex}
	
	З моменту заснування розробка Latex та відповідного програмного забезпечення була повністю переплетена з розвитком самого TEX. 
	Хоча є багато сумнівів щодо корисністі деяких аспектів фундаментальних моделей та дизайну
	TEX як двигуна форматування тексту, він був тоді і залишається зараз (the
	початок тисячоліття) єдиним зрілим, широко доступним,
	програмованим і дуже гнучким компілятором для тексту. Таким чином для
	 Леслі в той час, як і для нас зараз, це єдиний вибір фундаменту
	для практичної автоматизації високоякісного форматування.
	
	У дизайні Latex Леслі свідомо дозволив основному компілятору TEX безпосередньо впливати на більшість текстових питань.
	У типових системах обробки тексту тієї епохи, включаючи TEX,
	основні методи обробки тексту документа такі: кожен вхідний маркер, що надсилається до системи, обробляється
	як складна імперативна команда. У таких системах "символ в
	тексті документа", як правило, подія на клавіатурі або маркер на
	вхідний буфер, не просто призначений викликати створення
	'елемента у рядку' в 'об'єкті текстового класу', такий 'рядок'
	врешті обробляється деяким іншим модулем системи, або
	навіть зовнішніми програмами.
	
	
	TEX був розроблений у цій імперативній парадигмі, оскільки це призводить до високоефективності(і в часі, і в просторі) машини, незважаючи на те, що "набір тексту" є для TEX відносно складним обчислювальним процесом, що включає, в першу чергу, оптимізацію	вибіру гліфа та позиціонування над цілими абзацами контрольоваий	за допомогою високо настроюваного алгоритму динамічного програмування. Однак, оскільки цей процес набору даних був оптимізований для швидкості, то робити що-небудь, що недоступно в рамках цього монолітного процесу (як визначено дизайном TEX), є важким у здійсненні та помітно неефективним у використанні. Такі процеси мають центральне значення для	якості набору і особливо важливі в наборі	інших мов, крім американської англійської. Вони включають модифікацію	важливих підпроцесів, таких як вибір гліфу (як для лігатур)	та їх розміри і розміщення; переноси та вирівнювання.
	
	
	\subsection{Версії}
	
	LaTeX2e - це поточна версія LaTeX, відколи вона замінила LaTeX 2.09 у 1994 році. Станом на 2019 рік LaTeX3, який розпочався створюватися на початку 1990-х років, досі розвивається. Планові функції включають покращений синтаксис, підтримку гіперпосилання, новий інтерфейс користувача, доступ до довільних шрифтів та нову документацію.\cite{latex3}
	
	Існують численні комерційні реалізації всієї системи TeX. Постачальники систем можуть додавати додаткові функції, такі як додаткові шрифти та підтримку по телефону. LyX - це безкоштовний, WYSIWYM-процесор візуального документа, який використовує LaTeX для бек-енду. TeXmacs - безкоштовний редактор WYSIWYG з аналогічними функціями, як LaTeX, але з іншим механізмом набору тексту. Інші редактори WYSIWYG, які використовують LaTeX, включають Scientific Word у MS Windows. Та BaKoMa TeX для Windows, Mac та Linux.
	
	Доступно декілька дистрибутивів TeX, що підтримуються спільнотою, зокрема TeX Live (багатоплатформна), teTeX (застаріла на користь TeX Live, UNIX), fpTeX (застаріла), MiKTeX (Windows), proTeXt (Windows), MacTeX (TeX Live з додаванням специфічних програм для Mac), gwTeX (Mac OS X) (застарілий), OzTeX (Mac OS Classic), AmigaTeX (більше не доступний), PasTeX (AmigaOS, доступний у сховищі Aminet) та Auto-Latex Equations (Додаток Google Docs, який підтримує команди MathJax LaTeX).
	
	\subsection{Сумісність та конвертація}
	
	Документи LaTeX (* .tex) можна відкрити будь-яким текстовим редактором. Вони складаються з простого тексту та не містять прихованих кодів форматування чи двійкових інструкцій. Крім того, документи TeX можна поширити у формат Rich Text (* .rtf) або XML. Це можна зробити за допомогою безкоштовних програм LaTeX2RTF або TeX4ht. LaTeX також може бути конвертовано у PDF-файли за допомогою розширення LaTeX pdfLaTeX. Файли LaTeX, що містять текст Unicode, можуть бути оброблені в PDF-файли за допомогою пакету \textit{inputenc} або розширеннь Tee XeLaTeX і LuaLaTeX.
	
	\begin{itemize}
		\item HeVeA - це перетворювач, написаний на Ocaml, який перетворює документи LaTeX у HTML5. Він ліцензований відповідно до Q Public License.
		
		\item LaTeX2HTML - це перетворювач, написаний на Perl, який перетворює документи LaTeX в HTML. Таким чином, наприклад, наукові праці, головним чином набрані для друку, можна розмістити для перегляду в Інтернеті. Він ліцензований відповідно до GNU GPL v2.
		
		\item LaTeXML - це безкоштовне програмне забезпечення публічного домену, написане на Perl, яке перетворює документи LaTeX у різноманітні структуровані формати, включаючи HTML5, epub, jats, tei.
		\item Pandoc - це "універсальний конвертер документів", здатний трансформувати LaTeX у безліч різних форматів файлів, включаючи HTML5, epub, rtf та docx. Він ліцензований відповідно до GNU GPL v2.
	
	\end{itemize}

	LaTeX став стандартом для набору математичних виразів в наукових документах. Таким чином, існує кілька інструментів перетворення, орієнтованих на математичні вирази LaTeX, такі як перетворювачі в MathML  або Computer Algebra System.
	
	\begin{itemize}
		\item  Mathoid - це веб-конвертер який використовує Node.js, він перетворює математичні входи, такі як LaTeX, у формати MathML та зображення, включаючи SVG та PNG. Він використовується у Вікіпедії для відображення математики.
		\item TeXZillais перетворювач на JavaScript з LaTeX в MathML. Це один з найшвидших перетворювачів LaTeX в MathML.
		\item  LaCASt - це перетворювач, написаний на Java, який перетворює семантичний діалект LaTeX в Maple та Mathematica.
	\end{itemize}


	\subsection{Особливості Latex}
	
	Створення LaTeX документа це програмування: Ви створюєте текстовий файл в LaTeX-розмітці, макроси LaTeX обробляють його і видають конкретний документ.
	
	Такий підхід відрізняється від використовуваного в WYSIWYG (What You See Is What You Get - те, що ви бачите, то і отримуєте) програмах, таких, як Openoffice Writer або Microsoft Word.
	
	В LaTeХ: 
	
	\begin{itemize}
		\item Під час редагування документа Ви не можете (зазвичай) побачити його остаточний варіант.
		
		\item 	Вам, як правило, потрібно знати необхідні команди розмітки LaTeX.
		
		\item Інколи складно отримати необхідний вигляд документа.
		
	\end{itemize}


		З іншого боку, у LaTeX є і переваги:
		
	\begin{itemize}
		\item Оформлення тексту відокремлено від вмісту. Ви повною мірою зосереджуєтеся на структурі та вмісті документу і забуваєте про те, як буде виглядати друкований варіант.
		\item Стиль, шрифти, оформлення таблиць і малюнків т. д. узгоджено у всьому документі.
		\item Одне і те саме оформлення можна використовувати для будь-якого числа документів.
		\item Легко набирає математичні формули.
		\item Легко створюються алфавітні вказівники, посилання та бібліографічні списки.
		\item Більші документи можуть бути розподілені на декілька файлів і працювати з ними окремо, в тому числі з використанням системи управління версіями.
		\item Вам не потрібно вручну налаштовувати шрифти, розмір тексту, високий шрифт - за це відповідає  LaTeX.
		\item  Вам доведеться правильно структурувати ваш документ.
		\item Файли з вихідними текстами можна переглянути і змінити в любому текстовому редакторі.
	
	\end{itemize}

	Підхід LaTeX до створення документа можна назвати WYSIWYM (What You See Is What You Mean - що бачиш, то і думаєш): під час набору тексту Ви не бачите остаточний варіант документа, тільки логічну структуру цього документа. Про оформлення замість Вас подбає LaTeX.

	
		
			\newSection{Інструментарій Latex}{sec:sec2}{\ESKDfontV}

Почати роботу з Latex необхідно зі створення нового документа з розширенням \textit{.tex}, при умові що у Вас встановлені всі необхідні пакети Latex. Як вже було сказано раніше треба мати встановлений дистрибутив та редактор. В дистрибутиві будуть знаходитися пакети та компілятори для створення документів. В якості дистрибутива встановлено \textit{texlive}, а редактор вихідного документа \textit{TexStudio}. TexStudio надає зручні можливості для роботи з  командами,наявні автодоповнення, сполучення клавіш - це все робить його  середовищем  для розробки Latex.

\subsection{Перший документ}

Створюємо файл з розширенням .tex та записуємо в нього наступні рядки:

\begin{lstlisting}
	\documentclass{article}
	
	\begin{document}
	
	\end{document}
\end{lstlisting}

На виході буде пустий документ. Перший рядок вказує на клас документу.Форматування за замовчуванням у документах LATEX визначається класом, який використовується цим документом. Стандартний вигляд можна змінити, а додаткові функції можна додати за допомогою пакета. Імена файлів класу мають розширення .cls, імена файлів пакунків мають розширення .sty.

Для різних типів документів потрібні різні класи, тобто для резюме буде потрібен інший клас, ніж для  наукового документа. У цьому випадку клас - це \textit{article}, найпростіший і найпоширеніший клас LATEX. \ Інші \ типи \ документів, \ над \newpage \noindent якими ви можете працювати, можуть вимагати різних класів, таких як \textbf{book} або \textit{report}.

Далі з команди \textit{begin} починається так зване тіло документа. В ньому буде знаходитися вміст документа, аж до \textit{end}

Щоб переглянути документ треба провести компіляцію. До прикладу такою командою:

\begin{lstlisting}
	pdflatex <your document> 
\end{lstlisting}

\subsection{Преамбула документа}

У попередньому прикладі текст був введений після команди  \verb|\begin {document} |. Все у .tex-файлі до цього моменту називається преамбулою. У преамбулі ви визначаєте тип документа, який ви пишете, мову, якою ви пишете, пакети, які ви хочете використовувати та декілька інших елементів. Наприклад, звичайна преамбула документа виглядатиме так:

\begin{lstlisting}
	\documentclass[12pt, letterpaper]{article}
	\usepackage[utf8]{inputenc}
\end{lstlisting}

\verb|\documentclass[12pt, letterpaper]{article}|. Як було сказано раніше, це визначає тип документа. Деякі додаткові параметри, що входять до квадратних дужок, можуть передаватися команді. Ці параметри повинні бути розділені комами. У прикладі додаткові параметри встановлюють розмір шрифту \textit{(12pt)} та розмір паперу (letterpaper). Звичайно, можна використовувати інші розміри шрифту \textit{(9pt, 11pt, 12pt)}, але якщо не вказано жодного, розмір за замовчуванням - 10pt. Що стосується розміру паперу, інші можливі значення - це папір формату А4 та legalpaper; 

\verb|\usepackage[utf8]{inputenc}|. Це кодування документа. Його можна опустити або змінити на інше кодування, але рекомендується utf-8. Якщо вам конкретно не потрібно інше кодування, або якщо ви не впевнені в цьому, додайте цей рядок до преамбули.


Для форматування тексту, задання формату сторінки, додавання графічних елементів та задання всіх можливих параметрів і налаштувань використовуються різні команди, які можна задати в класі документу або використовуючи додаткові пакети. 

\subsection{Макет сторінки}

За замовчуванням всі параметри документа встановлюють класи документів. Однак,
якщо ви хочете змінити ці параметри , є команди, які дозволяють вам це зробити. Команди, що контролюють функції, що стосуються всього документа, повинні бути розміщені в преамбулі.

\subsubsection{Параграфи}\label{par}

Щоб почати новий абзац, залиште порожній рядок або використовуйте команду \verb|\par|. Команди \verb|\parindent| і \verb|\parskip| задяють відступ абзацу та розділення абзацу. 

Для задання міжрядкового інтервалу можна використати команду \verb|\renewcommand{\baselinestretch}{1.5}|, яка встановить його на 1.5.

Абзацний відступ задається командою \verb|\parindent 1.25cm|.\par Перейти на новий рядок \verb|\\ або \par|

\subsubsection{Вирівнювання тексту}

За замовчуванням LATEX вирівнює ваш текст горизонтально, так що лівий і правий відступи є гладкими. Якщо ви віддаєте перевагу "вирівнювання справа" , ви можете використовувати:
\verb|\raggedright|
Зауважте, що це має побічний ефект для відступів абзацу. Якщо ви хочете залишити відступ абзаців, потрібно спеціально задати його (тобто \verb|\parindent = 1.5em|) після \verb|\raggedright| команди.

\subsubsection{Коментарі}

Як і будь-який код, який ви пишете, часто корисно включати коментарі. Коментарі - це фрагменти тексту, які ви можете включити в документ, які не будуть надруковані, і жодним чином не вплинуть на документ. Вони корисні для організації вашої роботи, ведення приміток або коментування рядків / розділів під час налагодження. Щоб зробити коментар у LATEX, просто напишіть символ \verb|%| на початку рядка.

\subsubsection{Вигляд тексту}

Зараз ми розглянемо кілька простих команд форматування тексту.

\begin{itemize}
	\item Жирний: Жирний текст у LaTeX пишеться командою \verb|\textbf {...}|
	\item Курсив: Курсивний текст у LaTeX пишеться командою \verb|\textit {...}|
	\item Підкреслення: Підкреслений текст у LaTeX пишеться командою \verb|\underline {...}|
\end{itemize}

LaTeX має кілька команд-модифікаторів розміру шрифту (від найбільших до найменших):

\begin{lstlisting}
		\Huge
		\huge
		\LARGE
		\Large
		\large
		\normalsize (default)
		\small
		\footnotesize
		\scriptsize
		\tiny
\end{lstlisting}

\subsubsection{Додавання зображень, таблиць}

Стандартний комплект графіки LaTeX включає два пакети для імпорту графіки:
\textit{graphics} та \textit{graphicx}. Розширений пакет, \textit{graphicsx}, забезпечує більш зручний спосіб подачі параметрів і рекомендується. Тому перший крок - це введіть у свою преамбулу команду:
\verb|\usepackage {graphicsx}|. Цей пакет визначає нову команду під назвою \verb|\includegraphics|, яка дозволяє вам вказувати назву графічного файлу, а також надавати необов'язкові аргументи для масштабування чи обертання. Отже, для прикладу потрібно вставити графіку (названу, наприклад, myfigure.eps
або myfigure.pdf) використовуйте команду \verb|\includegraphics|, наприклад:

\begin{lstlisting}
	\includegraphics[width=4in]{myfigure}
\end{lstlisting}


Tabular середовище є стандартним в LATEX для створення таблиць. Ви повинні вказати параметр для цього середовища, в цьому випадку {c c c}. Це говорить про те, що LATEX буде три стовпці і текст у кожному з них повинен бути в центрі. Ви також можете використовувати r, щоб вирівняти текст праворуч, а l - для вирівнювання ліворуч. Символ \& використовується для визначення розділення у записах таблиці. У кожному рядку завжди повинно бути на один менше символів розділення, ніж кількість стовпців. Для переходу до наступного рядка таблиці використовуємо команду нового рядка \verb|\\|.

\begin{lstlisting}
	\begin{tabular}{ c c }
		cell 1 & cell 2 \\
		cell 3 & cell 4 
	\end{tabular}
\end{lstlisting}

Ви можете додати межі за допомогою команди горизонтальної лінії \verb|hline| та параметра вертикальної лінії |.

\begin{lstlisting}
	\begin{tabular}{ |c|c| }
		\hline 
		cell 1 & cell 2 \\ \hline
		cell 3 & cell 4 \\ \hline
	\end{tabular}
\end{lstlisting}


{| c | c | c | }: Тут вказується, що у таблиці будуть використані три стовпчики, розділені вертикальною лінією. | символ вказує, що ці стовпці повинні бути розділені вертикальною лінією. 

\verb|hline|: буде вставлена горизонтальна лінія. Тут ми ввели горизонтальні лінії вгорі та внизу таблиці. Немає обмежень у кількості разів, коли ви можете використовувати \verb|\hline|.

\subsubsection{Структурування}

Команди для організації документа відрізняються залежно від типу документа, найпростішою формою організації є секціонування, доступне у всіх форматах.

Команда \verb|section{}| позначає початок нового розділу, всередині дужок встановлюється заголовок. Нумерація розділів є автоматичною і її можна відключити, включивши * в команду розділу як \verb|\section*{}|. Ми також можемо мати підрозділи \verb|\subsection{}|, і пункти  \verb|sububsection{}|. 

\subsubsection{Підписи до зображень, таблиць}\label{capt}

В \textit{figure} або \textit{table}
середовищах, ви можете надати підпис із командою \verb|\caption {caption text}|. Зазвичай підписи для таблиці вводяться над таблицею, а підпис до зображення
нижче зображення.

Якщо потрібно якось модифікувати вигляд підпису то можна скористатися пакетом \textit{caption}. Наприклад відцентрувати підпис зображення, та змінити слово Рис. на Рисунок:

\begin{lstlisting}
		\captionsetup[figure]{justification=centering, labelsep=defffis,
		format=plain}		
		\addto\captionsukrainian{\renewcommand{\figurename}{ Рисунок }}
\end{lstlisting}


\subsection{Створення та перевизначення команд}

Щоб додати свої власні команди, використовуйте команду:

\begin{lstlisting}
		\newcommand{\name}[num]{definition}
\end{lstlisting}


В основному, команда вимагає двох аргументів: ім'я команди, яку ви хочете створити, і визначення команди. Зауважте, що ім'я команди можна, але не потрібно вкладати в дужки, як вам подобається. Аргумент num у квадратних дужках є необов’язковим і визначає кількість аргументів, які приймає нова команда (можливо до 9). Якщо він відсутній, він за замовчуванням дорівнює 0, тобто аргумент не дозволений.

Для прикладу створимо команду додавання зображення:

\begin{lstlisting}
		\newcommand{\addimg}[4]{
			\begin{figure}
				\centering
				\includegraphics[width=#2\linewidth]{#1}
				\caption{#3} \label{#4}
			\end{figure}
		}
\end{lstlisting}

Команда отримує 4 аргументи, для використання аргументу вводять \# та номер параметра.

Щоб перевизначити команду треба ввести \verb|\renewcommand{cmd}{def}|. Ми вже використовували її для визначення підпису до малюнків на ст. \pageref{capt} та для міжрядкового інтервалу ст. \pageref{par}

Ще багато команд можна було б описати, проте цього достатньо для початку. В наступному розділі буде відбуватись створення шаблону, в процесі використань нових команд буде надано  їхній короткий опис.


 
			\newSection{Розробка шаблону}{sec:sec3}{}

Почати роботу з виготовлення шаблону потрібно з вимог до завдання, а саме: потрібно розробити шаблон для курсових робіт з дотриманням ДСТУ 3008:2015. Ось деякі плавила оформлення:

\begin{itemize}
	\item Аркуші формату А4 (210х297 мм).
	\item Шрифт 14 розміру з 1,5 інтервалом.
	\item  віддаль між рядками повинна бути однакова і рівна 8-10 мм.
	\item  відстань між заголовками підрозділів або пунктів і подальшим або попереднім текстом 15-20 мм;
	\item  відстань між назвою розділу і назвою підрозділу або пункту 18-22 мм;
	\item  абзацний відступ повинен бути однаковим впродовж усього тексту записки і дорівнювати 10-15 мм.
\end{itemize}

Ще одною умовою є додавання рамок до роботи. І останній пункт, додатковий - зверстати титульну сторінку.

\subsection{Написання преамбули}

\subsubsection{Клас документа та кодування}

Для звичайної курсової роботи, без рамок, можна було б скористатися класом документа \textit{article} або \textit{extarticle}, проте довелося б реалізовувати рамки вручну. На просторах інтернету можна знайти колекцію пакетів \textit{eskdx}, який надає нам такий функціонал. Колекція пакетів надає 3 класи: \textit{eskdtext}(для текстової документації), \textit{eskdgraph}(для креслення схем) і \textit{eskdtab}(для документів, разбитих на графи). Для нашого випадку підходить eskdtext, його і використаємо.

\newpage

Ось як буде виглядати підключення класу:

\begin{lstlisting}
		\documentclass[14pt,ukrainian,utf8, simple, pointsection,
		floatsection ]{eskdtext} 
\end{lstlisting}

В додаткових параметрах вказано 14 розмір шрифту, українську мову(вибір тільки з 2, інша - російська), кодування, simple - відображати тільки основні графи, останні два параметри вказують на нумерування пунктів та фігур в межах секцій.

Для роботи з кирилицею потрібно підключити мовний пакет. На вибір є 2 основних: \textit{babel} та \textit{poliglossia}, скористаємося другим. Перевага другого пакету в тому що кириличні символи кодуються правильно, навідміну від babel, який робить заміну \textit{і} на латиницю. Це допоможе при проходженні роботи на антиплагіат.

Підключаємо його та ще пару пакетів для кодування кирилиці:

\begin{lstlisting}
		\usepackage{fontspec}
		\usepackage{xecyr}
		\usepackage{polyglossia}
		\setmainlanguage{ukrainian}
		\usepackage{xunicode, xltxtra}
\end{lstlisting}


\subsubsection{Специфічні налаштування класу}


Пакети ESKDX надають багато налаштувань, їх ми будемо зберігати в окремому файлі з назвою \textit{ESKDXconfig.tex}. Він є частиною преамбули тож і підключатиметься там. В цьому файлі зберігатимуться попередні команди і також для налаштування шрифтів.

\begin{lstlisting}
		\defaultfontfeatures{Ligatures=TeX}
		\setmainfont{Times New Roman} 
		\newfontfamily\cyrillicfont{Times New Roman}
		\setotherlanguage{english}
		\setmonofont{FreeMono}
\end{lstlisting}

Вказавши деякі команди можна створити титульну сторінку, проте її формати не підходить, тому ми будемо писати свою титулку в наступному підрозділі. Зараз тільки вкажемо деякі необхідні команди для задання інформації на рамках.
\begin{lstlisting}
			\ESKDsignature{ КПКН 20.055.014.000 ПЗ
			\ESKDcolumnIX{{\small ТНТУ, ФІС, КН СН-31}}
			\ESKDtitle{\ESKDfontIII Створення шаблону для курсової в Latex}
			\ESKDauthor{ **** А. В. }
			\ESKDchecker{ **** О. Б. }
			
\end{lstlisting}

Останнє в цьому файлі це налаштування стилів відображення секцій:

\begin{lstlisting}
			\ESKDsectAlign{section}{Center}
			\ESKDsectStyle{section}{\normalsize \bfseries \uppercase}
			\ESKDsectStyle{subsection}{\normalsize \bfseries}
			\ESKDsectSkip{section}{0pt}{0.8cm}
			\ESKDsectSkip{subsection}{0.8cm}{0.5cm}
			\ESKDsectSkip{subsubsection}{0.5cm}{0.1pt}
\end{lstlisting}

Секції центруємо, робимо 14 шрифтом, жирний та все у верхньому регістрі. Підсекції - те саме, тільки у звичайному регісті. Також налаштовано відступи між секціями.

Для цього файлу - це все, повний обсяг буде наведено в додатках.

\subsubsection{Створення титульної сторінки}

Налаштування титульної сторінки будуть знаходитися в окремому файлі \textit{title.tex}.

Спершу налаштуємо колонтитули, оскільки це найлегше що можна зробити зараз:
\begin{lstlisting}
		\usepackage{fancyhdr} % Колонтитули
		\pagestyle{fancy}
		
		\fancypagestyle{firststyle}{
		\renewcommand{\headrulewidth}{0pt}
		\fancyfoot{}
		\cfoot{Тернопіль 2020}
		}
		\renewcommand{\headrulewidth}{0pt}
		\fancyfoot{}
		\fancyhead{}
\end{lstlisting}

Використали пакет \textit{fancyhdr} і створили стиль колонтитула для сторінки в якому по центрі внизу записали необхідні дані. 

Створимо нову команду для титульної сторінки, яка буде приймати аргументи для їх встановлення на сторінку: найменування вищого навчального закладу, кафедра, назва роботи, тема, дисципліна, і т.д. Також буде створено ще 2 команди для завдання та календарного плану. Команди можуть приймати до 9 параметрів, в нашому випадку їх є більше, тому скористаємося пакетом \textit{keyval}, який дозволить використовувати опційні параметри, які не обмежуються кількістю.
Оголошення змінних та задання стандартних значень:

\begin{lstlisting}
			\define@key{titlee}{university}{\def\tl@university{#1}}
			\define@key{titlee}{katedra}{\def\tl@katedra{#1}}
			\define@key{titlee}{type}{\def\tl@type{#1}}
			\define@key{titlee}{discipline}{\def\tl@discipline{#1}}
			\define@key{titlee}{thema}{\def\tl@thema{#1}}
			\define@key{titlee}{kurs}{\def\tl@kurs{#1}}
			\define@key{titlee}{group}{\def\tl@group{#1}}
			\define@key{titlee}{specialty}{\def\tl@specialty{#1}}
			\define@key{titlee}{author}{\def\tl@author{#1}}
			\define@key{titlee}{posada}{\def\tl@posada{#1}}
			\define@key{titlee}{kerivnyk}{\def\tl@kerivnyk{#1}}
			\define@key{titlee}{pidpys}{\def\tl@pidpys{#1}}
			
			% zavdannia
			\define@key{titlee}{semestr}{\def\tl@semestr{#1}}
			\define@key{titlee}{date}{\def\tl@date{#1}}
			\define@key{titlee}{fulldate}{\def\tl@fulldate{#1}}
			%kalendar
			\define@key{titlee}{enddate}{\def\tl@enddate{#1}}
			\define@key{titlee}{sources}{\def\tl@sources{#1}}
			\define@key{titlee}{zapyska}{\def\tl@zapyska{#1}}
			\define@key{titlee}{graphika}{\def\tl@graphika{#1}}
			
			\setkeys{titlee}{university= University ,katedra = Kафедра,
			type= Курсова робота ,thema= {Тема \\ \ } , discipline = Предмет,
			kurs=№, group=ke-4, specialty=122 - CS, author=\qquad \qquad
			\qquad \qquad,posada=,kerivnyk=Teacher, semestr=4, date=,
			sources=sources, graphika=,enddate=,fulldate=, zapyska={\
			\quad \\ \ \\ \ \\ \ \\ \ \\ \ }, pidpys=}%			
\end{lstlisting}


Створюємо нові команди, всі будуть приймати один опційний аргумент, який потім в \verb|\setkeys{}|встановиться у відповідні змінні. Для використання змінних потрібно щоб вони перебували в групі, це оголошується \verb|\begingroup%|:

\begin{lstlisting}
			\newcommand{\setzavdannia}[1][]{
			\begingroup%
			\setkeys{titlee}{#1}% Set new keys
			...content
			\endgroup%
			}
\end{lstlisting}

Далі проведено опис основних моментів створення титульної сторінки, для детальнішого огляду див. додатки.

Починаємо зверху, де треба вказати навч. заклад та кафедру:

\begin{lstlisting}
		\centering
		Міністерство освіти і науки України\\
		\tl@university
		\hrule
		{\scriptsize (повне найменування вищого навчального закладу)}
		\hspace{0.2cm}
		Кафедра \tl@katedra
		\hrule
		{\scriptsize (повна назва кафедри)}
\end{lstlisting}

Команда \verb|\centering| відцентрує текст. У 2 рядку встановлюємо на параметр на його місце та малюємо горизонтальну лінію на всю ширину - \verb|\hrule|. \verb|\scriptsize| зменшує шрифт на дуже маленький.

Результат:

\addimghere{3}{1}{Верхня частина титулки}{}

Наступним створюємо тип та назву роботи:

\begin{lstlisting}
			\vspace{3cm}
			\begin{center}
			\textbf{ \large \tl@type}
			\end{center}
			
			з  \textbf{<<\tl@discipline>>}
			\hrule
			
			{\scriptsize (назва дисципліни)}
			\hspace{0.2cm}
			\setlength{\unitlength}{1cm}
			
			\begin{picture}(0,0)
				\put(-9,-1.85){\line(1,0){18}}
				\put(-9,-0.75){\line(1,0){18}}
				\put(-9,-1.3){\line(1,0){18}}
			\end{picture}
			
			на тему : \textbf{<<\tl@thema>>}
			}			
					
\end{lstlisting}

Команди \verb|\vspace{}, \hspace{}| роблять відступ вертикально та горизонтально на вказану відстань. Середовище \verb|\begin{picture}(0,0)| Робить область для створення простих графіків, фігур. В дужках вказано її розміри, в нас нема розміру для зручності, оскільки будуть малюватися лінії серед тексту. 

Команда \textit{put} ставить певну фігуру за вказаними координатами, її можна використовувати тільки в середовищі \textit{picture}. \textit{Line} - фігура лінії, в дужках осі, в параметрі довжина.

Ось результат:

\addimghere{4}{1}{Назва роботи, тема}{}

Перейдемо до останньої частини, де вказується інформація про автора та керівника. 

\begin{lstlisting}
			\vspace{5cm}
			\hfill
			\begin{minipage}{0.5\linewidth}
				\begin{tabular}{lp{0.1\linewidth}ccp{0.145\linewidth}}
					Студента & \centering \tl@kurs & курсу, & групи & \tl@group
					\\
					\cline{2-2} \cline{5-5}
				\end{tabular}
				\vspace{0.1cm}
				\begin{tabular}{lc}
					спеціальності & \tl@specialty \\
					\hline
					\tl@author    & \tl@pidpys    \\
					\hline
				\end{tabular}
				\vspace{-0.8cm}\hspace{0.3cm}	{ \centering\scriptsize (прізвище 
				та ініціали)  \hspace{2cm}(підпис студента)} \\
				
				\begin{tabular}{p{0.3\textwidth}p{0.6\textwidth}}
					Керівник: & \tl@posada \\
					\cline{2-2}
					\multicolumn{2}{c}{\tl@kerivnyk}\\
					\hline
				\end{tabular}
				\centering
				{\scriptsize (посада, вчене звання, науковий ступінь, прізвище
				та ініціали) }
			\end{minipage}
\end{lstlisting}

\addimghere{5}{1}{Остання частина титульної сторінки}{}


Поєднання \textit{begin\{minipage\}} та \textit{hfill} робить окрему область сторінки на половину її ширини та розташовує її справа. Для організації розмітки використовується таблиця, командами \verb|\cline{2-2} \cline{5-5}|, можна підкреслити необхідні стовпці в рядку.

Поєднуючи таблиці та малювання ліній, було зроблено необхідні лінії для 3 сторінок: титулка, завдання. календарний план.

\subsection{Завершення преамбули}

Для роботи із зображеннями підключаємо пакет \textit{graphicx}, вказуємо їхнє розташування:

\begin{lstlisting}
			\usepackage{graphicx} % Вставка картинок 
			\graphicspath{{images/}}
\end{lstlisting}

Напишемо пару нових команд для вставлення картинок. Перша команда вставляє одне зображення в зручному для Latex місці, приймає 4 команди: назва, ширина, підпис, позначка. Друга команда робить все те саме, проте вставляє картинку так як вона є в тексті. Остання команда вставляє 2 зображення поруч з одним підписом до них.

\begin{lstlisting}
			\newcommand{\addimg}[4]{ % add one img
				\begin{figure}
					\centering
					\includegraphics[width=#2\linewidth]{#1}
					\caption{#3} \label{#4}
				\end{figure}
			}
			\newcommand{\addimghere}[4]{ % add img here
				\begin{figure}[H]
					\centering
					\includegraphics[width=#2\linewidth]{#1}
					\caption{#3} \label{#4}
				\end{figure}
			}
			\newcommand{\addtwoimghere}[4]{ % two img side by side
				\begin{figure}[H]
					\centering
					\begin{subfigure}[t]{0.45\textwidth}
						\includegraphics[width=\textwidth]{#1}
					\end{subfigure}
					\begin{subfigure}[t]{0.45\textwidth}
						\centering
						\includegraphics[width=\textwidth]{#2}
						
					\end{subfigure}
					\caption{#3}\label{#4}
				
				\end{figure}
			}
\end{lstlisting}

Створимо ще дві функції для роботи із секціями. Команда \textit{newSection} додає нову секція з нової сторінки, також є позначення для посилання на неї, останнє - це застосування розширеної рамки і вказанням того самого розділу. Друга команда схожа, тільки секції будуть без нумерації.
\begin{lstlisting}
			\newcommand{\newSection}[3]{
				\newpage
				\section{\uppercase{#1}}
				\label{#2}
				\ESKDcolumnI{#3#1}
				\ESKDthisStyle{formII}
			}
			
			\newcommand{\anonsection}[2]{
				\newpage
				\phantomsection
				\addcontentsline{toc}{section}{\uppercase{#1}}
				\section*{\uppercase{#1}}
				\ESKDcolumnI{\uppercase{#1}}
				\label{#2}
				\ESKDthisStyle{formII}
			}
\end{lstlisting}

Один дуже важливий нюанс. Пакет ESKDX не надає можливості зміни даних в рамках в тілі документа, як це зробили ми, це призведе до помилки. Один з варіантів вирішення є написання власного стилю рамки, проте це дуже важко і потребує поглиблених знань пакету. Другий варіант - зміна пари рядків коду в початкових файлах пакету. Під час компіляції можна побачити що викликається команда збереження рамки в преамбулі та застосовується без змін далі, тому треба замість застосування - заново намалювати. Отже у пакетному файлі \textit{eskdstamp.sty} потрібно знайти 2 форму рамок(\verb|\ESKD@stamp@ii@box|) та скопіювати код створення рамки в місце де застосовується збережена рамка. 

Налаштуємо підписи до малюнків та таблиць:

\begin{lstlisting}
			\RequirePackage{caption}
			\DeclareCaptionLabelSeparator{defffis}{ -- } % Розділювач
			\captionsetup[figure]{justification=centering, labelsep=defffis,
			 format=plain} % Підпис малюнка по центру
			\captionsetup[table]{justification=raggedleft, labelsep=defffis, 
			format=plain, singlelinecheck=false} % Підпис таблиці справа
			\addto\captionsukrainian{\renewcommand{\figurename}{Рисунок}} 
			% Ім' я фігури
\end{lstlisting}

Оформимо відображення початкового коду, використовуючи пакет \textit{listings}:

\begin{lstlisting}
			\usepackage{listings}
			\lstset{
				basicstyle=\small\ttfamily,
				breaklines=true,
				tabsize=2,                  
				extendedchars=\true,
				keepspaces=true,
				literate={--}{{-{}-}}2,     
				literate={---}{{-{}-{}-}}3, 
				texcl=true, }
\end{lstlisting}

Останнє це оформлення секцій у вступі, зробимо так щоб відображалися крапки від секції до номера сторінки і щоб усе було нежирним:

\begin{lstlisting}
			\makeatletter
				%format sections in tableofcontents
				\renewcommand{\l@section}
					{\@dottedtocline{1}{0em}{1.25em}}
				\renewcommand{\l@subsection}
					{\@dottedtocline{2}{1.25em}{1.75em}}
				\renewcommand{\l@subsubsection}
					{\@dottedtocline{3}{2.75em}{2.6em}}
			\makeatother
\end{lstlisting}

Преамубла буде в окремому файлі \textit{preambule.tex}. Для вставлення інших файлів використовуєтсья команда \textit{input} або \textit{include}. Враховуючи титульну сторінку і конфіг класу, остаточний вигляд преамбули:

\begin{lstlisting}
			\documentclass[14pt,ukrainian,utf8, simple, 
			pointsection,floatsection ]{eskdtext} 
			
			\usepackage[figure,table]{totalcount} % counting figures, tables
\usepackage{eskdtotal} % total figures, tables...

\makeatletter
%format sections in tableofcontents
\renewcommand{\l@section}{\@dottedtocline{1}{0em}{1.25em}}
\renewcommand{\l@subsection}{\@dottedtocline{2}{1.25em}{1.75em}}
\renewcommand{\l@subsubsection}{\@dottedtocline{3}{2.75em}{2.6em}}
\makeatother

\usepackage{hyperref}
\hypersetup{
	colorlinks=false,
	hidelinks,
	bookmarks=true,
	,unicode=true
	,pdfcreator={XeLaTeX}
	,pdfa=true}

\renewcommand{\baselinestretch}{1.5} % Полуторный межстрочный интервал
\usepackage{graphicx}
\graphicspath{{images/}}

\usepackage[none]{hyphenat} % без переносів
\sloppy

% Формат подрисуночных надписей
\RequirePackage{caption}
\DeclareCaptionLabelSeparator{defffis}{ -- } % Разделитель
\captionsetup[figure]{justification=centering,
labelsep=defffis, format=plain} % Подпись рисунка по центру
\captionsetup[table]{justification=raggedleft, labelsep=defffis, 
format=plain, singlelinecheck=false} % Подпись таблицы справа
\addto\captionsukrainian{\renewcommand{\figurename}{Рисунок}}

\usepackage{float}
\usepackage{wrapfig}
\usepackage{subcaption}
\usepackage{array,tabularx,tabulary,booktabs}
\newcommand{\newSection}[3]{
	\newpage
	\section{\uppercase{#1}}
	\label{#2}
	\ESKDcolumnI{#3#1}
	\ESKDthisStyle{formII}
}
\newcommand{\anonsection}[2]{
	\newpage
	\phantomsection
	\addcontentsline{toc}{section}{\uppercase{#1}}
	\section*{\uppercase{#1}}
	\ESKDcolumnI{\uppercase{#1}}
	\label{#2}
	\ESKDthisStyle{formII}
}
% Списки
\usepackage{enumitem}
\setlist[enumerate,itemize]{leftmargin=1.5cm} % Отступы в списках
\setlist{nosep} % no separations

\usepackage{listings} % Оформление исходного кода
\lstset{
basicstyle=\small\ttfamily, % Размер и тип шрифта
breaklines=true,            % Перенос строк
tabsize=2,                  % Размер табуляции
extendedchars=\true,
keepspaces=true,
%frame=single,               % Рамка
literate={--}{{-{}-}}2,     % Корректно отображать двойной дефис
literate={---}{{-{}-{}-}}3,  % Корректно отображать тройной дефис
texcl=true,
}
\newcommand{\addimg}[4]{ 
	\begin{figure}
		\centering
		\includegraphics[width=#2\linewidth]{#1}
		\caption{#3} \label{#4}
	\end{figure}
}
\newcommand{\addimghere}[4]{ 
	\begin{figure}[H]
		\centering
		\includegraphics[width=#2\linewidth]{#1}
		\caption{#3} \label{#4}
	\end{figure}
}
\newcommand{\addtwoimghere}[4]{
	\begin{figure}[H]
		\centering
		\begin{subfigure}[t]{0.45\textwidth}
			\includegraphics[width=\textwidth]{#1}

			%\subcaption{}\label{sub:2a}
		\end{subfigure}
		\begin{subfigure}[t]{0.45\textwidth}
			\centering
			\includegraphics[width=\textwidth]{#2}
			%	\subcaption{}\label{sub:2b}
		\end{subfigure}
		\caption{#3}\label{#4}

	\end{figure}
}



			
			\usepackage{keyval}% http://ctan.org/pkg/keyval
\usepackage{lmodern} % font-size

\usepackage{multirow}
\newcolumntype{C}[1]{>{\centering\arraybackslash}m{#1}}
%\usepackage{enumerate}

\usepackage{fancyhdr} % Колонтитулы
\pagestyle{fancy}

\fancypagestyle{firststyle}{
	\renewcommand{\headrulewidth}{0pt}
	\fancyfoot{}
	\cfoot{Тернопіль 2020}
}
\renewcommand{\headrulewidth}{0pt}
\fancyfoot{}
\fancyhead{}

\makeatletter

\define@key{titlee}{university}{\def\tl@university{#1}}
\define@key{titlee}{katedra}{\def\tl@katedra{#1}}
\define@key{titlee}{type}{\def\tl@type{#1}}
\define@key{titlee}{discipline}{\def\tl@discipline{#1}}
\define@key{titlee}{thema}{\def\tl@thema{#1}}
\define@key{titlee}{kurs}{\def\tl@kurs{#1}}
\define@key{titlee}{group}{\def\tl@group{#1}}
\define@key{titlee}{specialty}{\def\tl@specialty{#1}}
\define@key{titlee}{author}{\def\tl@author{#1}}
\define@key{titlee}{posada}{\def\tl@posada{#1}}
\define@key{titlee}{kerivnyk}{\def\tl@kerivnyk{#1}}
\define@key{titlee}{pidpys}{\def\tl@pidpys{#1}}

% zavdannia
\define@key{titlee}{semestr}{\def\tl@semestr{#1}}
\define@key{titlee}{date}{\def\tl@date{#1}}
\define@key{titlee}{fulldate}{\def\tl@fulldate{#1}}
%kalendar
\define@key{titlee}{enddate}{\def\tl@enddate{#1}}
\define@key{titlee}{sources}{\def\tl@sources{#1}}
\define@key{titlee}{zapyska}{\def\tl@zapyska{#1}}
\define@key{titlee}{graphika}{\def\tl@graphika{#1}}

\setkeys{titlee}{university= University ,katedra= Kафедра,
type= Курсова робота ,thema= {Тема \\ \ } , discipline=Предмет, kurs=№, 
group=ke-4, specialty=122 - CS, author=\qquad \qquad \qquad
\qquad,posada=,kerivnyk=Teacher, semestr=4, date=,
sources=sources, graphika=,enddate=,fulldate=, zapyska={\ \quad \\ \ \\
\ \\ \ \\ \ \\ \ },pidpys=}%


\newcommand{\setzavdannia}[1][]{
	\begingroup%
	\setkeys{titlee}{#1}% Set new keys

	{\linespread{1}\selectfont
		\begin{center}
			Міністерство освіти і науки України\\
			\tl@university
		\end{center}
		\raggedright
		\setlength{\unitlength}{1cm}
		\begin{picture}(0,0)
			\put(2.45,-0.66){\line(1,0){15.05}}
			\put(2.7,-1.26){\line(1,0){14.8}}
			\put(0,-1.88){\line(1,0){17.5}}
			\put(3.15,-2.54){\line(1,0){14.35}}
		\end{picture}

		Кафедра \hspace{1cm} \tl@katedra

		Дисципліна \hspace{1cm} \tl@discipline

		Спеціальність \hspace{1cm} \tl@specialty

		\hspace{-0.35cm}
		\begin{tabular}{lp{0.05\linewidth}cC{0.1\linewidth}
		cC{0.05\textwidth}}

			Курс & \centering \tl@kurs & Група & \tl@group & Семестр 
			& \tl@semestr \\
			\cline{2-2} \cline{4-4} \cline{6-6}
		\end{tabular}
		\vspace{1.5cm}
		\begin{center}
			ЗАВДАННЯ\\
			на курсову роботу\\
			Студента\\
			\underline{\tl@author}\\
			{\small (прізвище, ім’я, по батькові)}
		\end{center}

		%\setlength{\unitlength}{1cm}
		\begin{picture}(0,0)
			\put(4.1,-0.66){\line(1,0){13.8}}
			\put(0.75,-1.26){\line(1,0){17.15}}

			\put(0.75,-1.88){\line(1,0){17.15}}
			\put(10,-3.1){\line(1,0){7.9}}
			\put(6.1,-3.7){\line(1,0){11.8}}
			\put(0.75,-4.36){\line(1,0){17.15}}
			\put(3.4,-5.55){\line(1,0){14.5}}
			\put(0.75,-6.2){\line(1,0){17.15}}
			\put(0.75,-6.84){\line(1,0){17.15}}
			\put(0.75,-7.46){\line(1,0){17.15}}
			\put(0.75,-8.01){\line(1,0){17.15}}
			\put(0.75,-8.68){\line(1,0){17.15}}
			\put(0.75,-9.3){\line(1,0){17.15}}

		\end{picture}
		\begin{enumerate}[label={\arabic*.},leftmargin=1cm]
			\item Тема роботи: \tl@thema

			      \vspace{0.66cm}

			\item Строк здачі студентом закінченої роботи \quad  \tl@enddate
			\item Вихідні дані до роботи: \tl@sources
			      \vspace{0.66cm}
			\item Зміст розрахунково - пояснювальної записки (перелік питань, 
			які підлягають розробці): \tl@zapyska

			      \begin{picture}(0,0)

				      \put(-0.2,-1.28){\line(1,0){17.15}}
				      \put(4.75,-1.88){\line(1,0){12.15}}

			      \end{picture}
			\item  Перелік графічного матеріалу (із точним зазначенням
			 обов’язкових креслень): \tl@graphika

			\item Дата видачі завдання: \quad \tl@date
		\end{enumerate}
	}
	\endgroup%
}
\newcommand{\settitle}[2][]{%

	\begingroup%
	\setkeys{titlee}{#1}% Set new keys
	{\linespread{1}\selectfont
		\centering
		Міністерство освіти і науки України\\
	
		\tl@university
		\hrule
	
		{\scriptsize (повне найменування вищого навчального закладу)}

		\hspace{0.2cm}
		Кафедра \tl@katedra
		\hrule
	
		{\scriptsize (повна назва кафедри)}

		\vspace{3cm}
		\begin{center}
			\textbf{ \large \tl@type}
		\end{center}

		з  \textbf{<<\tl@discipline>>}

		\hrule
		%	\vspace{0.2cm}
		{\scriptsize (назва дисципліни)}
		\hspace{0.2cm}


		\setlength{\unitlength}{1cm}
		\begin{picture}(0,0)
			\put(-9,-1.85){\line(1,0){18}}
			\put(-9,-0.75){\line(1,0){18}}
			\put(-9,-1.3){\line(1,0){18}}

		\end{picture}


		на тему : \textbf{<<\tl@thema>>}

	}
	{\linespread{1.2}\selectfont
		\vspace{5cm}
		\hfill
		\begin{minipage}{0.5\linewidth}


			\begin{tabular}{lp{0.1\linewidth}ccp{0.145\linewidth}}
				Студента & \centering \tl@kurs & курсу, & групи & \tl@group \\
				\cline{2-2} \cline{5-5}
			\end{tabular}

			\vspace{0.1cm}
			\begin{tabular}{lc}
				спеціальності & \tl@specialty \\
				\hline
				\tl@author    & \tl@pidpys    \\
				\hline
			\end{tabular}
			\vspace{-0.8cm}\hspace{0.3cm}	{ \centering\scriptsize (прізвище
			та ініціали)  \hspace{2cm}(підпис студента)} \\

			\begin{tabular}{p{0.3\textwidth}p{0.6\textwidth}}
				Керівник: & \tl@posada \\
				\cline{2-2}
				\multicolumn{2}{c}{\tl@kerivnyk}\\
				\hline
			\end{tabular}
			\centering
			{\scriptsize (посада, вчене звання, науковий ступінь, прізвище
			та ініціали) }

			\setlength{\unitlength}{1cm}
			\begin{picture}(0,0)
				\put(5.5, 0){\line(1,0){3.45}}
				\put(7.35, -0.7){\line(1,0){1.6}}
				\put(3, -0.7){\line(1,0){1.6}}
				\put(-2.3, -1.65){{\fontsize{10}{10} {\selectfont Члени комісії
				:}}}
			\end{picture}
			\raggedright
			{\fontsize
			{10}{10} \selectfont Оцінка за національною шкалою:}

			\begin{tabular}{lp{0.15\textwidth}lp{0.15\textwidth}}
				\fontsize
				{10}{10} \selectfont Кількість балів: &   & \fontsize{10}{10}
			 \selectfont Оцінка ECTS &
			\end{tabular}

			%\hspace{-2.2cm}\vspace{0.2cm}
			\hspace{0.2cm}
			\begin{tabular}{p{0.26\textwidth}p{0.001\textwidth}
			C{0.56\textwidth}}
				  &   & \tl@kerivnyk \\
				\cline{1-1} \cline{3-3}
			\end{tabular}
			\vspace{-0.89cm}	\centering

			{\scriptsize \hspace{-0.7cm}	(підпис)  \hspace{2.4cm}   (прізвище
			 та ініціали) }

			\hspace{0.2cm}
			\begin{tabular}{p{0.26\textwidth}p{0.001\textwidth}
			C{0.56\textwidth}}
				  &   &   \\
				\cline{1-1} \cline{3-3}
			\end{tabular}
			\vspace{-0.89cm}	\centering

			{\scriptsize \hspace{-0.7cm}	(підпис)  \hspace{2.4cm}   (прізвище
			 та ініціали) }

			\hspace{0.2cm}
			\begin{tabular}{p{0.26\textwidth}p{0.001\textwidth}C
			{0.56\textwidth}}
				  &   &   \\
				\cline{1-1} \cline{3-3}
			\end{tabular}
			\vspace{-0.89cm}\centering

			{\scriptsize \hspace{-0.7cm}	(підпис)  \hspace{2.4cm}   (прізвище
			та ініціали) }
		\end{minipage}

	}


	\thispagestyle{firststyle} % колонтитул

	\endgroup%

}

\newcommand{\setkalendar}[2][]{
	\newgeometry{right=3cm,left=1cm}

	\begingroup%
	\setkeys{titlee}{#1}% Set new keys
	{\linespread{1}\selectfont
		\begin{center}
			\textbf{Календарний план}
		\end{center}
		\centering
		\begin{tabulary}{1.0\textwidth}{|C{1cm}|L|C{3cm}|C{2.5cm}|}
			\hline
			\fontsize{10}{10} \selectfont № п/п &
			 \fontsize{10}{10} \selectfont Назва етапів курсового проекту
			( роботи )& \fontsize{10}{10} \selectfont Строк виконання етапів
			проекту (роботи) & \fontsize{10}{10} \selectfont Примітки \\
			\hline
			#2
			& & & \\ \hline
			& & & \\ \hline
			& & & \\ \hline
			& & & \\ \hline
			& & & \\ \hline
			& & & \\ \hline


		\end{tabulary}

		\vspace{1cm}\hspace{-0.5cm}
		\raggedright

		\begin{tabular}{p{2cm}p{5cm}p{2cm}C{5cm}}
			Студент &   &   & \tl@author \\
			\cline{2-2} \cline{4-4}
		\end{tabular}
		\centering

		{\scriptsize \hspace{3.5cm}	(підпис)  \hspace{5.4cm}   (прізвище, 
		ім’я, по батькові) }
		\begin{tabular}{p{2cm}p{5cm}p{2cm}C{5cm}}
			Керівник &   &   & \tl@kerivnyk \\
			\cline{2-2} \cline{4-4}
		\end{tabular}
		\centering

		{\scriptsize \hspace{3.5cm}	(підпис)  \hspace{5.4cm}   (прізвище,
		ім’я, по батькові) }

		\hspace{0.5cm}
		\raggedright
		\vspace{1cm}
		\begin{tabular}{c C{5cm} }
			Дата & \tl@fulldate \\
			\cline{2-2}
		\end{tabular}
	}
	\endgroup%
	\restoregeometry
}

\makeatother





			
			\usepackage{fontspec}
\usepackage{xecyr}
\usepackage{polyglossia}
\setmainlanguage{ukrainian}
\usepackage{xunicode, xltxtra}
\usepackage{cmap}	
\defaultfontfeatures{Ligatures=TeX}

\setmainfont{Times New Roman} 
\newfontfamily\cyrillicfont{Times New Roman}
\setotherlanguage{english}
\setmonofont{FreeMono}

%% Название документа
\ESKDtitle{\ESKDfontIII Створення шаблону для курсової в Latex}
\ESKDauthor{ ****** А.В. }
\ESKDchecker{****** О.Б. }

\ESKDsignature{ КПКН 20.055.014.000 ПЗ }
\ESKDcolumnIX{{\small ТНТУ, ФІС, КН ****}}

\renewcommand{\ESKDcolumnVIIname}{\ESKDfontII Аркуш}

\ESKDsectAlign{section}{Center}
\ESKDsectStyle{section}{\normalsize \bfseries \uppercase}
\ESKDsectStyle{subsection}{\normalsize \bfseries}
\ESKDsectSkip{section}{0pt}{0.8cm}
\ESKDsectSkip{subsection}{0.8cm}{0.5cm}
\ESKDsectSkip{subsubsection}{0.5cm}{0.1pt}

\end{lstlisting}

\subsection{Тіло документа}

В першу чергу необхідно застосувати створені команди для титульної сторінки і завдання з календарним планом, передаючи необхідні параметри:

\addimghere{6}{1}{Використання команд для перших 3 сторінок}{}

Команда \verb|\ESKDthisStyle{empty}| робить стиль сторінки без рамки. Для зручності ці команди буде винесено в окремий файл \textit{titlepage.tex}. Також розділи і все інше буде окремо:

\begin{lstlisting}
			\begin{document}
			
			\renewcommand{\ESKDcolumnVIIname}{\ESKDfontIII Аркуш}
			
\ESKDthisStyle{empty}
	
	\settitle[
	thema= Створення шаблону для курсової в Latex ,
	university=Тернопільський національний технічний університет імені Івана Пулюя,
	katedra=комп'ютерних наук,
	type=КУРСОВА РОБОТА,
	discipline={Комп’ютерні системи обробки текстової, графічної та мультимедійної
	інформації},
	kurs=3,
	group=****,
	specialty=122 Комп'ютерні науки,
	author=***********,
	kerivnyk=***********,
	]
	
	
	\newpage
	\ESKDthisStyle{empty}
	
	\setzavdannia[
	university=Тернопільський національний технічний університет імені Івана Пулюя,
	katedra= комп'ютерних наук,
	kurs= 3,
	discipline ={Комп’ютерні системи обробки текстової, графічної та мультимедійної інформації},
	specialty=  122 Комп'ютерні науки, 
	group= ****, 
	semestr= 6,
	author=******** Андрія Володимировича,
	thema= Створення шаблону для курсової в Latex \\ \ \\ \ \\ ,
	sources=,
	zapyska={Створення шаблону для курсової в Latex, Реферат, Зміст, Вступ, Розділ 1.Latex.Історія.Версії, Розділ 2. Інструментарій Latex. Розділ 3. Розробка шаблону для курсової, Розділ 4.
		Порівняння шаблону з іншими аналогами. Висновок, Список літературних
		джерел, Додатки \\ \ \\ \ \\ \  },
	graphika={Презентація – ХХ слайдів у форматі .PPTX, x додатки},
	date=01.05.2020,
	enddate=22.06.2020
	]

	\newpage
	\ESKDthisStyle{empty}
	\setkalendar[
		author=****** А.В.,
		kerivnyk=******* О.Б.,
		fulldate=1 травня 2020р.
	]{
	1 & Дата видачі індивідуального завдання & 01.05.2020 & Виконано \\ \hline
	2 & Підготовка до виконання курсової & 04.05.2020 & Виконано \\ \hline
	3&Формування структури курсової&08.04.2020&Виконано \\ \hline
	4&Збір інформації для про Latex&11.05.2020&Виконано \\ \hline
	5&Написання 1 розділу&13.05.2020&Виконано \\ \hline
	6&Написання 2 розділу& 16.05.2020&Виконано \\ \hline
	7&Створоення шаблону & 20.06.2020 & Виконано \\ \hline
	8&Написання 3 розділу&25.05.2020&Виконано \\ \hline
	9&Пошук інших шаблонів і порівняння&28.05.2020&Виконано \\ \hline
	10& Написання 4 розділу & 01.06.2020& Виконано \\ \hline
	11&Висновок курсової&05.06.2020&Виконано \\ \hline
	12&Захист курсової роботи& 22.06.2020 & \\ \hline
	  & & & \\ \hline
	    & & & \\ \hline
	      & & & \\ \hline
	        & & & \\ \hline
	               
	}

			\ESKDthisStyle{empty}

\begin{center}
	\uppercase{\textbf{Реферат}}
\end{center}
	
	Курсова робота // Створення шаблону для курсової в Latex// ******* Андрій Володимирович // Тернопільський національний
	технічний
	університет
	імені
	Івана
	Пулюя,
	факультет
	комп’ютерно-інформаційних систем та програмної інженерії, кафедра комп’ютерних наук,
	група **** // Тернопіль, 2020, сторінок \pageref{LastPage}, рисунків \totalfigures{} , джерел \ESKDtotal{bibitem} , таблиць \totaltables{}
	креслень 0, додатків \ESKDtotal{appendix}.
	
	Ключові слова: Latex, шаблон, шаблон для Latex, мова розмітки даних, Tex, типографія.
	
	В даній курсовій роботі було проведено опис систему для видавництва документів  Latex, та на його основі було створено шаблон для курсових робіт. Після створення здійснено порівняння з іншими шаблонами.
\newpage
			\ESKDthisStyle{formII}
			
			\tableofcontents
			\anonsection{Вступ}{sec:intro}
\ESKDthisStyle{formII}

Написання та оформлення текстів у високій якості потребує спеціального програмного забезпечення(ПЗ). Особливо важливо мати можливість використовувати системи набору тексту для написання документів, книг, статтей та інших форм видавничої справи в науковій галузі. Через необхідність написання складних текстів з красивим оформленням для наукових праць -- ця тема є актуальною для мене, оскільки, навчаючись в університеті потрібно вести звітність про виконання завдань, і найкращий спосіб вирішення цієї проблеми - це розробити шаблон для робіт, і сконцентруватися на самому завданні.

Багато ПЗ можна привести для прикладу, но ми зосередимося на наборі макророзширень системи комп'ютерної верстки \textbf{Latex}, яка розглядається в цій курсовій.

В ході наступних розділів буде розглянуто історію створення Latex, його основні можливості. Далі буде проводитись розробка шаблону для курсових робіт. Маючи такий шаблон, його можна буде легко модифікувати для використання в лабораторних роботах. В останньому розділі буде порівняння готового шаблону з іншими шаблонами, знайденими в мережі інтернет.

			\newSection{Історія Latex}{sec:seccc}{\ESKDfontV}


Говорячи про Latex потрібно вказати що він є, надбудовою над TEX - оригінальною системою текстового препроцесора. Все що можна зробити в Latex можна і в оригінальні системі TEX. Програмне забезпечення для набору тексту TEX було розроблено Дональдом Е. Кнутом в кінці 1970-х. Він був випущений з ліцензією з відкритим кодом і став стандартом наукового видавництва. Тепер TEX використовується для набору та публікації більшої частини світової інформації наукової літератури з фізики та математики.

Однією з найважливіших причин, по якій люди використовують LATEX, є те, що він відокремлює зміст документа від стилю. Це означає, що після написання вмісту вашого документа ми можемо легко змінити його вигляд. Так само ви можете створити один стиль документа, який можна використовувати для стандартизації зовнішнього вигляду безлічі різних документів. Це дозволяє науковим журналам створювати шаблони для публікацій. Ці шаблони мають заздалегідь зроблений макет, що означає, що потрібно додавати лише вміст.

Основний вплив для широкого розповсюдження здійснив
Леслі Лампорт у союзі з Пітером Гордоном в Аддісоні-Веслі,
версія 2.09 від приблизно середини 80-х років, яка походить від  системи TEX Дональда Кнута, яка досить швидко поширилася поза спільнотою північноамериканських математиків, які підтримували  розвиток TEX від його створення як один із "особистих інструментів продуктивності" Дона, створеного просто щоб забезпечити швидке завершення та типографічну якість його книги <<Мистецтво комп'ютерного програмування>> \cite{Knuth} Менш прямий, але, ймовірно, ширший вплив випливає з того, що вона є першою широко використовуваною мовою для опису логічної структури широкого кола документів таким \ чином \ її впровадження філософії логічного \ проектування, \ яке \newpage \noindent використовується Brain Reid in Scribe \cite{Reid}: "під час написання документа ви повинні
переймайтеся його логічним змістом, а не його візуальним оформленням."

Тоді Latex по-різному описувався як "TEX для мас" та "Написання, звільнене від негнучкого керування форматом ". Не зовсім
зрозуміло, чи все це було зроблено навмисне Леслі як особливість дизайну, але, безумовно, він не очікував що згодом, здійснить такий широкий вплив. Поширеність Latex була, навіть у кінці 1980-х років, більшою порівняно з більшістю некомерційного програмного забезпечення того часу. Хороші новини швидко поширювалися і 1994 року Леслі міг написати «Latex зараз надзвичайно популярний у науковій та академічній спільнотах, і він широко використовується у індустрії." Але цей рівень повсюдності все-таки був мізерним порівняно з сьогоднішнім днем, коли він став для багатьох професіоналів, невід'ємним інструментом, присутність якого є дуже важливою.

\subsection{Складові Latex(TeX)}

Робота з Latex пов'язана з так званими <<рівнями>> \cite{levels}:
\begin{enumerate}[label={\arabic*.},leftmargin=1.45cm]
	\item \textbf{Дистрибутиви}: MiKTeX, TeX Live,… Це великі колекції програмного забезпечення, пов'язаного з TeX, для завантаження та встановлення. Коли хтось каже: «Мені потрібно встановити TeX на свою машину», він зазвичай шукає дистрибутив.
	\item \textbf{Редактори}: Emacs, vim, TeXworks, TeXShop, TeXnicCenter, WinEdt,… ці редактори - це те, що ви використовуєте для створення файлу документа. Деякі (наприклад, TeXShop) присвячені спеціально TeX, інші (наприклад, Emacs) можуть використовуватися для редагування будь-яких файлів. Документи TeX не залежать від будь-якого конкретного редактора; сама програма набору TeX не включає редактор.
	\item \textbf{Компілятори}: TeX, pdfTeX, XeTeX, LuaTeX,… це виконувані бінарні файли, які реалізують різні варіанти TeX. Наприклад, pdfTeX реалізує прямий вихід у форматі PDF (якого немає в оригінальному TeX Кнута), LuaTeX забезпечує доступ до багатьох внутрішніх систем через вбудовану мову Lua і т.д. Коли хтось каже: "TeX не може знайти мої шрифти", вони зазвичай мають на увазі компілятор.
	\item \textbf{Формати}: LaTeX, звичайний TeX,… Це мови на основі TeX, якими фактично пишуть документи. Коли хтось каже, що "TeX дає мені невідому помилку", вони зазвичай мають на увазі формат. (До речі, "LaTeX" вже багато років означає "LaTeX2e".)
	\item \textbf{Пакети}: geometry, lm, ... Це доповнення до основної системи TeX, розроблені незалежно, надаючи додаткові функції набору тексту, шрифти, документацію тощо. Пакет може або не може працювати з будь-яким заданим форматом та/або компілятором; наприклад, багато є розроблено спеціально для LaTeX, але є і багато для інших. Сайти CTAN надають доступ до переважної більшості пакетів у світі TeX; CTAN, як правило, є джерелом, яке використовується дистрибутивами.
\end{enumerate}	

	Особливу увагу треба звернути на компілятори, в цій курсовій я використовую \textit{Xelatex}. Його особливістю є використання системний шрифтів, на відміну від Latex, який має свої вбудовані. Для оформлення документів в Україні згідно з ДСТУ 3008:2015 потрібно використовувати шрифт \textit{Times new Roman}, що і дозволяє зробити Xelatex. 
	
	\subsection{TeX і Latex}
	
	З моменту заснування розробка Latex та відповідного програмного забезпечення була повністю переплетена з розвитком самого TEX. 
	Хоча є багато сумнівів щодо корисністі деяких аспектів фундаментальних моделей та дизайну
	TEX як двигуна форматування тексту, він був тоді і залишається зараз (the
	початок тисячоліття) єдиним зрілим, широко доступним,
	програмованим і дуже гнучким компілятором для тексту. Таким чином для
	 Леслі в той час, як і для нас зараз, це єдиний вибір фундаменту
	для практичної автоматизації високоякісного форматування.
	
	У дизайні Latex Леслі свідомо дозволив основному компілятору TEX безпосередньо впливати на більшість текстових питань.
	У типових системах обробки тексту тієї епохи, включаючи TEX,
	основні методи обробки тексту документа такі: кожен вхідний маркер, що надсилається до системи, обробляється
	як складна імперативна команда. У таких системах "символ в
	тексті документа", як правило, подія на клавіатурі або маркер на
	вхідний буфер, не просто призначений викликати створення
	'елемента у рядку' в 'об'єкті текстового класу', такий 'рядок'
	врешті обробляється деяким іншим модулем системи, або
	навіть зовнішніми програмами.
	
	
	TEX був розроблений у цій імперативній парадигмі, оскільки це призводить до високоефективності(і в часі, і в просторі) машини, незважаючи на те, що "набір тексту" є для TEX відносно складним обчислювальним процесом, що включає, в першу чергу, оптимізацію	вибіру гліфа та позиціонування над цілими абзацами контрольоваий	за допомогою високо настроюваного алгоритму динамічного програмування. Однак, оскільки цей процес набору даних був оптимізований для швидкості, то робити що-небудь, що недоступно в рамках цього монолітного процесу (як визначено дизайном TEX), є важким у здійсненні та помітно неефективним у використанні. Такі процеси мають центральне значення для	якості набору і особливо важливі в наборі	інших мов, крім американської англійської. Вони включають модифікацію	важливих підпроцесів, таких як вибір гліфу (як для лігатур)	та їх розміри і розміщення; переноси та вирівнювання.
	
	
	\subsection{Версії}
	
	LaTeX2e - це поточна версія LaTeX, відколи вона замінила LaTeX 2.09 у 1994 році. Станом на 2019 рік LaTeX3, який розпочався створюватися на початку 1990-х років, досі розвивається. Планові функції включають покращений синтаксис, підтримку гіперпосилання, новий інтерфейс користувача, доступ до довільних шрифтів та нову документацію.\cite{latex3}
	
	Існують численні комерційні реалізації всієї системи TeX. Постачальники систем можуть додавати додаткові функції, такі як додаткові шрифти та підтримку по телефону. LyX - це безкоштовний, WYSIWYM-процесор візуального документа, який використовує LaTeX для бек-енду. TeXmacs - безкоштовний редактор WYSIWYG з аналогічними функціями, як LaTeX, але з іншим механізмом набору тексту. Інші редактори WYSIWYG, які використовують LaTeX, включають Scientific Word у MS Windows. Та BaKoMa TeX для Windows, Mac та Linux.
	
	Доступно декілька дистрибутивів TeX, що підтримуються спільнотою, зокрема TeX Live (багатоплатформна), teTeX (застаріла на користь TeX Live, UNIX), fpTeX (застаріла), MiKTeX (Windows), proTeXt (Windows), MacTeX (TeX Live з додаванням специфічних програм для Mac), gwTeX (Mac OS X) (застарілий), OzTeX (Mac OS Classic), AmigaTeX (більше не доступний), PasTeX (AmigaOS, доступний у сховищі Aminet) та Auto-Latex Equations (Додаток Google Docs, який підтримує команди MathJax LaTeX).
	
	\subsection{Сумісність та конвертація}
	
	Документи LaTeX (* .tex) можна відкрити будь-яким текстовим редактором. Вони складаються з простого тексту та не містять прихованих кодів форматування чи двійкових інструкцій. Крім того, документи TeX можна поширити у формат Rich Text (* .rtf) або XML. Це можна зробити за допомогою безкоштовних програм LaTeX2RTF або TeX4ht. LaTeX також може бути конвертовано у PDF-файли за допомогою розширення LaTeX pdfLaTeX. Файли LaTeX, що містять текст Unicode, можуть бути оброблені в PDF-файли за допомогою пакету \textit{inputenc} або розширеннь Tee XeLaTeX і LuaLaTeX.
	
	\begin{itemize}
		\item HeVeA - це перетворювач, написаний на Ocaml, який перетворює документи LaTeX у HTML5. Він ліцензований відповідно до Q Public License.
		
		\item LaTeX2HTML - це перетворювач, написаний на Perl, який перетворює документи LaTeX в HTML. Таким чином, наприклад, наукові праці, головним чином набрані для друку, можна розмістити для перегляду в Інтернеті. Він ліцензований відповідно до GNU GPL v2.
		
		\item LaTeXML - це безкоштовне програмне забезпечення публічного домену, написане на Perl, яке перетворює документи LaTeX у різноманітні структуровані формати, включаючи HTML5, epub, jats, tei.
		\item Pandoc - це "універсальний конвертер документів", здатний трансформувати LaTeX у безліч різних форматів файлів, включаючи HTML5, epub, rtf та docx. Він ліцензований відповідно до GNU GPL v2.
	
	\end{itemize}

	LaTeX став стандартом для набору математичних виразів в наукових документах. Таким чином, існує кілька інструментів перетворення, орієнтованих на математичні вирази LaTeX, такі як перетворювачі в MathML  або Computer Algebra System.
	
	\begin{itemize}
		\item  Mathoid - це веб-конвертер який використовує Node.js, він перетворює математичні входи, такі як LaTeX, у формати MathML та зображення, включаючи SVG та PNG. Він використовується у Вікіпедії для відображення математики.
		\item TeXZillais перетворювач на JavaScript з LaTeX в MathML. Це один з найшвидших перетворювачів LaTeX в MathML.
		\item  LaCASt - це перетворювач, написаний на Java, який перетворює семантичний діалект LaTeX в Maple та Mathematica.
	\end{itemize}


	\subsection{Особливості Latex}
	
	Створення LaTeX документа це програмування: Ви створюєте текстовий файл в LaTeX-розмітці, макроси LaTeX обробляють його і видають конкретний документ.
	
	Такий підхід відрізняється від використовуваного в WYSIWYG (What You See Is What You Get - те, що ви бачите, то і отримуєте) програмах, таких, як Openoffice Writer або Microsoft Word.
	
	В LaTeХ: 
	
	\begin{itemize}
		\item Під час редагування документа Ви не можете (зазвичай) побачити його остаточний варіант.
		
		\item 	Вам, як правило, потрібно знати необхідні команди розмітки LaTeX.
		
		\item Інколи складно отримати необхідний вигляд документа.
		
	\end{itemize}


		З іншого боку, у LaTeX є і переваги:
		
	\begin{itemize}
		\item Оформлення тексту відокремлено від вмісту. Ви повною мірою зосереджуєтеся на структурі та вмісті документу і забуваєте про те, як буде виглядати друкований варіант.
		\item Стиль, шрифти, оформлення таблиць і малюнків т. д. узгоджено у всьому документі.
		\item Одне і те саме оформлення можна використовувати для будь-якого числа документів.
		\item Легко набирає математичні формули.
		\item Легко створюються алфавітні вказівники, посилання та бібліографічні списки.
		\item Більші документи можуть бути розподілені на декілька файлів і працювати з ними окремо, в тому числі з використанням системи управління версіями.
		\item Вам не потрібно вручну налаштовувати шрифти, розмір тексту, високий шрифт - за це відповідає  LaTeX.
		\item  Вам доведеться правильно структурувати ваш документ.
		\item Файли з вихідними текстами можна переглянути і змінити в любому текстовому редакторі.
	
	\end{itemize}

	Підхід LaTeX до створення документа можна назвати WYSIWYM (What You See Is What You Mean - що бачиш, то і думаєш): під час набору тексту Ви не бачите остаточний варіант документа, тільки логічну структуру цього документа. Про оформлення замість Вас подбає LaTeX.

	
		
			\newSection{Інструментарій Latex}{sec:sec2}{\ESKDfontV}

Почати роботу з Latex необхідно зі створення нового документа з розширенням \textit{.tex}, при умові що у Вас встановлені всі необхідні пакети Latex. Як вже було сказано раніше треба мати встановлений дистрибутив та редактор. В дистрибутиві будуть знаходитися пакети та компілятори для створення документів. В якості дистрибутива встановлено \textit{texlive}, а редактор вихідного документа \textit{TexStudio}. TexStudio надає зручні можливості для роботи з  командами,наявні автодоповнення, сполучення клавіш - це все робить його  середовищем  для розробки Latex.

\subsection{Перший документ}

Створюємо файл з розширенням .tex та записуємо в нього наступні рядки:

\begin{lstlisting}
	\documentclass{article}
	
	\begin{document}
	
	\end{document}
\end{lstlisting}

На виході буде пустий документ. Перший рядок вказує на клас документу.Форматування за замовчуванням у документах LATEX визначається класом, який використовується цим документом. Стандартний вигляд можна змінити, а додаткові функції можна додати за допомогою пакета. Імена файлів класу мають розширення .cls, імена файлів пакунків мають розширення .sty.

Для різних типів документів потрібні різні класи, тобто для резюме буде потрібен інший клас, ніж для  наукового документа. У цьому випадку клас - це \textit{article}, найпростіший і найпоширеніший клас LATEX. \ Інші \ типи \ документів, \ над \newpage \noindent якими ви можете працювати, можуть вимагати різних класів, таких як \textbf{book} або \textit{report}.

Далі з команди \textit{begin} починається так зване тіло документа. В ньому буде знаходитися вміст документа, аж до \textit{end}

Щоб переглянути документ треба провести компіляцію. До прикладу такою командою:

\begin{lstlisting}
	pdflatex <your document> 
\end{lstlisting}

\subsection{Преамбула документа}

У попередньому прикладі текст був введений після команди  \verb|\begin {document} |. Все у .tex-файлі до цього моменту називається преамбулою. У преамбулі ви визначаєте тип документа, який ви пишете, мову, якою ви пишете, пакети, які ви хочете використовувати та декілька інших елементів. Наприклад, звичайна преамбула документа виглядатиме так:

\begin{lstlisting}
	\documentclass[12pt, letterpaper]{article}
	\usepackage[utf8]{inputenc}
\end{lstlisting}

\verb|\documentclass[12pt, letterpaper]{article}|. Як було сказано раніше, це визначає тип документа. Деякі додаткові параметри, що входять до квадратних дужок, можуть передаватися команді. Ці параметри повинні бути розділені комами. У прикладі додаткові параметри встановлюють розмір шрифту \textit{(12pt)} та розмір паперу (letterpaper). Звичайно, можна використовувати інші розміри шрифту \textit{(9pt, 11pt, 12pt)}, але якщо не вказано жодного, розмір за замовчуванням - 10pt. Що стосується розміру паперу, інші можливі значення - це папір формату А4 та legalpaper; 

\verb|\usepackage[utf8]{inputenc}|. Це кодування документа. Його можна опустити або змінити на інше кодування, але рекомендується utf-8. Якщо вам конкретно не потрібно інше кодування, або якщо ви не впевнені в цьому, додайте цей рядок до преамбули.


Для форматування тексту, задання формату сторінки, додавання графічних елементів та задання всіх можливих параметрів і налаштувань використовуються різні команди, які можна задати в класі документу або використовуючи додаткові пакети. 

\subsection{Макет сторінки}

За замовчуванням всі параметри документа встановлюють класи документів. Однак,
якщо ви хочете змінити ці параметри , є команди, які дозволяють вам це зробити. Команди, що контролюють функції, що стосуються всього документа, повинні бути розміщені в преамбулі.

\subsubsection{Параграфи}\label{par}

Щоб почати новий абзац, залиште порожній рядок або використовуйте команду \verb|\par|. Команди \verb|\parindent| і \verb|\parskip| задяють відступ абзацу та розділення абзацу. 

Для задання міжрядкового інтервалу можна використати команду \verb|\renewcommand{\baselinestretch}{1.5}|, яка встановить його на 1.5.

Абзацний відступ задається командою \verb|\parindent 1.25cm|.\par Перейти на новий рядок \verb|\\ або \par|

\subsubsection{Вирівнювання тексту}

За замовчуванням LATEX вирівнює ваш текст горизонтально, так що лівий і правий відступи є гладкими. Якщо ви віддаєте перевагу "вирівнювання справа" , ви можете використовувати:
\verb|\raggedright|
Зауважте, що це має побічний ефект для відступів абзацу. Якщо ви хочете залишити відступ абзаців, потрібно спеціально задати його (тобто \verb|\parindent = 1.5em|) після \verb|\raggedright| команди.

\subsubsection{Коментарі}

Як і будь-який код, який ви пишете, часто корисно включати коментарі. Коментарі - це фрагменти тексту, які ви можете включити в документ, які не будуть надруковані, і жодним чином не вплинуть на документ. Вони корисні для організації вашої роботи, ведення приміток або коментування рядків / розділів під час налагодження. Щоб зробити коментар у LATEX, просто напишіть символ \verb|%| на початку рядка.

\subsubsection{Вигляд тексту}

Зараз ми розглянемо кілька простих команд форматування тексту.

\begin{itemize}
	\item Жирний: Жирний текст у LaTeX пишеться командою \verb|\textbf {...}|
	\item Курсив: Курсивний текст у LaTeX пишеться командою \verb|\textit {...}|
	\item Підкреслення: Підкреслений текст у LaTeX пишеться командою \verb|\underline {...}|
\end{itemize}

LaTeX має кілька команд-модифікаторів розміру шрифту (від найбільших до найменших):

\begin{lstlisting}
		\Huge
		\huge
		\LARGE
		\Large
		\large
		\normalsize (default)
		\small
		\footnotesize
		\scriptsize
		\tiny
\end{lstlisting}

\subsubsection{Додавання зображень, таблиць}

Стандартний комплект графіки LaTeX включає два пакети для імпорту графіки:
\textit{graphics} та \textit{graphicx}. Розширений пакет, \textit{graphicsx}, забезпечує більш зручний спосіб подачі параметрів і рекомендується. Тому перший крок - це введіть у свою преамбулу команду:
\verb|\usepackage {graphicsx}|. Цей пакет визначає нову команду під назвою \verb|\includegraphics|, яка дозволяє вам вказувати назву графічного файлу, а також надавати необов'язкові аргументи для масштабування чи обертання. Отже, для прикладу потрібно вставити графіку (названу, наприклад, myfigure.eps
або myfigure.pdf) використовуйте команду \verb|\includegraphics|, наприклад:

\begin{lstlisting}
	\includegraphics[width=4in]{myfigure}
\end{lstlisting}


Tabular середовище є стандартним в LATEX для створення таблиць. Ви повинні вказати параметр для цього середовища, в цьому випадку {c c c}. Це говорить про те, що LATEX буде три стовпці і текст у кожному з них повинен бути в центрі. Ви також можете використовувати r, щоб вирівняти текст праворуч, а l - для вирівнювання ліворуч. Символ \& використовується для визначення розділення у записах таблиці. У кожному рядку завжди повинно бути на один менше символів розділення, ніж кількість стовпців. Для переходу до наступного рядка таблиці використовуємо команду нового рядка \verb|\\|.

\begin{lstlisting}
	\begin{tabular}{ c c }
		cell 1 & cell 2 \\
		cell 3 & cell 4 
	\end{tabular}
\end{lstlisting}

Ви можете додати межі за допомогою команди горизонтальної лінії \verb|hline| та параметра вертикальної лінії |.

\begin{lstlisting}
	\begin{tabular}{ |c|c| }
		\hline 
		cell 1 & cell 2 \\ \hline
		cell 3 & cell 4 \\ \hline
	\end{tabular}
\end{lstlisting}


{| c | c | c | }: Тут вказується, що у таблиці будуть використані три стовпчики, розділені вертикальною лінією. | символ вказує, що ці стовпці повинні бути розділені вертикальною лінією. 

\verb|hline|: буде вставлена горизонтальна лінія. Тут ми ввели горизонтальні лінії вгорі та внизу таблиці. Немає обмежень у кількості разів, коли ви можете використовувати \verb|\hline|.

\subsubsection{Структурування}

Команди для організації документа відрізняються залежно від типу документа, найпростішою формою організації є секціонування, доступне у всіх форматах.

Команда \verb|section{}| позначає початок нового розділу, всередині дужок встановлюється заголовок. Нумерація розділів є автоматичною і її можна відключити, включивши * в команду розділу як \verb|\section*{}|. Ми також можемо мати підрозділи \verb|\subsection{}|, і пункти  \verb|sububsection{}|. 

\subsubsection{Підписи до зображень, таблиць}\label{capt}

В \textit{figure} або \textit{table}
середовищах, ви можете надати підпис із командою \verb|\caption {caption text}|. Зазвичай підписи для таблиці вводяться над таблицею, а підпис до зображення
нижче зображення.

Якщо потрібно якось модифікувати вигляд підпису то можна скористатися пакетом \textit{caption}. Наприклад відцентрувати підпис зображення, та змінити слово Рис. на Рисунок:

\begin{lstlisting}
		\captionsetup[figure]{justification=centering, labelsep=defffis,
		format=plain}		
		\addto\captionsukrainian{\renewcommand{\figurename}{ Рисунок }}
\end{lstlisting}


\subsection{Створення та перевизначення команд}

Щоб додати свої власні команди, використовуйте команду:

\begin{lstlisting}
		\newcommand{\name}[num]{definition}
\end{lstlisting}


В основному, команда вимагає двох аргументів: ім'я команди, яку ви хочете створити, і визначення команди. Зауважте, що ім'я команди можна, але не потрібно вкладати в дужки, як вам подобається. Аргумент num у квадратних дужках є необов’язковим і визначає кількість аргументів, які приймає нова команда (можливо до 9). Якщо він відсутній, він за замовчуванням дорівнює 0, тобто аргумент не дозволений.

Для прикладу створимо команду додавання зображення:

\begin{lstlisting}
		\newcommand{\addimg}[4]{
			\begin{figure}
				\centering
				\includegraphics[width=#2\linewidth]{#1}
				\caption{#3} \label{#4}
			\end{figure}
		}
\end{lstlisting}

Команда отримує 4 аргументи, для використання аргументу вводять \# та номер параметра.

Щоб перевизначити команду треба ввести \verb|\renewcommand{cmd}{def}|. Ми вже використовували її для визначення підпису до малюнків на ст. \pageref{capt} та для міжрядкового інтервалу ст. \pageref{par}

Ще багато команд можна було б описати, проте цього достатньо для початку. В наступному розділі буде відбуватись створення шаблону, в процесі використань нових команд буде надано  їхній короткий опис.


 
			\newSection{Розробка шаблону}{sec:sec3}{}

Почати роботу з виготовлення шаблону потрібно з вимог до завдання, а саме: потрібно розробити шаблон для курсових робіт з дотриманням ДСТУ 3008:2015. Ось деякі плавила оформлення:

\begin{itemize}
	\item Аркуші формату А4 (210х297 мм).
	\item Шрифт 14 розміру з 1,5 інтервалом.
	\item  віддаль між рядками повинна бути однакова і рівна 8-10 мм.
	\item  відстань між заголовками підрозділів або пунктів і подальшим або попереднім текстом 15-20 мм;
	\item  відстань між назвою розділу і назвою підрозділу або пункту 18-22 мм;
	\item  абзацний відступ повинен бути однаковим впродовж усього тексту записки і дорівнювати 10-15 мм.
\end{itemize}

Ще одною умовою є додавання рамок до роботи. І останній пункт, додатковий - зверстати титульну сторінку.

\subsection{Написання преамбули}

\subsubsection{Клас документа та кодування}

Для звичайної курсової роботи, без рамок, можна було б скористатися класом документа \textit{article} або \textit{extarticle}, проте довелося б реалізовувати рамки вручну. На просторах інтернету можна знайти колекцію пакетів \textit{eskdx}, який надає нам такий функціонал. Колекція пакетів надає 3 класи: \textit{eskdtext}(для текстової документації), \textit{eskdgraph}(для креслення схем) і \textit{eskdtab}(для документів, разбитих на графи). Для нашого випадку підходить eskdtext, його і використаємо.

\newpage

Ось як буде виглядати підключення класу:

\begin{lstlisting}
		\documentclass[14pt,ukrainian,utf8, simple, pointsection,
		floatsection ]{eskdtext} 
\end{lstlisting}

В додаткових параметрах вказано 14 розмір шрифту, українську мову(вибір тільки з 2, інша - російська), кодування, simple - відображати тільки основні графи, останні два параметри вказують на нумерування пунктів та фігур в межах секцій.

Для роботи з кирилицею потрібно підключити мовний пакет. На вибір є 2 основних: \textit{babel} та \textit{poliglossia}, скористаємося другим. Перевага другого пакету в тому що кириличні символи кодуються правильно, навідміну від babel, який робить заміну \textit{і} на латиницю. Це допоможе при проходженні роботи на антиплагіат.

Підключаємо його та ще пару пакетів для кодування кирилиці:

\begin{lstlisting}
		\usepackage{fontspec}
		\usepackage{xecyr}
		\usepackage{polyglossia}
		\setmainlanguage{ukrainian}
		\usepackage{xunicode, xltxtra}
\end{lstlisting}


\subsubsection{Специфічні налаштування класу}


Пакети ESKDX надають багато налаштувань, їх ми будемо зберігати в окремому файлі з назвою \textit{ESKDXconfig.tex}. Він є частиною преамбули тож і підключатиметься там. В цьому файлі зберігатимуться попередні команди і також для налаштування шрифтів.

\begin{lstlisting}
		\defaultfontfeatures{Ligatures=TeX}
		\setmainfont{Times New Roman} 
		\newfontfamily\cyrillicfont{Times New Roman}
		\setotherlanguage{english}
		\setmonofont{FreeMono}
\end{lstlisting}

Вказавши деякі команди можна створити титульну сторінку, проте її формати не підходить, тому ми будемо писати свою титулку в наступному підрозділі. Зараз тільки вкажемо деякі необхідні команди для задання інформації на рамках.
\begin{lstlisting}
			\ESKDsignature{ КПКН 20.055.014.000 ПЗ
			\ESKDcolumnIX{{\small ТНТУ, ФІС, КН СН-31}}
			\ESKDtitle{\ESKDfontIII Створення шаблону для курсової в Latex}
			\ESKDauthor{ **** А. В. }
			\ESKDchecker{ **** О. Б. }
			
\end{lstlisting}

Останнє в цьому файлі це налаштування стилів відображення секцій:

\begin{lstlisting}
			\ESKDsectAlign{section}{Center}
			\ESKDsectStyle{section}{\normalsize \bfseries \uppercase}
			\ESKDsectStyle{subsection}{\normalsize \bfseries}
			\ESKDsectSkip{section}{0pt}{0.8cm}
			\ESKDsectSkip{subsection}{0.8cm}{0.5cm}
			\ESKDsectSkip{subsubsection}{0.5cm}{0.1pt}
\end{lstlisting}

Секції центруємо, робимо 14 шрифтом, жирний та все у верхньому регістрі. Підсекції - те саме, тільки у звичайному регісті. Також налаштовано відступи між секціями.

Для цього файлу - це все, повний обсяг буде наведено в додатках.

\subsubsection{Створення титульної сторінки}

Налаштування титульної сторінки будуть знаходитися в окремому файлі \textit{title.tex}.

Спершу налаштуємо колонтитули, оскільки це найлегше що можна зробити зараз:
\begin{lstlisting}
		\usepackage{fancyhdr} % Колонтитули
		\pagestyle{fancy}
		
		\fancypagestyle{firststyle}{
		\renewcommand{\headrulewidth}{0pt}
		\fancyfoot{}
		\cfoot{Тернопіль 2020}
		}
		\renewcommand{\headrulewidth}{0pt}
		\fancyfoot{}
		\fancyhead{}
\end{lstlisting}

Використали пакет \textit{fancyhdr} і створили стиль колонтитула для сторінки в якому по центрі внизу записали необхідні дані. 

Створимо нову команду для титульної сторінки, яка буде приймати аргументи для їх встановлення на сторінку: найменування вищого навчального закладу, кафедра, назва роботи, тема, дисципліна, і т.д. Також буде створено ще 2 команди для завдання та календарного плану. Команди можуть приймати до 9 параметрів, в нашому випадку їх є більше, тому скористаємося пакетом \textit{keyval}, який дозволить використовувати опційні параметри, які не обмежуються кількістю.
Оголошення змінних та задання стандартних значень:

\begin{lstlisting}
			\define@key{titlee}{university}{\def\tl@university{#1}}
			\define@key{titlee}{katedra}{\def\tl@katedra{#1}}
			\define@key{titlee}{type}{\def\tl@type{#1}}
			\define@key{titlee}{discipline}{\def\tl@discipline{#1}}
			\define@key{titlee}{thema}{\def\tl@thema{#1}}
			\define@key{titlee}{kurs}{\def\tl@kurs{#1}}
			\define@key{titlee}{group}{\def\tl@group{#1}}
			\define@key{titlee}{specialty}{\def\tl@specialty{#1}}
			\define@key{titlee}{author}{\def\tl@author{#1}}
			\define@key{titlee}{posada}{\def\tl@posada{#1}}
			\define@key{titlee}{kerivnyk}{\def\tl@kerivnyk{#1}}
			\define@key{titlee}{pidpys}{\def\tl@pidpys{#1}}
			
			% zavdannia
			\define@key{titlee}{semestr}{\def\tl@semestr{#1}}
			\define@key{titlee}{date}{\def\tl@date{#1}}
			\define@key{titlee}{fulldate}{\def\tl@fulldate{#1}}
			%kalendar
			\define@key{titlee}{enddate}{\def\tl@enddate{#1}}
			\define@key{titlee}{sources}{\def\tl@sources{#1}}
			\define@key{titlee}{zapyska}{\def\tl@zapyska{#1}}
			\define@key{titlee}{graphika}{\def\tl@graphika{#1}}
			
			\setkeys{titlee}{university= University ,katedra = Kафедра,
			type= Курсова робота ,thema= {Тема \\ \ } , discipline = Предмет,
			kurs=№, group=ke-4, specialty=122 - CS, author=\qquad \qquad
			\qquad \qquad,posada=,kerivnyk=Teacher, semestr=4, date=,
			sources=sources, graphika=,enddate=,fulldate=, zapyska={\
			\quad \\ \ \\ \ \\ \ \\ \ \\ \ }, pidpys=}%			
\end{lstlisting}


Створюємо нові команди, всі будуть приймати один опційний аргумент, який потім в \verb|\setkeys{}|встановиться у відповідні змінні. Для використання змінних потрібно щоб вони перебували в групі, це оголошується \verb|\begingroup%|:

\begin{lstlisting}
			\newcommand{\setzavdannia}[1][]{
			\begingroup%
			\setkeys{titlee}{#1}% Set new keys
			...content
			\endgroup%
			}
\end{lstlisting}

Далі проведено опис основних моментів створення титульної сторінки, для детальнішого огляду див. додатки.

Починаємо зверху, де треба вказати навч. заклад та кафедру:

\begin{lstlisting}
		\centering
		Міністерство освіти і науки України\\
		\tl@university
		\hrule
		{\scriptsize (повне найменування вищого навчального закладу)}
		\hspace{0.2cm}
		Кафедра \tl@katedra
		\hrule
		{\scriptsize (повна назва кафедри)}
\end{lstlisting}

Команда \verb|\centering| відцентрує текст. У 2 рядку встановлюємо на параметр на його місце та малюємо горизонтальну лінію на всю ширину - \verb|\hrule|. \verb|\scriptsize| зменшує шрифт на дуже маленький.

Результат:

\addimghere{3}{1}{Верхня частина титулки}{}

Наступним створюємо тип та назву роботи:

\begin{lstlisting}
			\vspace{3cm}
			\begin{center}
			\textbf{ \large \tl@type}
			\end{center}
			
			з  \textbf{<<\tl@discipline>>}
			\hrule
			
			{\scriptsize (назва дисципліни)}
			\hspace{0.2cm}
			\setlength{\unitlength}{1cm}
			
			\begin{picture}(0,0)
				\put(-9,-1.85){\line(1,0){18}}
				\put(-9,-0.75){\line(1,0){18}}
				\put(-9,-1.3){\line(1,0){18}}
			\end{picture}
			
			на тему : \textbf{<<\tl@thema>>}
			}			
					
\end{lstlisting}

Команди \verb|\vspace{}, \hspace{}| роблять відступ вертикально та горизонтально на вказану відстань. Середовище \verb|\begin{picture}(0,0)| Робить область для створення простих графіків, фігур. В дужках вказано її розміри, в нас нема розміру для зручності, оскільки будуть малюватися лінії серед тексту. 

Команда \textit{put} ставить певну фігуру за вказаними координатами, її можна використовувати тільки в середовищі \textit{picture}. \textit{Line} - фігура лінії, в дужках осі, в параметрі довжина.

Ось результат:

\addimghere{4}{1}{Назва роботи, тема}{}

Перейдемо до останньої частини, де вказується інформація про автора та керівника. 

\begin{lstlisting}
			\vspace{5cm}
			\hfill
			\begin{minipage}{0.5\linewidth}
				\begin{tabular}{lp{0.1\linewidth}ccp{0.145\linewidth}}
					Студента & \centering \tl@kurs & курсу, & групи & \tl@group
					\\
					\cline{2-2} \cline{5-5}
				\end{tabular}
				\vspace{0.1cm}
				\begin{tabular}{lc}
					спеціальності & \tl@specialty \\
					\hline
					\tl@author    & \tl@pidpys    \\
					\hline
				\end{tabular}
				\vspace{-0.8cm}\hspace{0.3cm}	{ \centering\scriptsize (прізвище 
				та ініціали)  \hspace{2cm}(підпис студента)} \\
				
				\begin{tabular}{p{0.3\textwidth}p{0.6\textwidth}}
					Керівник: & \tl@posada \\
					\cline{2-2}
					\multicolumn{2}{c}{\tl@kerivnyk}\\
					\hline
				\end{tabular}
				\centering
				{\scriptsize (посада, вчене звання, науковий ступінь, прізвище
				та ініціали) }
			\end{minipage}
\end{lstlisting}

\addimghere{5}{1}{Остання частина титульної сторінки}{}


Поєднання \textit{begin\{minipage\}} та \textit{hfill} робить окрему область сторінки на половину її ширини та розташовує її справа. Для організації розмітки використовується таблиця, командами \verb|\cline{2-2} \cline{5-5}|, можна підкреслити необхідні стовпці в рядку.

Поєднуючи таблиці та малювання ліній, було зроблено необхідні лінії для 3 сторінок: титулка, завдання. календарний план.

\subsection{Завершення преамбули}

Для роботи із зображеннями підключаємо пакет \textit{graphicx}, вказуємо їхнє розташування:

\begin{lstlisting}
			\usepackage{graphicx} % Вставка картинок 
			\graphicspath{{images/}}
\end{lstlisting}

Напишемо пару нових команд для вставлення картинок. Перша команда вставляє одне зображення в зручному для Latex місці, приймає 4 команди: назва, ширина, підпис, позначка. Друга команда робить все те саме, проте вставляє картинку так як вона є в тексті. Остання команда вставляє 2 зображення поруч з одним підписом до них.

\begin{lstlisting}
			\newcommand{\addimg}[4]{ % add one img
				\begin{figure}
					\centering
					\includegraphics[width=#2\linewidth]{#1}
					\caption{#3} \label{#4}
				\end{figure}
			}
			\newcommand{\addimghere}[4]{ % add img here
				\begin{figure}[H]
					\centering
					\includegraphics[width=#2\linewidth]{#1}
					\caption{#3} \label{#4}
				\end{figure}
			}
			\newcommand{\addtwoimghere}[4]{ % two img side by side
				\begin{figure}[H]
					\centering
					\begin{subfigure}[t]{0.45\textwidth}
						\includegraphics[width=\textwidth]{#1}
					\end{subfigure}
					\begin{subfigure}[t]{0.45\textwidth}
						\centering
						\includegraphics[width=\textwidth]{#2}
						
					\end{subfigure}
					\caption{#3}\label{#4}
				
				\end{figure}
			}
\end{lstlisting}

Створимо ще дві функції для роботи із секціями. Команда \textit{newSection} додає нову секція з нової сторінки, також є позначення для посилання на неї, останнє - це застосування розширеної рамки і вказанням того самого розділу. Друга команда схожа, тільки секції будуть без нумерації.
\begin{lstlisting}
			\newcommand{\newSection}[3]{
				\newpage
				\section{\uppercase{#1}}
				\label{#2}
				\ESKDcolumnI{#3#1}
				\ESKDthisStyle{formII}
			}
			
			\newcommand{\anonsection}[2]{
				\newpage
				\phantomsection
				\addcontentsline{toc}{section}{\uppercase{#1}}
				\section*{\uppercase{#1}}
				\ESKDcolumnI{\uppercase{#1}}
				\label{#2}
				\ESKDthisStyle{formII}
			}
\end{lstlisting}

Один дуже важливий нюанс. Пакет ESKDX не надає можливості зміни даних в рамках в тілі документа, як це зробили ми, це призведе до помилки. Один з варіантів вирішення є написання власного стилю рамки, проте це дуже важко і потребує поглиблених знань пакету. Другий варіант - зміна пари рядків коду в початкових файлах пакету. Під час компіляції можна побачити що викликається команда збереження рамки в преамбулі та застосовується без змін далі, тому треба замість застосування - заново намалювати. Отже у пакетному файлі \textit{eskdstamp.sty} потрібно знайти 2 форму рамок(\verb|\ESKD@stamp@ii@box|) та скопіювати код створення рамки в місце де застосовується збережена рамка. 

Налаштуємо підписи до малюнків та таблиць:

\begin{lstlisting}
			\RequirePackage{caption}
			\DeclareCaptionLabelSeparator{defffis}{ -- } % Розділювач
			\captionsetup[figure]{justification=centering, labelsep=defffis,
			 format=plain} % Підпис малюнка по центру
			\captionsetup[table]{justification=raggedleft, labelsep=defffis, 
			format=plain, singlelinecheck=false} % Підпис таблиці справа
			\addto\captionsukrainian{\renewcommand{\figurename}{Рисунок}} 
			% Ім' я фігури
\end{lstlisting}

Оформимо відображення початкового коду, використовуючи пакет \textit{listings}:

\begin{lstlisting}
			\usepackage{listings}
			\lstset{
				basicstyle=\small\ttfamily,
				breaklines=true,
				tabsize=2,                  
				extendedchars=\true,
				keepspaces=true,
				literate={--}{{-{}-}}2,     
				literate={---}{{-{}-{}-}}3, 
				texcl=true, }
\end{lstlisting}

Останнє це оформлення секцій у вступі, зробимо так щоб відображалися крапки від секції до номера сторінки і щоб усе було нежирним:

\begin{lstlisting}
			\makeatletter
				%format sections in tableofcontents
				\renewcommand{\l@section}
					{\@dottedtocline{1}{0em}{1.25em}}
				\renewcommand{\l@subsection}
					{\@dottedtocline{2}{1.25em}{1.75em}}
				\renewcommand{\l@subsubsection}
					{\@dottedtocline{3}{2.75em}{2.6em}}
			\makeatother
\end{lstlisting}

Преамубла буде в окремому файлі \textit{preambule.tex}. Для вставлення інших файлів використовуєтсья команда \textit{input} або \textit{include}. Враховуючи титульну сторінку і конфіг класу, остаточний вигляд преамбули:

\begin{lstlisting}
			\documentclass[14pt,ukrainian,utf8, simple, 
			pointsection,floatsection ]{eskdtext} 
			
			\usepackage[figure,table]{totalcount} % counting figures, tables
\usepackage{eskdtotal} % total figures, tables...

\makeatletter
%format sections in tableofcontents
\renewcommand{\l@section}{\@dottedtocline{1}{0em}{1.25em}}
\renewcommand{\l@subsection}{\@dottedtocline{2}{1.25em}{1.75em}}
\renewcommand{\l@subsubsection}{\@dottedtocline{3}{2.75em}{2.6em}}
\makeatother

\usepackage{hyperref}
\hypersetup{
	colorlinks=false,
	hidelinks,
	bookmarks=true,
	,unicode=true
	,pdfcreator={XeLaTeX}
	,pdfa=true}

\renewcommand{\baselinestretch}{1.5} % Полуторный межстрочный интервал
\usepackage{graphicx}
\graphicspath{{images/}}

\usepackage[none]{hyphenat} % без переносів
\sloppy

% Формат подрисуночных надписей
\RequirePackage{caption}
\DeclareCaptionLabelSeparator{defffis}{ -- } % Разделитель
\captionsetup[figure]{justification=centering,
labelsep=defffis, format=plain} % Подпись рисунка по центру
\captionsetup[table]{justification=raggedleft, labelsep=defffis, 
format=plain, singlelinecheck=false} % Подпись таблицы справа
\addto\captionsukrainian{\renewcommand{\figurename}{Рисунок}}

\usepackage{float}
\usepackage{wrapfig}
\usepackage{subcaption}
\usepackage{array,tabularx,tabulary,booktabs}
\newcommand{\newSection}[3]{
	\newpage
	\section{\uppercase{#1}}
	\label{#2}
	\ESKDcolumnI{#3#1}
	\ESKDthisStyle{formII}
}
\newcommand{\anonsection}[2]{
	\newpage
	\phantomsection
	\addcontentsline{toc}{section}{\uppercase{#1}}
	\section*{\uppercase{#1}}
	\ESKDcolumnI{\uppercase{#1}}
	\label{#2}
	\ESKDthisStyle{formII}
}
% Списки
\usepackage{enumitem}
\setlist[enumerate,itemize]{leftmargin=1.5cm} % Отступы в списках
\setlist{nosep} % no separations

\usepackage{listings} % Оформление исходного кода
\lstset{
basicstyle=\small\ttfamily, % Размер и тип шрифта
breaklines=true,            % Перенос строк
tabsize=2,                  % Размер табуляции
extendedchars=\true,
keepspaces=true,
%frame=single,               % Рамка
literate={--}{{-{}-}}2,     % Корректно отображать двойной дефис
literate={---}{{-{}-{}-}}3,  % Корректно отображать тройной дефис
texcl=true,
}
\newcommand{\addimg}[4]{ 
	\begin{figure}
		\centering
		\includegraphics[width=#2\linewidth]{#1}
		\caption{#3} \label{#4}
	\end{figure}
}
\newcommand{\addimghere}[4]{ 
	\begin{figure}[H]
		\centering
		\includegraphics[width=#2\linewidth]{#1}
		\caption{#3} \label{#4}
	\end{figure}
}
\newcommand{\addtwoimghere}[4]{
	\begin{figure}[H]
		\centering
		\begin{subfigure}[t]{0.45\textwidth}
			\includegraphics[width=\textwidth]{#1}

			%\subcaption{}\label{sub:2a}
		\end{subfigure}
		\begin{subfigure}[t]{0.45\textwidth}
			\centering
			\includegraphics[width=\textwidth]{#2}
			%	\subcaption{}\label{sub:2b}
		\end{subfigure}
		\caption{#3}\label{#4}

	\end{figure}
}



			
			\usepackage{keyval}% http://ctan.org/pkg/keyval
\usepackage{lmodern} % font-size

\usepackage{multirow}
\newcolumntype{C}[1]{>{\centering\arraybackslash}m{#1}}
%\usepackage{enumerate}

\usepackage{fancyhdr} % Колонтитулы
\pagestyle{fancy}

\fancypagestyle{firststyle}{
	\renewcommand{\headrulewidth}{0pt}
	\fancyfoot{}
	\cfoot{Тернопіль 2020}
}
\renewcommand{\headrulewidth}{0pt}
\fancyfoot{}
\fancyhead{}

\makeatletter

\define@key{titlee}{university}{\def\tl@university{#1}}
\define@key{titlee}{katedra}{\def\tl@katedra{#1}}
\define@key{titlee}{type}{\def\tl@type{#1}}
\define@key{titlee}{discipline}{\def\tl@discipline{#1}}
\define@key{titlee}{thema}{\def\tl@thema{#1}}
\define@key{titlee}{kurs}{\def\tl@kurs{#1}}
\define@key{titlee}{group}{\def\tl@group{#1}}
\define@key{titlee}{specialty}{\def\tl@specialty{#1}}
\define@key{titlee}{author}{\def\tl@author{#1}}
\define@key{titlee}{posada}{\def\tl@posada{#1}}
\define@key{titlee}{kerivnyk}{\def\tl@kerivnyk{#1}}
\define@key{titlee}{pidpys}{\def\tl@pidpys{#1}}

% zavdannia
\define@key{titlee}{semestr}{\def\tl@semestr{#1}}
\define@key{titlee}{date}{\def\tl@date{#1}}
\define@key{titlee}{fulldate}{\def\tl@fulldate{#1}}
%kalendar
\define@key{titlee}{enddate}{\def\tl@enddate{#1}}
\define@key{titlee}{sources}{\def\tl@sources{#1}}
\define@key{titlee}{zapyska}{\def\tl@zapyska{#1}}
\define@key{titlee}{graphika}{\def\tl@graphika{#1}}

\setkeys{titlee}{university= University ,katedra= Kафедра,
type= Курсова робота ,thema= {Тема \\ \ } , discipline=Предмет, kurs=№, 
group=ke-4, specialty=122 - CS, author=\qquad \qquad \qquad
\qquad,posada=,kerivnyk=Teacher, semestr=4, date=,
sources=sources, graphika=,enddate=,fulldate=, zapyska={\ \quad \\ \ \\
\ \\ \ \\ \ \\ \ },pidpys=}%


\newcommand{\setzavdannia}[1][]{
	\begingroup%
	\setkeys{titlee}{#1}% Set new keys

	{\linespread{1}\selectfont
		\begin{center}
			Міністерство освіти і науки України\\
			\tl@university
		\end{center}
		\raggedright
		\setlength{\unitlength}{1cm}
		\begin{picture}(0,0)
			\put(2.45,-0.66){\line(1,0){15.05}}
			\put(2.7,-1.26){\line(1,0){14.8}}
			\put(0,-1.88){\line(1,0){17.5}}
			\put(3.15,-2.54){\line(1,0){14.35}}
		\end{picture}

		Кафедра \hspace{1cm} \tl@katedra

		Дисципліна \hspace{1cm} \tl@discipline

		Спеціальність \hspace{1cm} \tl@specialty

		\hspace{-0.35cm}
		\begin{tabular}{lp{0.05\linewidth}cC{0.1\linewidth}
		cC{0.05\textwidth}}

			Курс & \centering \tl@kurs & Група & \tl@group & Семестр 
			& \tl@semestr \\
			\cline{2-2} \cline{4-4} \cline{6-6}
		\end{tabular}
		\vspace{1.5cm}
		\begin{center}
			ЗАВДАННЯ\\
			на курсову роботу\\
			Студента\\
			\underline{\tl@author}\\
			{\small (прізвище, ім’я, по батькові)}
		\end{center}

		%\setlength{\unitlength}{1cm}
		\begin{picture}(0,0)
			\put(4.1,-0.66){\line(1,0){13.8}}
			\put(0.75,-1.26){\line(1,0){17.15}}

			\put(0.75,-1.88){\line(1,0){17.15}}
			\put(10,-3.1){\line(1,0){7.9}}
			\put(6.1,-3.7){\line(1,0){11.8}}
			\put(0.75,-4.36){\line(1,0){17.15}}
			\put(3.4,-5.55){\line(1,0){14.5}}
			\put(0.75,-6.2){\line(1,0){17.15}}
			\put(0.75,-6.84){\line(1,0){17.15}}
			\put(0.75,-7.46){\line(1,0){17.15}}
			\put(0.75,-8.01){\line(1,0){17.15}}
			\put(0.75,-8.68){\line(1,0){17.15}}
			\put(0.75,-9.3){\line(1,0){17.15}}

		\end{picture}
		\begin{enumerate}[label={\arabic*.},leftmargin=1cm]
			\item Тема роботи: \tl@thema

			      \vspace{0.66cm}

			\item Строк здачі студентом закінченої роботи \quad  \tl@enddate
			\item Вихідні дані до роботи: \tl@sources
			      \vspace{0.66cm}
			\item Зміст розрахунково - пояснювальної записки (перелік питань, 
			які підлягають розробці): \tl@zapyska

			      \begin{picture}(0,0)

				      \put(-0.2,-1.28){\line(1,0){17.15}}
				      \put(4.75,-1.88){\line(1,0){12.15}}

			      \end{picture}
			\item  Перелік графічного матеріалу (із точним зазначенням
			 обов’язкових креслень): \tl@graphika

			\item Дата видачі завдання: \quad \tl@date
		\end{enumerate}
	}
	\endgroup%
}
\newcommand{\settitle}[2][]{%

	\begingroup%
	\setkeys{titlee}{#1}% Set new keys
	{\linespread{1}\selectfont
		\centering
		Міністерство освіти і науки України\\
	
		\tl@university
		\hrule
	
		{\scriptsize (повне найменування вищого навчального закладу)}

		\hspace{0.2cm}
		Кафедра \tl@katedra
		\hrule
	
		{\scriptsize (повна назва кафедри)}

		\vspace{3cm}
		\begin{center}
			\textbf{ \large \tl@type}
		\end{center}

		з  \textbf{<<\tl@discipline>>}

		\hrule
		%	\vspace{0.2cm}
		{\scriptsize (назва дисципліни)}
		\hspace{0.2cm}


		\setlength{\unitlength}{1cm}
		\begin{picture}(0,0)
			\put(-9,-1.85){\line(1,0){18}}
			\put(-9,-0.75){\line(1,0){18}}
			\put(-9,-1.3){\line(1,0){18}}

		\end{picture}


		на тему : \textbf{<<\tl@thema>>}

	}
	{\linespread{1.2}\selectfont
		\vspace{5cm}
		\hfill
		\begin{minipage}{0.5\linewidth}


			\begin{tabular}{lp{0.1\linewidth}ccp{0.145\linewidth}}
				Студента & \centering \tl@kurs & курсу, & групи & \tl@group \\
				\cline{2-2} \cline{5-5}
			\end{tabular}

			\vspace{0.1cm}
			\begin{tabular}{lc}
				спеціальності & \tl@specialty \\
				\hline
				\tl@author    & \tl@pidpys    \\
				\hline
			\end{tabular}
			\vspace{-0.8cm}\hspace{0.3cm}	{ \centering\scriptsize (прізвище
			та ініціали)  \hspace{2cm}(підпис студента)} \\

			\begin{tabular}{p{0.3\textwidth}p{0.6\textwidth}}
				Керівник: & \tl@posada \\
				\cline{2-2}
				\multicolumn{2}{c}{\tl@kerivnyk}\\
				\hline
			\end{tabular}
			\centering
			{\scriptsize (посада, вчене звання, науковий ступінь, прізвище
			та ініціали) }

			\setlength{\unitlength}{1cm}
			\begin{picture}(0,0)
				\put(5.5, 0){\line(1,0){3.45}}
				\put(7.35, -0.7){\line(1,0){1.6}}
				\put(3, -0.7){\line(1,0){1.6}}
				\put(-2.3, -1.65){{\fontsize{10}{10} {\selectfont Члени комісії
				:}}}
			\end{picture}
			\raggedright
			{\fontsize
			{10}{10} \selectfont Оцінка за національною шкалою:}

			\begin{tabular}{lp{0.15\textwidth}lp{0.15\textwidth}}
				\fontsize
				{10}{10} \selectfont Кількість балів: &   & \fontsize{10}{10}
			 \selectfont Оцінка ECTS &
			\end{tabular}

			%\hspace{-2.2cm}\vspace{0.2cm}
			\hspace{0.2cm}
			\begin{tabular}{p{0.26\textwidth}p{0.001\textwidth}
			C{0.56\textwidth}}
				  &   & \tl@kerivnyk \\
				\cline{1-1} \cline{3-3}
			\end{tabular}
			\vspace{-0.89cm}	\centering

			{\scriptsize \hspace{-0.7cm}	(підпис)  \hspace{2.4cm}   (прізвище
			 та ініціали) }

			\hspace{0.2cm}
			\begin{tabular}{p{0.26\textwidth}p{0.001\textwidth}
			C{0.56\textwidth}}
				  &   &   \\
				\cline{1-1} \cline{3-3}
			\end{tabular}
			\vspace{-0.89cm}	\centering

			{\scriptsize \hspace{-0.7cm}	(підпис)  \hspace{2.4cm}   (прізвище
			 та ініціали) }

			\hspace{0.2cm}
			\begin{tabular}{p{0.26\textwidth}p{0.001\textwidth}C
			{0.56\textwidth}}
				  &   &   \\
				\cline{1-1} \cline{3-3}
			\end{tabular}
			\vspace{-0.89cm}\centering

			{\scriptsize \hspace{-0.7cm}	(підпис)  \hspace{2.4cm}   (прізвище
			та ініціали) }
		\end{minipage}

	}


	\thispagestyle{firststyle} % колонтитул

	\endgroup%

}

\newcommand{\setkalendar}[2][]{
	\newgeometry{right=3cm,left=1cm}

	\begingroup%
	\setkeys{titlee}{#1}% Set new keys
	{\linespread{1}\selectfont
		\begin{center}
			\textbf{Календарний план}
		\end{center}
		\centering
		\begin{tabulary}{1.0\textwidth}{|C{1cm}|L|C{3cm}|C{2.5cm}|}
			\hline
			\fontsize{10}{10} \selectfont № п/п &
			 \fontsize{10}{10} \selectfont Назва етапів курсового проекту
			( роботи )& \fontsize{10}{10} \selectfont Строк виконання етапів
			проекту (роботи) & \fontsize{10}{10} \selectfont Примітки \\
			\hline
			#2
			& & & \\ \hline
			& & & \\ \hline
			& & & \\ \hline
			& & & \\ \hline
			& & & \\ \hline
			& & & \\ \hline


		\end{tabulary}

		\vspace{1cm}\hspace{-0.5cm}
		\raggedright

		\begin{tabular}{p{2cm}p{5cm}p{2cm}C{5cm}}
			Студент &   &   & \tl@author \\
			\cline{2-2} \cline{4-4}
		\end{tabular}
		\centering

		{\scriptsize \hspace{3.5cm}	(підпис)  \hspace{5.4cm}   (прізвище, 
		ім’я, по батькові) }
		\begin{tabular}{p{2cm}p{5cm}p{2cm}C{5cm}}
			Керівник &   &   & \tl@kerivnyk \\
			\cline{2-2} \cline{4-4}
		\end{tabular}
		\centering

		{\scriptsize \hspace{3.5cm}	(підпис)  \hspace{5.4cm}   (прізвище,
		ім’я, по батькові) }

		\hspace{0.5cm}
		\raggedright
		\vspace{1cm}
		\begin{tabular}{c C{5cm} }
			Дата & \tl@fulldate \\
			\cline{2-2}
		\end{tabular}
	}
	\endgroup%
	\restoregeometry
}

\makeatother





			
			\usepackage{fontspec}
\usepackage{xecyr}
\usepackage{polyglossia}
\setmainlanguage{ukrainian}
\usepackage{xunicode, xltxtra}
\usepackage{cmap}	
\defaultfontfeatures{Ligatures=TeX}

\setmainfont{Times New Roman} 
\newfontfamily\cyrillicfont{Times New Roman}
\setotherlanguage{english}
\setmonofont{FreeMono}

%% Название документа
\ESKDtitle{\ESKDfontIII Створення шаблону для курсової в Latex}
\ESKDauthor{ ****** А.В. }
\ESKDchecker{****** О.Б. }

\ESKDsignature{ КПКН 20.055.014.000 ПЗ }
\ESKDcolumnIX{{\small ТНТУ, ФІС, КН ****}}

\renewcommand{\ESKDcolumnVIIname}{\ESKDfontII Аркуш}

\ESKDsectAlign{section}{Center}
\ESKDsectStyle{section}{\normalsize \bfseries \uppercase}
\ESKDsectStyle{subsection}{\normalsize \bfseries}
\ESKDsectSkip{section}{0pt}{0.8cm}
\ESKDsectSkip{subsection}{0.8cm}{0.5cm}
\ESKDsectSkip{subsubsection}{0.5cm}{0.1pt}

\end{lstlisting}

\subsection{Тіло документа}

В першу чергу необхідно застосувати створені команди для титульної сторінки і завдання з календарним планом, передаючи необхідні параметри:

\addimghere{6}{1}{Використання команд для перших 3 сторінок}{}

Команда \verb|\ESKDthisStyle{empty}| робить стиль сторінки без рамки. Для зручності ці команди буде винесено в окремий файл \textit{titlepage.tex}. Також розділи і все інше буде окремо:

\begin{lstlisting}
			\begin{document}
			
			\renewcommand{\ESKDcolumnVIIname}{\ESKDfontIII Аркуш}
			
\ESKDthisStyle{empty}
	
	\settitle[
	thema= Створення шаблону для курсової в Latex ,
	university=Тернопільський національний технічний університет імені Івана Пулюя,
	katedra=комп'ютерних наук,
	type=КУРСОВА РОБОТА,
	discipline={Комп’ютерні системи обробки текстової, графічної та мультимедійної
	інформації},
	kurs=3,
	group=****,
	specialty=122 Комп'ютерні науки,
	author=***********,
	kerivnyk=***********,
	]
	
	
	\newpage
	\ESKDthisStyle{empty}
	
	\setzavdannia[
	university=Тернопільський національний технічний університет імені Івана Пулюя,
	katedra= комп'ютерних наук,
	kurs= 3,
	discipline ={Комп’ютерні системи обробки текстової, графічної та мультимедійної інформації},
	specialty=  122 Комп'ютерні науки, 
	group= ****, 
	semestr= 6,
	author=******** Андрія Володимировича,
	thema= Створення шаблону для курсової в Latex \\ \ \\ \ \\ ,
	sources=,
	zapyska={Створення шаблону для курсової в Latex, Реферат, Зміст, Вступ, Розділ 1.Latex.Історія.Версії, Розділ 2. Інструментарій Latex. Розділ 3. Розробка шаблону для курсової, Розділ 4.
		Порівняння шаблону з іншими аналогами. Висновок, Список літературних
		джерел, Додатки \\ \ \\ \ \\ \  },
	graphika={Презентація – ХХ слайдів у форматі .PPTX, x додатки},
	date=01.05.2020,
	enddate=22.06.2020
	]

	\newpage
	\ESKDthisStyle{empty}
	\setkalendar[
		author=****** А.В.,
		kerivnyk=******* О.Б.,
		fulldate=1 травня 2020р.
	]{
	1 & Дата видачі індивідуального завдання & 01.05.2020 & Виконано \\ \hline
	2 & Підготовка до виконання курсової & 04.05.2020 & Виконано \\ \hline
	3&Формування структури курсової&08.04.2020&Виконано \\ \hline
	4&Збір інформації для про Latex&11.05.2020&Виконано \\ \hline
	5&Написання 1 розділу&13.05.2020&Виконано \\ \hline
	6&Написання 2 розділу& 16.05.2020&Виконано \\ \hline
	7&Створоення шаблону & 20.06.2020 & Виконано \\ \hline
	8&Написання 3 розділу&25.05.2020&Виконано \\ \hline
	9&Пошук інших шаблонів і порівняння&28.05.2020&Виконано \\ \hline
	10& Написання 4 розділу & 01.06.2020& Виконано \\ \hline
	11&Висновок курсової&05.06.2020&Виконано \\ \hline
	12&Захист курсової роботи& 22.06.2020 & \\ \hline
	  & & & \\ \hline
	    & & & \\ \hline
	      & & & \\ \hline
	        & & & \\ \hline
	               
	}

			\ESKDthisStyle{empty}

\begin{center}
	\uppercase{\textbf{Реферат}}
\end{center}
	
	Курсова робота // Створення шаблону для курсової в Latex// ******* Андрій Володимирович // Тернопільський національний
	технічний
	університет
	імені
	Івана
	Пулюя,
	факультет
	комп’ютерно-інформаційних систем та програмної інженерії, кафедра комп’ютерних наук,
	група **** // Тернопіль, 2020, сторінок \pageref{LastPage}, рисунків \totalfigures{} , джерел \ESKDtotal{bibitem} , таблиць \totaltables{}
	креслень 0, додатків \ESKDtotal{appendix}.
	
	Ключові слова: Latex, шаблон, шаблон для Latex, мова розмітки даних, Tex, типографія.
	
	В даній курсовій роботі було проведено опис систему для видавництва документів  Latex, та на його основі було створено шаблон для курсових робіт. Після створення здійснено порівняння з іншими шаблонами.
\newpage
			\ESKDthisStyle{formII}
			
			\tableofcontents
			\anonsection{Вступ}{sec:intro}
\ESKDthisStyle{formII}

Написання та оформлення текстів у високій якості потребує спеціального програмного забезпечення(ПЗ). Особливо важливо мати можливість використовувати системи набору тексту для написання документів, книг, статтей та інших форм видавничої справи в науковій галузі. Через необхідність написання складних текстів з красивим оформленням для наукових праць -- ця тема є актуальною для мене, оскільки, навчаючись в університеті потрібно вести звітність про виконання завдань, і найкращий спосіб вирішення цієї проблеми - це розробити шаблон для робіт, і сконцентруватися на самому завданні.

Багато ПЗ можна привести для прикладу, но ми зосередимося на наборі макророзширень системи комп'ютерної верстки \textbf{Latex}, яка розглядається в цій курсовій.

В ході наступних розділів буде розглянуто історію створення Latex, його основні можливості. Далі буде проводитись розробка шаблону для курсових робіт. Маючи такий шаблон, його можна буде легко модифікувати для використання в лабораторних роботах. В останньому розділі буде порівняння готового шаблону з іншими шаблонами, знайденими в мережі інтернет.

			\newSection{Історія Latex}{sec:seccc}{\ESKDfontV}


Говорячи про Latex потрібно вказати що він є, надбудовою над TEX - оригінальною системою текстового препроцесора. Все що можна зробити в Latex можна і в оригінальні системі TEX. Програмне забезпечення для набору тексту TEX було розроблено Дональдом Е. Кнутом в кінці 1970-х. Він був випущений з ліцензією з відкритим кодом і став стандартом наукового видавництва. Тепер TEX використовується для набору та публікації більшої частини світової інформації наукової літератури з фізики та математики.

Однією з найважливіших причин, по якій люди використовують LATEX, є те, що він відокремлює зміст документа від стилю. Це означає, що після написання вмісту вашого документа ми можемо легко змінити його вигляд. Так само ви можете створити один стиль документа, який можна використовувати для стандартизації зовнішнього вигляду безлічі різних документів. Це дозволяє науковим журналам створювати шаблони для публікацій. Ці шаблони мають заздалегідь зроблений макет, що означає, що потрібно додавати лише вміст.

Основний вплив для широкого розповсюдження здійснив
Леслі Лампорт у союзі з Пітером Гордоном в Аддісоні-Веслі,
версія 2.09 від приблизно середини 80-х років, яка походить від  системи TEX Дональда Кнута, яка досить швидко поширилася поза спільнотою північноамериканських математиків, які підтримували  розвиток TEX від його створення як один із "особистих інструментів продуктивності" Дона, створеного просто щоб забезпечити швидке завершення та типографічну якість його книги <<Мистецтво комп'ютерного програмування>> \cite{Knuth} Менш прямий, але, ймовірно, ширший вплив випливає з того, що вона є першою широко використовуваною мовою для опису логічної структури широкого кола документів таким \ чином \ її впровадження філософії логічного \ проектування, \ яке \newpage \noindent використовується Brain Reid in Scribe \cite{Reid}: "під час написання документа ви повинні
переймайтеся його логічним змістом, а не його візуальним оформленням."

Тоді Latex по-різному описувався як "TEX для мас" та "Написання, звільнене від негнучкого керування форматом ". Не зовсім
зрозуміло, чи все це було зроблено навмисне Леслі як особливість дизайну, але, безумовно, він не очікував що згодом, здійснить такий широкий вплив. Поширеність Latex була, навіть у кінці 1980-х років, більшою порівняно з більшістю некомерційного програмного забезпечення того часу. Хороші новини швидко поширювалися і 1994 року Леслі міг написати «Latex зараз надзвичайно популярний у науковій та академічній спільнотах, і він широко використовується у індустрії." Але цей рівень повсюдності все-таки був мізерним порівняно з сьогоднішнім днем, коли він став для багатьох професіоналів, невід'ємним інструментом, присутність якого є дуже важливою.

\subsection{Складові Latex(TeX)}

Робота з Latex пов'язана з так званими <<рівнями>> \cite{levels}:
\begin{enumerate}[label={\arabic*.},leftmargin=1.45cm]
	\item \textbf{Дистрибутиви}: MiKTeX, TeX Live,… Це великі колекції програмного забезпечення, пов'язаного з TeX, для завантаження та встановлення. Коли хтось каже: «Мені потрібно встановити TeX на свою машину», він зазвичай шукає дистрибутив.
	\item \textbf{Редактори}: Emacs, vim, TeXworks, TeXShop, TeXnicCenter, WinEdt,… ці редактори - це те, що ви використовуєте для створення файлу документа. Деякі (наприклад, TeXShop) присвячені спеціально TeX, інші (наприклад, Emacs) можуть використовуватися для редагування будь-яких файлів. Документи TeX не залежать від будь-якого конкретного редактора; сама програма набору TeX не включає редактор.
	\item \textbf{Компілятори}: TeX, pdfTeX, XeTeX, LuaTeX,… це виконувані бінарні файли, які реалізують різні варіанти TeX. Наприклад, pdfTeX реалізує прямий вихід у форматі PDF (якого немає в оригінальному TeX Кнута), LuaTeX забезпечує доступ до багатьох внутрішніх систем через вбудовану мову Lua і т.д. Коли хтось каже: "TeX не може знайти мої шрифти", вони зазвичай мають на увазі компілятор.
	\item \textbf{Формати}: LaTeX, звичайний TeX,… Це мови на основі TeX, якими фактично пишуть документи. Коли хтось каже, що "TeX дає мені невідому помилку", вони зазвичай мають на увазі формат. (До речі, "LaTeX" вже багато років означає "LaTeX2e".)
	\item \textbf{Пакети}: geometry, lm, ... Це доповнення до основної системи TeX, розроблені незалежно, надаючи додаткові функції набору тексту, шрифти, документацію тощо. Пакет може або не може працювати з будь-яким заданим форматом та/або компілятором; наприклад, багато є розроблено спеціально для LaTeX, але є і багато для інших. Сайти CTAN надають доступ до переважної більшості пакетів у світі TeX; CTAN, як правило, є джерелом, яке використовується дистрибутивами.
\end{enumerate}	

	Особливу увагу треба звернути на компілятори, в цій курсовій я використовую \textit{Xelatex}. Його особливістю є використання системний шрифтів, на відміну від Latex, який має свої вбудовані. Для оформлення документів в Україні згідно з ДСТУ 3008:2015 потрібно використовувати шрифт \textit{Times new Roman}, що і дозволяє зробити Xelatex. 
	
	\subsection{TeX і Latex}
	
	З моменту заснування розробка Latex та відповідного програмного забезпечення була повністю переплетена з розвитком самого TEX. 
	Хоча є багато сумнівів щодо корисністі деяких аспектів фундаментальних моделей та дизайну
	TEX як двигуна форматування тексту, він був тоді і залишається зараз (the
	початок тисячоліття) єдиним зрілим, широко доступним,
	програмованим і дуже гнучким компілятором для тексту. Таким чином для
	 Леслі в той час, як і для нас зараз, це єдиний вибір фундаменту
	для практичної автоматизації високоякісного форматування.
	
	У дизайні Latex Леслі свідомо дозволив основному компілятору TEX безпосередньо впливати на більшість текстових питань.
	У типових системах обробки тексту тієї епохи, включаючи TEX,
	основні методи обробки тексту документа такі: кожен вхідний маркер, що надсилається до системи, обробляється
	як складна імперативна команда. У таких системах "символ в
	тексті документа", як правило, подія на клавіатурі або маркер на
	вхідний буфер, не просто призначений викликати створення
	'елемента у рядку' в 'об'єкті текстового класу', такий 'рядок'
	врешті обробляється деяким іншим модулем системи, або
	навіть зовнішніми програмами.
	
	
	TEX був розроблений у цій імперативній парадигмі, оскільки це призводить до високоефективності(і в часі, і в просторі) машини, незважаючи на те, що "набір тексту" є для TEX відносно складним обчислювальним процесом, що включає, в першу чергу, оптимізацію	вибіру гліфа та позиціонування над цілими абзацами контрольоваий	за допомогою високо настроюваного алгоритму динамічного програмування. Однак, оскільки цей процес набору даних був оптимізований для швидкості, то робити що-небудь, що недоступно в рамках цього монолітного процесу (як визначено дизайном TEX), є важким у здійсненні та помітно неефективним у використанні. Такі процеси мають центральне значення для	якості набору і особливо важливі в наборі	інших мов, крім американської англійської. Вони включають модифікацію	важливих підпроцесів, таких як вибір гліфу (як для лігатур)	та їх розміри і розміщення; переноси та вирівнювання.
	
	
	\subsection{Версії}
	
	LaTeX2e - це поточна версія LaTeX, відколи вона замінила LaTeX 2.09 у 1994 році. Станом на 2019 рік LaTeX3, який розпочався створюватися на початку 1990-х років, досі розвивається. Планові функції включають покращений синтаксис, підтримку гіперпосилання, новий інтерфейс користувача, доступ до довільних шрифтів та нову документацію.\cite{latex3}
	
	Існують численні комерційні реалізації всієї системи TeX. Постачальники систем можуть додавати додаткові функції, такі як додаткові шрифти та підтримку по телефону. LyX - це безкоштовний, WYSIWYM-процесор візуального документа, який використовує LaTeX для бек-енду. TeXmacs - безкоштовний редактор WYSIWYG з аналогічними функціями, як LaTeX, але з іншим механізмом набору тексту. Інші редактори WYSIWYG, які використовують LaTeX, включають Scientific Word у MS Windows. Та BaKoMa TeX для Windows, Mac та Linux.
	
	Доступно декілька дистрибутивів TeX, що підтримуються спільнотою, зокрема TeX Live (багатоплатформна), teTeX (застаріла на користь TeX Live, UNIX), fpTeX (застаріла), MiKTeX (Windows), proTeXt (Windows), MacTeX (TeX Live з додаванням специфічних програм для Mac), gwTeX (Mac OS X) (застарілий), OzTeX (Mac OS Classic), AmigaTeX (більше не доступний), PasTeX (AmigaOS, доступний у сховищі Aminet) та Auto-Latex Equations (Додаток Google Docs, який підтримує команди MathJax LaTeX).
	
	\subsection{Сумісність та конвертація}
	
	Документи LaTeX (* .tex) можна відкрити будь-яким текстовим редактором. Вони складаються з простого тексту та не містять прихованих кодів форматування чи двійкових інструкцій. Крім того, документи TeX можна поширити у формат Rich Text (* .rtf) або XML. Це можна зробити за допомогою безкоштовних програм LaTeX2RTF або TeX4ht. LaTeX також може бути конвертовано у PDF-файли за допомогою розширення LaTeX pdfLaTeX. Файли LaTeX, що містять текст Unicode, можуть бути оброблені в PDF-файли за допомогою пакету \textit{inputenc} або розширеннь Tee XeLaTeX і LuaLaTeX.
	
	\begin{itemize}
		\item HeVeA - це перетворювач, написаний на Ocaml, який перетворює документи LaTeX у HTML5. Він ліцензований відповідно до Q Public License.
		
		\item LaTeX2HTML - це перетворювач, написаний на Perl, який перетворює документи LaTeX в HTML. Таким чином, наприклад, наукові праці, головним чином набрані для друку, можна розмістити для перегляду в Інтернеті. Він ліцензований відповідно до GNU GPL v2.
		
		\item LaTeXML - це безкоштовне програмне забезпечення публічного домену, написане на Perl, яке перетворює документи LaTeX у різноманітні структуровані формати, включаючи HTML5, epub, jats, tei.
		\item Pandoc - це "універсальний конвертер документів", здатний трансформувати LaTeX у безліч різних форматів файлів, включаючи HTML5, epub, rtf та docx. Він ліцензований відповідно до GNU GPL v2.
	
	\end{itemize}

	LaTeX став стандартом для набору математичних виразів в наукових документах. Таким чином, існує кілька інструментів перетворення, орієнтованих на математичні вирази LaTeX, такі як перетворювачі в MathML  або Computer Algebra System.
	
	\begin{itemize}
		\item  Mathoid - це веб-конвертер який використовує Node.js, він перетворює математичні входи, такі як LaTeX, у формати MathML та зображення, включаючи SVG та PNG. Він використовується у Вікіпедії для відображення математики.
		\item TeXZillais перетворювач на JavaScript з LaTeX в MathML. Це один з найшвидших перетворювачів LaTeX в MathML.
		\item  LaCASt - це перетворювач, написаний на Java, який перетворює семантичний діалект LaTeX в Maple та Mathematica.
	\end{itemize}


	\subsection{Особливості Latex}
	
	Створення LaTeX документа це програмування: Ви створюєте текстовий файл в LaTeX-розмітці, макроси LaTeX обробляють його і видають конкретний документ.
	
	Такий підхід відрізняється від використовуваного в WYSIWYG (What You See Is What You Get - те, що ви бачите, то і отримуєте) програмах, таких, як Openoffice Writer або Microsoft Word.
	
	В LaTeХ: 
	
	\begin{itemize}
		\item Під час редагування документа Ви не можете (зазвичай) побачити його остаточний варіант.
		
		\item 	Вам, як правило, потрібно знати необхідні команди розмітки LaTeX.
		
		\item Інколи складно отримати необхідний вигляд документа.
		
	\end{itemize}


		З іншого боку, у LaTeX є і переваги:
		
	\begin{itemize}
		\item Оформлення тексту відокремлено від вмісту. Ви повною мірою зосереджуєтеся на структурі та вмісті документу і забуваєте про те, як буде виглядати друкований варіант.
		\item Стиль, шрифти, оформлення таблиць і малюнків т. д. узгоджено у всьому документі.
		\item Одне і те саме оформлення можна використовувати для будь-якого числа документів.
		\item Легко набирає математичні формули.
		\item Легко створюються алфавітні вказівники, посилання та бібліографічні списки.
		\item Більші документи можуть бути розподілені на декілька файлів і працювати з ними окремо, в тому числі з використанням системи управління версіями.
		\item Вам не потрібно вручну налаштовувати шрифти, розмір тексту, високий шрифт - за це відповідає  LaTeX.
		\item  Вам доведеться правильно структурувати ваш документ.
		\item Файли з вихідними текстами можна переглянути і змінити в любому текстовому редакторі.
	
	\end{itemize}

	Підхід LaTeX до створення документа можна назвати WYSIWYM (What You See Is What You Mean - що бачиш, то і думаєш): під час набору тексту Ви не бачите остаточний варіант документа, тільки логічну структуру цього документа. Про оформлення замість Вас подбає LaTeX.

	
		
			\newSection{Інструментарій Latex}{sec:sec2}{\ESKDfontV}

Почати роботу з Latex необхідно зі створення нового документа з розширенням \textit{.tex}, при умові що у Вас встановлені всі необхідні пакети Latex. Як вже було сказано раніше треба мати встановлений дистрибутив та редактор. В дистрибутиві будуть знаходитися пакети та компілятори для створення документів. В якості дистрибутива встановлено \textit{texlive}, а редактор вихідного документа \textit{TexStudio}. TexStudio надає зручні можливості для роботи з  командами,наявні автодоповнення, сполучення клавіш - це все робить його  середовищем  для розробки Latex.

\subsection{Перший документ}

Створюємо файл з розширенням .tex та записуємо в нього наступні рядки:

\begin{lstlisting}
	\documentclass{article}
	
	\begin{document}
	
	\end{document}
\end{lstlisting}

На виході буде пустий документ. Перший рядок вказує на клас документу.Форматування за замовчуванням у документах LATEX визначається класом, який використовується цим документом. Стандартний вигляд можна змінити, а додаткові функції можна додати за допомогою пакета. Імена файлів класу мають розширення .cls, імена файлів пакунків мають розширення .sty.

Для різних типів документів потрібні різні класи, тобто для резюме буде потрібен інший клас, ніж для  наукового документа. У цьому випадку клас - це \textit{article}, найпростіший і найпоширеніший клас LATEX. \ Інші \ типи \ документів, \ над \newpage \noindent якими ви можете працювати, можуть вимагати різних класів, таких як \textbf{book} або \textit{report}.

Далі з команди \textit{begin} починається так зване тіло документа. В ньому буде знаходитися вміст документа, аж до \textit{end}

Щоб переглянути документ треба провести компіляцію. До прикладу такою командою:

\begin{lstlisting}
	pdflatex <your document> 
\end{lstlisting}

\subsection{Преамбула документа}

У попередньому прикладі текст був введений після команди  \verb|\begin {document} |. Все у .tex-файлі до цього моменту називається преамбулою. У преамбулі ви визначаєте тип документа, який ви пишете, мову, якою ви пишете, пакети, які ви хочете використовувати та декілька інших елементів. Наприклад, звичайна преамбула документа виглядатиме так:

\begin{lstlisting}
	\documentclass[12pt, letterpaper]{article}
	\usepackage[utf8]{inputenc}
\end{lstlisting}

\verb|\documentclass[12pt, letterpaper]{article}|. Як було сказано раніше, це визначає тип документа. Деякі додаткові параметри, що входять до квадратних дужок, можуть передаватися команді. Ці параметри повинні бути розділені комами. У прикладі додаткові параметри встановлюють розмір шрифту \textit{(12pt)} та розмір паперу (letterpaper). Звичайно, можна використовувати інші розміри шрифту \textit{(9pt, 11pt, 12pt)}, але якщо не вказано жодного, розмір за замовчуванням - 10pt. Що стосується розміру паперу, інші можливі значення - це папір формату А4 та legalpaper; 

\verb|\usepackage[utf8]{inputenc}|. Це кодування документа. Його можна опустити або змінити на інше кодування, але рекомендується utf-8. Якщо вам конкретно не потрібно інше кодування, або якщо ви не впевнені в цьому, додайте цей рядок до преамбули.


Для форматування тексту, задання формату сторінки, додавання графічних елементів та задання всіх можливих параметрів і налаштувань використовуються різні команди, які можна задати в класі документу або використовуючи додаткові пакети. 

\subsection{Макет сторінки}

За замовчуванням всі параметри документа встановлюють класи документів. Однак,
якщо ви хочете змінити ці параметри , є команди, які дозволяють вам це зробити. Команди, що контролюють функції, що стосуються всього документа, повинні бути розміщені в преамбулі.

\subsubsection{Параграфи}\label{par}

Щоб почати новий абзац, залиште порожній рядок або використовуйте команду \verb|\par|. Команди \verb|\parindent| і \verb|\parskip| задяють відступ абзацу та розділення абзацу. 

Для задання міжрядкового інтервалу можна використати команду \verb|\renewcommand{\baselinestretch}{1.5}|, яка встановить його на 1.5.

Абзацний відступ задається командою \verb|\parindent 1.25cm|.\par Перейти на новий рядок \verb|\\ або \par|

\subsubsection{Вирівнювання тексту}

За замовчуванням LATEX вирівнює ваш текст горизонтально, так що лівий і правий відступи є гладкими. Якщо ви віддаєте перевагу "вирівнювання справа" , ви можете використовувати:
\verb|\raggedright|
Зауважте, що це має побічний ефект для відступів абзацу. Якщо ви хочете залишити відступ абзаців, потрібно спеціально задати його (тобто \verb|\parindent = 1.5em|) після \verb|\raggedright| команди.

\subsubsection{Коментарі}

Як і будь-який код, який ви пишете, часто корисно включати коментарі. Коментарі - це фрагменти тексту, які ви можете включити в документ, які не будуть надруковані, і жодним чином не вплинуть на документ. Вони корисні для організації вашої роботи, ведення приміток або коментування рядків / розділів під час налагодження. Щоб зробити коментар у LATEX, просто напишіть символ \verb|%| на початку рядка.

\subsubsection{Вигляд тексту}

Зараз ми розглянемо кілька простих команд форматування тексту.

\begin{itemize}
	\item Жирний: Жирний текст у LaTeX пишеться командою \verb|\textbf {...}|
	\item Курсив: Курсивний текст у LaTeX пишеться командою \verb|\textit {...}|
	\item Підкреслення: Підкреслений текст у LaTeX пишеться командою \verb|\underline {...}|
\end{itemize}

LaTeX має кілька команд-модифікаторів розміру шрифту (від найбільших до найменших):

\begin{lstlisting}
		\Huge
		\huge
		\LARGE
		\Large
		\large
		\normalsize (default)
		\small
		\footnotesize
		\scriptsize
		\tiny
\end{lstlisting}

\subsubsection{Додавання зображень, таблиць}

Стандартний комплект графіки LaTeX включає два пакети для імпорту графіки:
\textit{graphics} та \textit{graphicx}. Розширений пакет, \textit{graphicsx}, забезпечує більш зручний спосіб подачі параметрів і рекомендується. Тому перший крок - це введіть у свою преамбулу команду:
\verb|\usepackage {graphicsx}|. Цей пакет визначає нову команду під назвою \verb|\includegraphics|, яка дозволяє вам вказувати назву графічного файлу, а також надавати необов'язкові аргументи для масштабування чи обертання. Отже, для прикладу потрібно вставити графіку (названу, наприклад, myfigure.eps
або myfigure.pdf) використовуйте команду \verb|\includegraphics|, наприклад:

\begin{lstlisting}
	\includegraphics[width=4in]{myfigure}
\end{lstlisting}


Tabular середовище є стандартним в LATEX для створення таблиць. Ви повинні вказати параметр для цього середовища, в цьому випадку {c c c}. Це говорить про те, що LATEX буде три стовпці і текст у кожному з них повинен бути в центрі. Ви також можете використовувати r, щоб вирівняти текст праворуч, а l - для вирівнювання ліворуч. Символ \& використовується для визначення розділення у записах таблиці. У кожному рядку завжди повинно бути на один менше символів розділення, ніж кількість стовпців. Для переходу до наступного рядка таблиці використовуємо команду нового рядка \verb|\\|.

\begin{lstlisting}
	\begin{tabular}{ c c }
		cell 1 & cell 2 \\
		cell 3 & cell 4 
	\end{tabular}
\end{lstlisting}

Ви можете додати межі за допомогою команди горизонтальної лінії \verb|hline| та параметра вертикальної лінії |.

\begin{lstlisting}
	\begin{tabular}{ |c|c| }
		\hline 
		cell 1 & cell 2 \\ \hline
		cell 3 & cell 4 \\ \hline
	\end{tabular}
\end{lstlisting}


{| c | c | c | }: Тут вказується, що у таблиці будуть використані три стовпчики, розділені вертикальною лінією. | символ вказує, що ці стовпці повинні бути розділені вертикальною лінією. 

\verb|hline|: буде вставлена горизонтальна лінія. Тут ми ввели горизонтальні лінії вгорі та внизу таблиці. Немає обмежень у кількості разів, коли ви можете використовувати \verb|\hline|.

\subsubsection{Структурування}

Команди для організації документа відрізняються залежно від типу документа, найпростішою формою організації є секціонування, доступне у всіх форматах.

Команда \verb|section{}| позначає початок нового розділу, всередині дужок встановлюється заголовок. Нумерація розділів є автоматичною і її можна відключити, включивши * в команду розділу як \verb|\section*{}|. Ми також можемо мати підрозділи \verb|\subsection{}|, і пункти  \verb|sububsection{}|. 

\subsubsection{Підписи до зображень, таблиць}\label{capt}

В \textit{figure} або \textit{table}
середовищах, ви можете надати підпис із командою \verb|\caption {caption text}|. Зазвичай підписи для таблиці вводяться над таблицею, а підпис до зображення
нижче зображення.

Якщо потрібно якось модифікувати вигляд підпису то можна скористатися пакетом \textit{caption}. Наприклад відцентрувати підпис зображення, та змінити слово Рис. на Рисунок:

\begin{lstlisting}
		\captionsetup[figure]{justification=centering, labelsep=defffis,
		format=plain}		
		\addto\captionsukrainian{\renewcommand{\figurename}{ Рисунок }}
\end{lstlisting}


\subsection{Створення та перевизначення команд}

Щоб додати свої власні команди, використовуйте команду:

\begin{lstlisting}
		\newcommand{\name}[num]{definition}
\end{lstlisting}


В основному, команда вимагає двох аргументів: ім'я команди, яку ви хочете створити, і визначення команди. Зауважте, що ім'я команди можна, але не потрібно вкладати в дужки, як вам подобається. Аргумент num у квадратних дужках є необов’язковим і визначає кількість аргументів, які приймає нова команда (можливо до 9). Якщо він відсутній, він за замовчуванням дорівнює 0, тобто аргумент не дозволений.

Для прикладу створимо команду додавання зображення:

\begin{lstlisting}
		\newcommand{\addimg}[4]{
			\begin{figure}
				\centering
				\includegraphics[width=#2\linewidth]{#1}
				\caption{#3} \label{#4}
			\end{figure}
		}
\end{lstlisting}

Команда отримує 4 аргументи, для використання аргументу вводять \# та номер параметра.

Щоб перевизначити команду треба ввести \verb|\renewcommand{cmd}{def}|. Ми вже використовували її для визначення підпису до малюнків на ст. \pageref{capt} та для міжрядкового інтервалу ст. \pageref{par}

Ще багато команд можна було б описати, проте цього достатньо для початку. В наступному розділі буде відбуватись створення шаблону, в процесі використань нових команд буде надано  їхній короткий опис.


 
			\newSection{Розробка шаблону}{sec:sec3}{}

Почати роботу з виготовлення шаблону потрібно з вимог до завдання, а саме: потрібно розробити шаблон для курсових робіт з дотриманням ДСТУ 3008:2015. Ось деякі плавила оформлення:

\begin{itemize}
	\item Аркуші формату А4 (210х297 мм).
	\item Шрифт 14 розміру з 1,5 інтервалом.
	\item  віддаль між рядками повинна бути однакова і рівна 8-10 мм.
	\item  відстань між заголовками підрозділів або пунктів і подальшим або попереднім текстом 15-20 мм;
	\item  відстань між назвою розділу і назвою підрозділу або пункту 18-22 мм;
	\item  абзацний відступ повинен бути однаковим впродовж усього тексту записки і дорівнювати 10-15 мм.
\end{itemize}

Ще одною умовою є додавання рамок до роботи. І останній пункт, додатковий - зверстати титульну сторінку.

\subsection{Написання преамбули}

\subsubsection{Клас документа та кодування}

Для звичайної курсової роботи, без рамок, можна було б скористатися класом документа \textit{article} або \textit{extarticle}, проте довелося б реалізовувати рамки вручну. На просторах інтернету можна знайти колекцію пакетів \textit{eskdx}, який надає нам такий функціонал. Колекція пакетів надає 3 класи: \textit{eskdtext}(для текстової документації), \textit{eskdgraph}(для креслення схем) і \textit{eskdtab}(для документів, разбитих на графи). Для нашого випадку підходить eskdtext, його і використаємо.

\newpage

Ось як буде виглядати підключення класу:

\begin{lstlisting}
		\documentclass[14pt,ukrainian,utf8, simple, pointsection,
		floatsection ]{eskdtext} 
\end{lstlisting}

В додаткових параметрах вказано 14 розмір шрифту, українську мову(вибір тільки з 2, інша - російська), кодування, simple - відображати тільки основні графи, останні два параметри вказують на нумерування пунктів та фігур в межах секцій.

Для роботи з кирилицею потрібно підключити мовний пакет. На вибір є 2 основних: \textit{babel} та \textit{poliglossia}, скористаємося другим. Перевага другого пакету в тому що кириличні символи кодуються правильно, навідміну від babel, який робить заміну \textit{і} на латиницю. Це допоможе при проходженні роботи на антиплагіат.

Підключаємо його та ще пару пакетів для кодування кирилиці:

\begin{lstlisting}
		\usepackage{fontspec}
		\usepackage{xecyr}
		\usepackage{polyglossia}
		\setmainlanguage{ukrainian}
		\usepackage{xunicode, xltxtra}
\end{lstlisting}


\subsubsection{Специфічні налаштування класу}


Пакети ESKDX надають багато налаштувань, їх ми будемо зберігати в окремому файлі з назвою \textit{ESKDXconfig.tex}. Він є частиною преамбули тож і підключатиметься там. В цьому файлі зберігатимуться попередні команди і також для налаштування шрифтів.

\begin{lstlisting}
		\defaultfontfeatures{Ligatures=TeX}
		\setmainfont{Times New Roman} 
		\newfontfamily\cyrillicfont{Times New Roman}
		\setotherlanguage{english}
		\setmonofont{FreeMono}
\end{lstlisting}

Вказавши деякі команди можна створити титульну сторінку, проте її формати не підходить, тому ми будемо писати свою титулку в наступному підрозділі. Зараз тільки вкажемо деякі необхідні команди для задання інформації на рамках.
\begin{lstlisting}
			\ESKDsignature{ КПКН 20.055.014.000 ПЗ
			\ESKDcolumnIX{{\small ТНТУ, ФІС, КН СН-31}}
			\ESKDtitle{\ESKDfontIII Створення шаблону для курсової в Latex}
			\ESKDauthor{ **** А. В. }
			\ESKDchecker{ **** О. Б. }
			
\end{lstlisting}

Останнє в цьому файлі це налаштування стилів відображення секцій:

\begin{lstlisting}
			\ESKDsectAlign{section}{Center}
			\ESKDsectStyle{section}{\normalsize \bfseries \uppercase}
			\ESKDsectStyle{subsection}{\normalsize \bfseries}
			\ESKDsectSkip{section}{0pt}{0.8cm}
			\ESKDsectSkip{subsection}{0.8cm}{0.5cm}
			\ESKDsectSkip{subsubsection}{0.5cm}{0.1pt}
\end{lstlisting}

Секції центруємо, робимо 14 шрифтом, жирний та все у верхньому регістрі. Підсекції - те саме, тільки у звичайному регісті. Також налаштовано відступи між секціями.

Для цього файлу - це все, повний обсяг буде наведено в додатках.

\subsubsection{Створення титульної сторінки}

Налаштування титульної сторінки будуть знаходитися в окремому файлі \textit{title.tex}.

Спершу налаштуємо колонтитули, оскільки це найлегше що можна зробити зараз:
\begin{lstlisting}
		\usepackage{fancyhdr} % Колонтитули
		\pagestyle{fancy}
		
		\fancypagestyle{firststyle}{
		\renewcommand{\headrulewidth}{0pt}
		\fancyfoot{}
		\cfoot{Тернопіль 2020}
		}
		\renewcommand{\headrulewidth}{0pt}
		\fancyfoot{}
		\fancyhead{}
\end{lstlisting}

Використали пакет \textit{fancyhdr} і створили стиль колонтитула для сторінки в якому по центрі внизу записали необхідні дані. 

Створимо нову команду для титульної сторінки, яка буде приймати аргументи для їх встановлення на сторінку: найменування вищого навчального закладу, кафедра, назва роботи, тема, дисципліна, і т.д. Також буде створено ще 2 команди для завдання та календарного плану. Команди можуть приймати до 9 параметрів, в нашому випадку їх є більше, тому скористаємося пакетом \textit{keyval}, який дозволить використовувати опційні параметри, які не обмежуються кількістю.
Оголошення змінних та задання стандартних значень:

\begin{lstlisting}
			\define@key{titlee}{university}{\def\tl@university{#1}}
			\define@key{titlee}{katedra}{\def\tl@katedra{#1}}
			\define@key{titlee}{type}{\def\tl@type{#1}}
			\define@key{titlee}{discipline}{\def\tl@discipline{#1}}
			\define@key{titlee}{thema}{\def\tl@thema{#1}}
			\define@key{titlee}{kurs}{\def\tl@kurs{#1}}
			\define@key{titlee}{group}{\def\tl@group{#1}}
			\define@key{titlee}{specialty}{\def\tl@specialty{#1}}
			\define@key{titlee}{author}{\def\tl@author{#1}}
			\define@key{titlee}{posada}{\def\tl@posada{#1}}
			\define@key{titlee}{kerivnyk}{\def\tl@kerivnyk{#1}}
			\define@key{titlee}{pidpys}{\def\tl@pidpys{#1}}
			
			% zavdannia
			\define@key{titlee}{semestr}{\def\tl@semestr{#1}}
			\define@key{titlee}{date}{\def\tl@date{#1}}
			\define@key{titlee}{fulldate}{\def\tl@fulldate{#1}}
			%kalendar
			\define@key{titlee}{enddate}{\def\tl@enddate{#1}}
			\define@key{titlee}{sources}{\def\tl@sources{#1}}
			\define@key{titlee}{zapyska}{\def\tl@zapyska{#1}}
			\define@key{titlee}{graphika}{\def\tl@graphika{#1}}
			
			\setkeys{titlee}{university= University ,katedra = Kафедра,
			type= Курсова робота ,thema= {Тема \\ \ } , discipline = Предмет,
			kurs=№, group=ke-4, specialty=122 - CS, author=\qquad \qquad
			\qquad \qquad,posada=,kerivnyk=Teacher, semestr=4, date=,
			sources=sources, graphika=,enddate=,fulldate=, zapyska={\
			\quad \\ \ \\ \ \\ \ \\ \ \\ \ }, pidpys=}%			
\end{lstlisting}


Створюємо нові команди, всі будуть приймати один опційний аргумент, який потім в \verb|\setkeys{}|встановиться у відповідні змінні. Для використання змінних потрібно щоб вони перебували в групі, це оголошується \verb|\begingroup%|:

\begin{lstlisting}
			\newcommand{\setzavdannia}[1][]{
			\begingroup%
			\setkeys{titlee}{#1}% Set new keys
			...content
			\endgroup%
			}
\end{lstlisting}

Далі проведено опис основних моментів створення титульної сторінки, для детальнішого огляду див. додатки.

Починаємо зверху, де треба вказати навч. заклад та кафедру:

\begin{lstlisting}
		\centering
		Міністерство освіти і науки України\\
		\tl@university
		\hrule
		{\scriptsize (повне найменування вищого навчального закладу)}
		\hspace{0.2cm}
		Кафедра \tl@katedra
		\hrule
		{\scriptsize (повна назва кафедри)}
\end{lstlisting}

Команда \verb|\centering| відцентрує текст. У 2 рядку встановлюємо на параметр на його місце та малюємо горизонтальну лінію на всю ширину - \verb|\hrule|. \verb|\scriptsize| зменшує шрифт на дуже маленький.

Результат:

\addimghere{3}{1}{Верхня частина титулки}{}

Наступним створюємо тип та назву роботи:

\begin{lstlisting}
			\vspace{3cm}
			\begin{center}
			\textbf{ \large \tl@type}
			\end{center}
			
			з  \textbf{<<\tl@discipline>>}
			\hrule
			
			{\scriptsize (назва дисципліни)}
			\hspace{0.2cm}
			\setlength{\unitlength}{1cm}
			
			\begin{picture}(0,0)
				\put(-9,-1.85){\line(1,0){18}}
				\put(-9,-0.75){\line(1,0){18}}
				\put(-9,-1.3){\line(1,0){18}}
			\end{picture}
			
			на тему : \textbf{<<\tl@thema>>}
			}			
					
\end{lstlisting}

Команди \verb|\vspace{}, \hspace{}| роблять відступ вертикально та горизонтально на вказану відстань. Середовище \verb|\begin{picture}(0,0)| Робить область для створення простих графіків, фігур. В дужках вказано її розміри, в нас нема розміру для зручності, оскільки будуть малюватися лінії серед тексту. 

Команда \textit{put} ставить певну фігуру за вказаними координатами, її можна використовувати тільки в середовищі \textit{picture}. \textit{Line} - фігура лінії, в дужках осі, в параметрі довжина.

Ось результат:

\addimghere{4}{1}{Назва роботи, тема}{}

Перейдемо до останньої частини, де вказується інформація про автора та керівника. 

\begin{lstlisting}
			\vspace{5cm}
			\hfill
			\begin{minipage}{0.5\linewidth}
				\begin{tabular}{lp{0.1\linewidth}ccp{0.145\linewidth}}
					Студента & \centering \tl@kurs & курсу, & групи & \tl@group
					\\
					\cline{2-2} \cline{5-5}
				\end{tabular}
				\vspace{0.1cm}
				\begin{tabular}{lc}
					спеціальності & \tl@specialty \\
					\hline
					\tl@author    & \tl@pidpys    \\
					\hline
				\end{tabular}
				\vspace{-0.8cm}\hspace{0.3cm}	{ \centering\scriptsize (прізвище 
				та ініціали)  \hspace{2cm}(підпис студента)} \\
				
				\begin{tabular}{p{0.3\textwidth}p{0.6\textwidth}}
					Керівник: & \tl@posada \\
					\cline{2-2}
					\multicolumn{2}{c}{\tl@kerivnyk}\\
					\hline
				\end{tabular}
				\centering
				{\scriptsize (посада, вчене звання, науковий ступінь, прізвище
				та ініціали) }
			\end{minipage}
\end{lstlisting}

\addimghere{5}{1}{Остання частина титульної сторінки}{}


Поєднання \textit{begin\{minipage\}} та \textit{hfill} робить окрему область сторінки на половину її ширини та розташовує її справа. Для організації розмітки використовується таблиця, командами \verb|\cline{2-2} \cline{5-5}|, можна підкреслити необхідні стовпці в рядку.

Поєднуючи таблиці та малювання ліній, було зроблено необхідні лінії для 3 сторінок: титулка, завдання. календарний план.

\subsection{Завершення преамбули}

Для роботи із зображеннями підключаємо пакет \textit{graphicx}, вказуємо їхнє розташування:

\begin{lstlisting}
			\usepackage{graphicx} % Вставка картинок 
			\graphicspath{{images/}}
\end{lstlisting}

Напишемо пару нових команд для вставлення картинок. Перша команда вставляє одне зображення в зручному для Latex місці, приймає 4 команди: назва, ширина, підпис, позначка. Друга команда робить все те саме, проте вставляє картинку так як вона є в тексті. Остання команда вставляє 2 зображення поруч з одним підписом до них.

\begin{lstlisting}
			\newcommand{\addimg}[4]{ % add one img
				\begin{figure}
					\centering
					\includegraphics[width=#2\linewidth]{#1}
					\caption{#3} \label{#4}
				\end{figure}
			}
			\newcommand{\addimghere}[4]{ % add img here
				\begin{figure}[H]
					\centering
					\includegraphics[width=#2\linewidth]{#1}
					\caption{#3} \label{#4}
				\end{figure}
			}
			\newcommand{\addtwoimghere}[4]{ % two img side by side
				\begin{figure}[H]
					\centering
					\begin{subfigure}[t]{0.45\textwidth}
						\includegraphics[width=\textwidth]{#1}
					\end{subfigure}
					\begin{subfigure}[t]{0.45\textwidth}
						\centering
						\includegraphics[width=\textwidth]{#2}
						
					\end{subfigure}
					\caption{#3}\label{#4}
				
				\end{figure}
			}
\end{lstlisting}

Створимо ще дві функції для роботи із секціями. Команда \textit{newSection} додає нову секція з нової сторінки, також є позначення для посилання на неї, останнє - це застосування розширеної рамки і вказанням того самого розділу. Друга команда схожа, тільки секції будуть без нумерації.
\begin{lstlisting}
			\newcommand{\newSection}[3]{
				\newpage
				\section{\uppercase{#1}}
				\label{#2}
				\ESKDcolumnI{#3#1}
				\ESKDthisStyle{formII}
			}
			
			\newcommand{\anonsection}[2]{
				\newpage
				\phantomsection
				\addcontentsline{toc}{section}{\uppercase{#1}}
				\section*{\uppercase{#1}}
				\ESKDcolumnI{\uppercase{#1}}
				\label{#2}
				\ESKDthisStyle{formII}
			}
\end{lstlisting}

Один дуже важливий нюанс. Пакет ESKDX не надає можливості зміни даних в рамках в тілі документа, як це зробили ми, це призведе до помилки. Один з варіантів вирішення є написання власного стилю рамки, проте це дуже важко і потребує поглиблених знань пакету. Другий варіант - зміна пари рядків коду в початкових файлах пакету. Під час компіляції можна побачити що викликається команда збереження рамки в преамбулі та застосовується без змін далі, тому треба замість застосування - заново намалювати. Отже у пакетному файлі \textit{eskdstamp.sty} потрібно знайти 2 форму рамок(\verb|\ESKD@stamp@ii@box|) та скопіювати код створення рамки в місце де застосовується збережена рамка. 

Налаштуємо підписи до малюнків та таблиць:

\begin{lstlisting}
			\RequirePackage{caption}
			\DeclareCaptionLabelSeparator{defffis}{ -- } % Розділювач
			\captionsetup[figure]{justification=centering, labelsep=defffis,
			 format=plain} % Підпис малюнка по центру
			\captionsetup[table]{justification=raggedleft, labelsep=defffis, 
			format=plain, singlelinecheck=false} % Підпис таблиці справа
			\addto\captionsukrainian{\renewcommand{\figurename}{Рисунок}} 
			% Ім' я фігури
\end{lstlisting}

Оформимо відображення початкового коду, використовуючи пакет \textit{listings}:

\begin{lstlisting}
			\usepackage{listings}
			\lstset{
				basicstyle=\small\ttfamily,
				breaklines=true,
				tabsize=2,                  
				extendedchars=\true,
				keepspaces=true,
				literate={--}{{-{}-}}2,     
				literate={---}{{-{}-{}-}}3, 
				texcl=true, }
\end{lstlisting}

Останнє це оформлення секцій у вступі, зробимо так щоб відображалися крапки від секції до номера сторінки і щоб усе було нежирним:

\begin{lstlisting}
			\makeatletter
				%format sections in tableofcontents
				\renewcommand{\l@section}
					{\@dottedtocline{1}{0em}{1.25em}}
				\renewcommand{\l@subsection}
					{\@dottedtocline{2}{1.25em}{1.75em}}
				\renewcommand{\l@subsubsection}
					{\@dottedtocline{3}{2.75em}{2.6em}}
			\makeatother
\end{lstlisting}

Преамубла буде в окремому файлі \textit{preambule.tex}. Для вставлення інших файлів використовуєтсья команда \textit{input} або \textit{include}. Враховуючи титульну сторінку і конфіг класу, остаточний вигляд преамбули:

\begin{lstlisting}
			\documentclass[14pt,ukrainian,utf8, simple, 
			pointsection,floatsection ]{eskdtext} 
			
			\include{inc/preambula}
			
			\input{inc/title}
			
			\include{inc/ESKDXconfig}
\end{lstlisting}

\subsection{Тіло документа}

В першу чергу необхідно застосувати створені команди для титульної сторінки і завдання з календарним планом, передаючи необхідні параметри:

\addimghere{6}{1}{Використання команд для перших 3 сторінок}{}

Команда \verb|\ESKDthisStyle{empty}| робить стиль сторінки без рамки. Для зручності ці команди буде винесено в окремий файл \textit{titlepage.tex}. Також розділи і все інше буде окремо:

\begin{lstlisting}
			\begin{document}
			
			\renewcommand{\ESKDcolumnVIIname}{\ESKDfontIII Аркуш}
			\input{inc/titlepage}
			\input{chapters/0-1-referat}
			\ESKDthisStyle{formII}
			
			\tableofcontents
			\input{chapters/0-intro}
			\input{chapters/1-chap}
			\input{chapters/2-chap} 
			\input{chapters/3-chap}
			\input{chapters/5-bibl}
			%\ESKDappendix{ тип }{ заголовок }
			
			\end{document}
\end{lstlisting}

Для створення змісту використовується команда \verb|\tableofcontents|, для бібліографії - середовище \verb|\begin{thebibliography}| з пунктами \textit{bibitem} всередині.
			\input{chapters/5-bibl}
			%\ESKDappendix{ тип }{ заголовок }
			
			\end{document}
\end{lstlisting}

Для створення змісту використовується команда \verb|\tableofcontents|, для бібліографії - середовище \verb|\begin{thebibliography}| з пунктами \textit{bibitem} всередині.
			\input{chapters/5-bibl}
			%\ESKDappendix{ тип }{ заголовок }
			
			\end{document}
\end{lstlisting}

Для створення змісту використовується команда \verb|\tableofcontents|, для бібліографії - середовище \verb|\begin{thebibliography}| з пунктами \textit{bibitem} всередині.
			\input{chapters/5-bibl}
			%\ESKDappendix{ тип }{ заголовок }
			
			\end{document}
\end{lstlisting}

Для створення змісту використовується команда \verb|\tableofcontents|, для бібліографії - середовище \verb|\begin{thebibliography}| з пунктами \textit{bibitem} всередині.