\newSection{Історія Latex}{sec:seccc}{\ESKDfontV}


Говорячи про Latex потрібно вказати що він є, надбудовою над TEX - оригінальною системою текстового препроцесора. Все що можна зробити в Latex можна і в оригінальні системі TEX. Програмне забезпечення для набору тексту TEX було розроблено Дональдом Е. Кнутом в кінці 1970-х. Він був випущений з ліцензією з відкритим кодом і став стандартом наукового видавництва. Тепер TEX використовується для набору та публікації більшої частини світової інформації наукової літератури з фізики та математики.

Однією з найважливіших причин, по якій люди використовують LATEX, є те, що він відокремлює зміст документа від стилю. Це означає, що після написання вмісту вашого документа ми можемо легко змінити його вигляд. Так само ви можете створити один стиль документа, який можна використовувати для стандартизації зовнішнього вигляду безлічі різних документів. Це дозволяє науковим журналам створювати шаблони для публікацій. Ці шаблони мають заздалегідь зроблений макет, що означає, що потрібно додавати лише вміст.

Основний вплив для широкого розповсюдження здійснив
Леслі Лампорт у союзі з Пітером Гордоном в Аддісоні-Веслі,
версія 2.09 від приблизно середини 80-х років, яка походить від  системи TEX Дональда Кнута, яка досить швидко поширилася поза спільнотою північноамериканських математиків, які підтримували  розвиток TEX від його створення як один із "особистих інструментів продуктивності" Дона, створеного просто щоб забезпечити швидке завершення та типографічну якість його книги <<Мистецтво комп'ютерного програмування>> \cite{Knuth} Менш прямий, але, ймовірно, ширший вплив випливає з того, що вона є першою широко використовуваною мовою для опису логічної структури широкого кола документів таким \ чином \ її впровадження філософії логічного \ проектування, \ яке \newpage \noindent використовується Brain Reid in Scribe \cite{Reid}: "під час написання документа ви повинні
переймайтеся його логічним змістом, а не його візуальним оформленням."

Тоді Latex по-різному описувався як "TEX для мас" та "Написання, звільнене від негнучкого керування форматом ". Не зовсім
зрозуміло, чи все це було зроблено навмисне Леслі як особливість дизайну, але, безумовно, він не очікував що згодом, здійснить такий широкий вплив. Поширеність Latex була, навіть у кінці 1980-х років, більшою порівняно з більшістю некомерційного програмного забезпечення того часу. Хороші новини швидко поширювалися і 1994 року Леслі міг написати «Latex зараз надзвичайно популярний у науковій та академічній спільнотах, і він широко використовується у індустрії." Але цей рівень повсюдності все-таки був мізерним порівняно з сьогоднішнім днем, коли він став для багатьох професіоналів, невід'ємним інструментом, присутність якого є дуже важливою.

\subsection{Складові Latex(TeX)}

Робота з Latex пов'язана з так званими <<рівнями>> \cite{levels}:
\begin{enumerate}[label={\arabic*.},leftmargin=1.45cm]
	\item \textbf{Дистрибутиви}: MiKTeX, TeX Live,… Це великі колекції програмного забезпечення, пов'язаного з TeX, для завантаження та встановлення. Коли хтось каже: «Мені потрібно встановити TeX на свою машину», він зазвичай шукає дистрибутив.
	\item \textbf{Редактори}: Emacs, vim, TeXworks, TeXShop, TeXnicCenter, WinEdt,… ці редактори - це те, що ви використовуєте для створення файлу документа. Деякі (наприклад, TeXShop) присвячені спеціально TeX, інші (наприклад, Emacs) можуть використовуватися для редагування будь-яких файлів. Документи TeX не залежать від будь-якого конкретного редактора; сама програма набору TeX не включає редактор.
	\item \textbf{Компілятори}: TeX, pdfTeX, XeTeX, LuaTeX,… це виконувані бінарні файли, які реалізують різні варіанти TeX. Наприклад, pdfTeX реалізує прямий вихід у форматі PDF (якого немає в оригінальному TeX Кнута), LuaTeX забезпечує доступ до багатьох внутрішніх систем через вбудовану мову Lua і т.д. Коли хтось каже: "TeX не може знайти мої шрифти", вони зазвичай мають на увазі компілятор.
	\item \textbf{Формати}: LaTeX, звичайний TeX,… Це мови на основі TeX, якими фактично пишуть документи. Коли хтось каже, що "TeX дає мені невідому помилку", вони зазвичай мають на увазі формат. (До речі, "LaTeX" вже багато років означає "LaTeX2e".)
	\item \textbf{Пакети}: geometry, lm, ... Це доповнення до основної системи TeX, розроблені незалежно, надаючи додаткові функції набору тексту, шрифти, документацію тощо. Пакет може або не може працювати з будь-яким заданим форматом та/або компілятором; наприклад, багато є розроблено спеціально для LaTeX, але є і багато для інших. Сайти CTAN надають доступ до переважної більшості пакетів у світі TeX; CTAN, як правило, є джерелом, яке використовується дистрибутивами.
\end{enumerate}	

	Особливу увагу треба звернути на компілятори, в цій курсовій я використовую \textit{Xelatex}. Його особливістю є використання системний шрифтів, на відміну від Latex, який має свої вбудовані. Для оформлення документів в Україні згідно з ДСТУ 3008:2015 потрібно використовувати шрифт \textit{Times new Roman}, що і дозволяє зробити Xelatex. 
	
	\subsection{TeX і Latex}
	
	З моменту заснування розробка Latex та відповідного програмного забезпечення була повністю переплетена з розвитком самого TEX. 
	Хоча є багато сумнівів щодо корисністі деяких аспектів фундаментальних моделей та дизайну
	TEX як двигуна форматування тексту, він був тоді і залишається зараз (the
	початок тисячоліття) єдиним зрілим, широко доступним,
	програмованим і дуже гнучким компілятором для тексту. Таким чином для
	 Леслі в той час, як і для нас зараз, це єдиний вибір фундаменту
	для практичної автоматизації високоякісного форматування.
	
	У дизайні Latex Леслі свідомо дозволив основному компілятору TEX безпосередньо впливати на більшість текстових питань.
	У типових системах обробки тексту тієї епохи, включаючи TEX,
	основні методи обробки тексту документа такі: кожен вхідний маркер, що надсилається до системи, обробляється
	як складна імперативна команда. У таких системах "символ в
	тексті документа", як правило, подія на клавіатурі або маркер на
	вхідний буфер, не просто призначений викликати створення
	'елемента у рядку' в 'об'єкті текстового класу', такий 'рядок'
	врешті обробляється деяким іншим модулем системи, або
	навіть зовнішніми програмами.
	
	
	TEX був розроблений у цій імперативній парадигмі, оскільки це призводить до високоефективності(і в часі, і в просторі) машини, незважаючи на те, що "набір тексту" є для TEX відносно складним обчислювальним процесом, що включає, в першу чергу, оптимізацію	вибіру гліфа та позиціонування над цілими абзацами контрольоваий	за допомогою високо настроюваного алгоритму динамічного програмування. Однак, оскільки цей процес набору даних був оптимізований для швидкості, то робити що-небудь, що недоступно в рамках цього монолітного процесу (як визначено дизайном TEX), є важким у здійсненні та помітно неефективним у використанні. Такі процеси мають центральне значення для	якості набору і особливо важливі в наборі	інших мов, крім американської англійської. Вони включають модифікацію	важливих підпроцесів, таких як вибір гліфу (як для лігатур)	та їх розміри і розміщення; переноси та вирівнювання.
	
	
	\subsection{Версії}
	
	LaTeX2e - це поточна версія LaTeX, відколи вона замінила LaTeX 2.09 у 1994 році. Станом на 2019 рік LaTeX3, який розпочався створюватися на початку 1990-х років, досі розвивається. Планові функції включають покращений синтаксис, підтримку гіперпосилання, новий інтерфейс користувача, доступ до довільних шрифтів та нову документацію.\cite{latex3}
	
	Існують численні комерційні реалізації всієї системи TeX. Постачальники систем можуть додавати додаткові функції, такі як додаткові шрифти та підтримку по телефону. LyX - це безкоштовний, WYSIWYM-процесор візуального документа, який використовує LaTeX для бек-енду. TeXmacs - безкоштовний редактор WYSIWYG з аналогічними функціями, як LaTeX, але з іншим механізмом набору тексту. Інші редактори WYSIWYG, які використовують LaTeX, включають Scientific Word у MS Windows. Та BaKoMa TeX для Windows, Mac та Linux.
	
	Доступно декілька дистрибутивів TeX, що підтримуються спільнотою, зокрема TeX Live (багатоплатформна), teTeX (застаріла на користь TeX Live, UNIX), fpTeX (застаріла), MiKTeX (Windows), proTeXt (Windows), MacTeX (TeX Live з додаванням специфічних програм для Mac), gwTeX (Mac OS X) (застарілий), OzTeX (Mac OS Classic), AmigaTeX (більше не доступний), PasTeX (AmigaOS, доступний у сховищі Aminet) та Auto-Latex Equations (Додаток Google Docs, який підтримує команди MathJax LaTeX).
	
	\subsection{Сумісність та конвертація}
	
	Документи LaTeX (* .tex) можна відкрити будь-яким текстовим редактором. Вони складаються з простого тексту та не містять прихованих кодів форматування чи двійкових інструкцій. Крім того, документи TeX можна поширити у формат Rich Text (* .rtf) або XML. Це можна зробити за допомогою безкоштовних програм LaTeX2RTF або TeX4ht. LaTeX також може бути конвертовано у PDF-файли за допомогою розширення LaTeX pdfLaTeX. Файли LaTeX, що містять текст Unicode, можуть бути оброблені в PDF-файли за допомогою пакету \textit{inputenc} або розширеннь Tee XeLaTeX і LuaLaTeX.
	
	\begin{itemize}
		\item HeVeA - це перетворювач, написаний на Ocaml, який перетворює документи LaTeX у HTML5. Він ліцензований відповідно до Q Public License.
		
		\item LaTeX2HTML - це перетворювач, написаний на Perl, який перетворює документи LaTeX в HTML. Таким чином, наприклад, наукові праці, головним чином набрані для друку, можна розмістити для перегляду в Інтернеті. Він ліцензований відповідно до GNU GPL v2.
		
		\item LaTeXML - це безкоштовне програмне забезпечення публічного домену, написане на Perl, яке перетворює документи LaTeX у різноманітні структуровані формати, включаючи HTML5, epub, jats, tei.
		\item Pandoc - це "універсальний конвертер документів", здатний трансформувати LaTeX у безліч різних форматів файлів, включаючи HTML5, epub, rtf та docx. Він ліцензований відповідно до GNU GPL v2.
	
	\end{itemize}

	LaTeX став стандартом для набору математичних виразів в наукових документах. Таким чином, існує кілька інструментів перетворення, орієнтованих на математичні вирази LaTeX, такі як перетворювачі в MathML  або Computer Algebra System.
	
	\begin{itemize}
		\item  Mathoid - це веб-конвертер який використовує Node.js, він перетворює математичні входи, такі як LaTeX, у формати MathML та зображення, включаючи SVG та PNG. Він використовується у Вікіпедії для відображення математики.
		\item TeXZillais перетворювач на JavaScript з LaTeX в MathML. Це один з найшвидших перетворювачів LaTeX в MathML.
		\item  LaCASt - це перетворювач, написаний на Java, який перетворює семантичний діалект LaTeX в Maple та Mathematica.
	\end{itemize}


	\subsection{Особливості Latex}
	
	Створення LaTeX документа це програмування: Ви створюєте текстовий файл в LaTeX-розмітці, макроси LaTeX обробляють його і видають конкретний документ.
	
	Такий підхід відрізняється від використовуваного в WYSIWYG (What You See Is What You Get - те, що ви бачите, то і отримуєте) програмах, таких, як Openoffice Writer або Microsoft Word.
	
	В LaTeХ: 
	
	\begin{itemize}
		\item Під час редагування документа Ви не можете (зазвичай) побачити його остаточний варіант.
		
		\item 	Вам, як правило, потрібно знати необхідні команди розмітки LaTeX.
		
		\item Інколи складно отримати необхідний вигляд документа.
		
	\end{itemize}


		З іншого боку, у LaTeX є і переваги:
		
	\begin{itemize}
		\item Оформлення тексту відокремлено від вмісту. Ви повною мірою зосереджуєтеся на структурі та вмісті документу і забуваєте про те, як буде виглядати друкований варіант.
		\item Стиль, шрифти, оформлення таблиць і малюнків т. д. узгоджено у всьому документі.
		\item Одне і те саме оформлення можна використовувати для будь-якого числа документів.
		\item Легко набирає математичні формули.
		\item Легко створюються алфавітні вказівники, посилання та бібліографічні списки.
		\item Більші документи можуть бути розподілені на декілька файлів і працювати з ними окремо, в тому числі з використанням системи управління версіями.
		\item Вам не потрібно вручну налаштовувати шрифти, розмір тексту, високий шрифт - за це відповідає  LaTeX.
		\item  Вам доведеться правильно структурувати ваш документ.
		\item Файли з вихідними текстами можна переглянути і змінити в любому текстовому редакторі.
	
	\end{itemize}

	Підхід LaTeX до створення документа можна назвати WYSIWYM (What You See Is What You Mean - що бачиш, то і думаєш): під час набору тексту Ви не бачите остаточний варіант документа, тільки логічну структуру цього документа. Про оформлення замість Вас подбає LaTeX.

	
		