\anonsection{Вступ}{sec:intro}
\ESKDthisStyle{formII}

Написання та оформлення текстів у високій якості потребує спеціального програмного забезпечення(ПЗ). Особливо важливо мати можливість використовувати системи набору тексту для написання документів, книг, статтей та інших форм видавничої справи в науковій галузі. Через необхідність написання складних текстів з красивим оформленням для наукових праць -- ця тема є актуальною для мене, оскільки, навчаючись в університеті потрібно вести звітність про виконання завдань, і найкращий спосіб вирішення цієї проблеми - це розробити шаблон для робіт, і сконцентруватися на самому завданні.

Багато ПЗ можна привести для прикладу, но ми зосередимося на наборі макророзширень системи комп'ютерної верстки \textbf{Latex}, яка розглядається в цій курсовій.

В ході наступних розділів буде розглянуто історію створення Latex, його основні можливості. Далі буде проводитись розробка шаблону для курсових робіт. Маючи такий шаблон, його можна буде легко модифікувати для використання в лабораторних роботах. В останньому розділі буде порівняння готового шаблону з іншими шаблонами, знайденими в мережі інтернет.
