\newSection{Інструментарій Latex}{sec:sec2}{\ESKDfontV}

Почати роботу з Latex необхідно зі створення нового документа з розширенням \textit{.tex}, при умові що у Вас встановлені всі необхідні пакети Latex. Як вже було сказано раніше треба мати встановлений дистрибутив та редактор. В дистрибутиві будуть знаходитися пакети та компілятори для створення документів. В якості дистрибутива встановлено \textit{texlive}, а редактор вихідного документа \textit{TexStudio}. TexStudio надає зручні можливості для роботи з  командами,наявні автодоповнення, сполучення клавіш - це все робить його  середовищем  для розробки Latex.

\subsection{Перший документ}

Створюємо файл з розширенням .tex та записуємо в нього наступні рядки:

\begin{lstlisting}
	\documentclass{article}
	
	\begin{document}
	
	\end{document}
\end{lstlisting}

На виході буде пустий документ. Перший рядок вказує на клас документу.Форматування за замовчуванням у документах LATEX визначається класом, який використовується цим документом. Стандартний вигляд можна змінити, а додаткові функції можна додати за допомогою пакета. Імена файлів класу мають розширення .cls, імена файлів пакунків мають розширення .sty.

Для різних типів документів потрібні різні класи, тобто для резюме буде потрібен інший клас, ніж для  наукового документа. У цьому випадку клас - це \textit{article}, найпростіший і найпоширеніший клас LATEX. \ Інші \ типи \ документів, \ над \newpage \noindent якими ви можете працювати, можуть вимагати різних класів, таких як \textbf{book} або \textit{report}.

Далі з команди \textit{begin} починається так зване тіло документа. В ньому буде знаходитися вміст документа, аж до \textit{end}

Щоб переглянути документ треба провести компіляцію. До прикладу такою командою:

\begin{lstlisting}
	pdflatex <your document> 
\end{lstlisting}

\subsection{Преамбула документа}

У попередньому прикладі текст був введений після команди  \verb|\begin {document} |. Все у .tex-файлі до цього моменту називається преамбулою. У преамбулі ви визначаєте тип документа, який ви пишете, мову, якою ви пишете, пакети, які ви хочете використовувати та декілька інших елементів. Наприклад, звичайна преамбула документа виглядатиме так:

\begin{lstlisting}
	\documentclass[12pt, letterpaper]{article}
	\usepackage[utf8]{inputenc}
\end{lstlisting}

\verb|\documentclass[12pt, letterpaper]{article}|. Як було сказано раніше, це визначає тип документа. Деякі додаткові параметри, що входять до квадратних дужок, можуть передаватися команді. Ці параметри повинні бути розділені комами. У прикладі додаткові параметри встановлюють розмір шрифту \textit{(12pt)} та розмір паперу (letterpaper). Звичайно, можна використовувати інші розміри шрифту \textit{(9pt, 11pt, 12pt)}, але якщо не вказано жодного, розмір за замовчуванням - 10pt. Що стосується розміру паперу, інші можливі значення - це папір формату А4 та legalpaper; 

\verb|\usepackage[utf8]{inputenc}|. Це кодування документа. Його можна опустити або змінити на інше кодування, але рекомендується utf-8. Якщо вам конкретно не потрібно інше кодування, або якщо ви не впевнені в цьому, додайте цей рядок до преамбули.


Для форматування тексту, задання формату сторінки, додавання графічних елементів та задання всіх можливих параметрів і налаштувань використовуються різні команди, які можна задати в класі документу або використовуючи додаткові пакети. 

\subsection{Макет сторінки}

За замовчуванням всі параметри документа встановлюють класи документів. Однак,
якщо ви хочете змінити ці параметри , є команди, які дозволяють вам це зробити. Команди, що контролюють функції, що стосуються всього документа, повинні бути розміщені в преамбулі.

\subsubsection{Параграфи}\label{par}

Щоб почати новий абзац, залиште порожній рядок або використовуйте команду \verb|\par|. Команди \verb|\parindent| і \verb|\parskip| задяють відступ абзацу та розділення абзацу. 

Для задання міжрядкового інтервалу можна використати команду \verb|\renewcommand{\baselinestretch}{1.5}|, яка встановить його на 1.5.

Абзацний відступ задається командою \verb|\parindent 1.25cm|.\par Перейти на новий рядок \verb|\\ або \par|

\subsubsection{Вирівнювання тексту}

За замовчуванням LATEX вирівнює ваш текст горизонтально, так що лівий і правий відступи є гладкими. Якщо ви віддаєте перевагу "вирівнювання справа" , ви можете використовувати:
\verb|\raggedright|
Зауважте, що це має побічний ефект для відступів абзацу. Якщо ви хочете залишити відступ абзаців, потрібно спеціально задати його (тобто \verb|\parindent = 1.5em|) після \verb|\raggedright| команди.

\subsubsection{Коментарі}

Як і будь-який код, який ви пишете, часто корисно включати коментарі. Коментарі - це фрагменти тексту, які ви можете включити в документ, які не будуть надруковані, і жодним чином не вплинуть на документ. Вони корисні для організації вашої роботи, ведення приміток або коментування рядків / розділів під час налагодження. Щоб зробити коментар у LATEX, просто напишіть символ \verb|%| на початку рядка.

\subsubsection{Вигляд тексту}

Зараз ми розглянемо кілька простих команд форматування тексту.

\begin{itemize}
	\item Жирний: Жирний текст у LaTeX пишеться командою \verb|\textbf {...}|
	\item Курсив: Курсивний текст у LaTeX пишеться командою \verb|\textit {...}|
	\item Підкреслення: Підкреслений текст у LaTeX пишеться командою \verb|\underline {...}|
\end{itemize}

LaTeX має кілька команд-модифікаторів розміру шрифту (від найбільших до найменших):

\begin{lstlisting}
		\Huge
		\huge
		\LARGE
		\Large
		\large
		\normalsize (default)
		\small
		\footnotesize
		\scriptsize
		\tiny
\end{lstlisting}

\subsubsection{Додавання зображень, таблиць}

Стандартний комплект графіки LaTeX включає два пакети для імпорту графіки:
\textit{graphics} та \textit{graphicx}. Розширений пакет, \textit{graphicsx}, забезпечує більш зручний спосіб подачі параметрів і рекомендується. Тому перший крок - це введіть у свою преамбулу команду:
\verb|\usepackage {graphicsx}|. Цей пакет визначає нову команду під назвою \verb|\includegraphics|, яка дозволяє вам вказувати назву графічного файлу, а також надавати необов'язкові аргументи для масштабування чи обертання. Отже, для прикладу потрібно вставити графіку (названу, наприклад, myfigure.eps
або myfigure.pdf) використовуйте команду \verb|\includegraphics|, наприклад:

\begin{lstlisting}
	\includegraphics[width=4in]{myfigure}
\end{lstlisting}


Tabular середовище є стандартним в LATEX для створення таблиць. Ви повинні вказати параметр для цього середовища, в цьому випадку {c c c}. Це говорить про те, що LATEX буде три стовпці і текст у кожному з них повинен бути в центрі. Ви також можете використовувати r, щоб вирівняти текст праворуч, а l - для вирівнювання ліворуч. Символ \& використовується для визначення розділення у записах таблиці. У кожному рядку завжди повинно бути на один менше символів розділення, ніж кількість стовпців. Для переходу до наступного рядка таблиці використовуємо команду нового рядка \verb|\\|.

\begin{lstlisting}
	\begin{tabular}{ c c }
		cell 1 & cell 2 \\
		cell 3 & cell 4 
	\end{tabular}
\end{lstlisting}

Ви можете додати межі за допомогою команди горизонтальної лінії \verb|hline| та параметра вертикальної лінії |.

\begin{lstlisting}
	\begin{tabular}{ |c|c| }
		\hline 
		cell 1 & cell 2 \\ \hline
		cell 3 & cell 4 \\ \hline
	\end{tabular}
\end{lstlisting}


{| c | c | c | }: Тут вказується, що у таблиці будуть використані три стовпчики, розділені вертикальною лінією. | символ вказує, що ці стовпці повинні бути розділені вертикальною лінією. 

\verb|hline|: буде вставлена горизонтальна лінія. Тут ми ввели горизонтальні лінії вгорі та внизу таблиці. Немає обмежень у кількості разів, коли ви можете використовувати \verb|\hline|.

\subsubsection{Структурування}

Команди для організації документа відрізняються залежно від типу документа, найпростішою формою організації є секціонування, доступне у всіх форматах.

Команда \verb|section{}| позначає початок нового розділу, всередині дужок встановлюється заголовок. Нумерація розділів є автоматичною і її можна відключити, включивши * в команду розділу як \verb|\section*{}|. Ми також можемо мати підрозділи \verb|\subsection{}|, і пункти  \verb|sububsection{}|. 

\subsubsection{Підписи до зображень, таблиць}\label{capt}

В \textit{figure} або \textit{table}
середовищах, ви можете надати підпис із командою \verb|\caption {caption text}|. Зазвичай підписи для таблиці вводяться над таблицею, а підпис до зображення
нижче зображення.

Якщо потрібно якось модифікувати вигляд підпису то можна скористатися пакетом \textit{caption}. Наприклад відцентрувати підпис зображення, та змінити слово Рис. на Рисунок:

\begin{lstlisting}
		\captionsetup[figure]{justification=centering, labelsep=defffis,
		format=plain}		
		\addto\captionsukrainian{\renewcommand{\figurename}{ Рисунок }}
\end{lstlisting}


\subsection{Створення та перевизначення команд}

Щоб додати свої власні команди, використовуйте команду:

\begin{lstlisting}
		\newcommand{\name}[num]{definition}
\end{lstlisting}


В основному, команда вимагає двох аргументів: ім'я команди, яку ви хочете створити, і визначення команди. Зауважте, що ім'я команди можна, але не потрібно вкладати в дужки, як вам подобається. Аргумент num у квадратних дужках є необов’язковим і визначає кількість аргументів, які приймає нова команда (можливо до 9). Якщо він відсутній, він за замовчуванням дорівнює 0, тобто аргумент не дозволений.

Для прикладу створимо команду додавання зображення:

\begin{lstlisting}
		\newcommand{\addimg}[4]{
			\begin{figure}
				\centering
				\includegraphics[width=#2\linewidth]{#1}
				\caption{#3} \label{#4}
			\end{figure}
		}
\end{lstlisting}

Команда отримує 4 аргументи, для використання аргументу вводять \# та номер параметра.

Щоб перевизначити команду треба ввести \verb|\renewcommand{cmd}{def}|. Ми вже використовували її для визначення підпису до малюнків на ст. \pageref{capt} та для міжрядкового інтервалу ст. \pageref{par}

Ще багато команд можна було б описати, проте цього достатньо для початку. В наступному розділі буде відбуватись створення шаблону, в процесі використань нових команд буде надано  їхній короткий опис.


