\usepackage[figure,table]{totalcount} % counting figures, tables
\usepackage{eskdtotal} % total figures, tables...

\makeatletter
%format sections in tableofcontents
\renewcommand{\l@section}{\@dottedtocline{1}{0em}{1.25em}}
\renewcommand{\l@subsection}{\@dottedtocline{2}{1.25em}{1.75em}}
\renewcommand{\l@subsubsection}{\@dottedtocline{3}{2.75em}{2.6em}}
\makeatother

\usepackage{hyperref}
\hypersetup{
	colorlinks=false,
	hidelinks,
	bookmarks=true,
	,unicode=true
	,pdfcreator={XeLaTeX}
	,pdfa=true}

\renewcommand{\baselinestretch}{1.5} % Полуторный межстрочный интервал
\usepackage{graphicx}
\graphicspath{{images/}}

\usepackage[none]{hyphenat} % без переносів
\sloppy

% Формат подрисуночных надписей
\RequirePackage{caption}
\DeclareCaptionLabelSeparator{defffis}{ -- } % Разделитель
\captionsetup[figure]{justification=centering,
labelsep=defffis, format=plain} % Подпись рисунка по центру
\captionsetup[table]{justification=raggedleft, labelsep=defffis, 
format=plain, singlelinecheck=false} % Подпись таблицы справа
\addto\captionsukrainian{\renewcommand{\figurename}{Рисунок}}

\usepackage{float}
\usepackage{wrapfig}
\usepackage{subcaption}
\usepackage{array,tabularx,tabulary,booktabs}
\newcommand{\newSection}[3]{
	\newpage
	\section{\uppercase{#1}}
	\label{#2}
	\ESKDcolumnI{#3#1}
	\ESKDthisStyle{formII}
}
\newcommand{\anonsection}[2]{
	\newpage
	\phantomsection
	\addcontentsline{toc}{section}{\uppercase{#1}}
	\section*{\uppercase{#1}}
	\ESKDcolumnI{\uppercase{#1}}
	\label{#2}
	\ESKDthisStyle{formII}
}
% Списки
\usepackage{enumitem}
\setlist[enumerate,itemize]{leftmargin=1.5cm} % Отступы в списках
\setlist{nosep} % no separations

\usepackage{listings} % Оформление исходного кода
\lstset{
basicstyle=\small\ttfamily, % Размер и тип шрифта
breaklines=true,            % Перенос строк
tabsize=2,                  % Размер табуляции
extendedchars=\true,
keepspaces=true,
%frame=single,               % Рамка
literate={--}{{-{}-}}2,     % Корректно отображать двойной дефис
literate={---}{{-{}-{}-}}3,  % Корректно отображать тройной дефис
texcl=true,
}
\newcommand{\addimg}[4]{ 
	\begin{figure}
		\centering
		\includegraphics[width=#2\linewidth]{#1}
		\caption{#3} \label{#4}
	\end{figure}
}
\newcommand{\addimghere}[4]{ 
	\begin{figure}[H]
		\centering
		\includegraphics[width=#2\linewidth]{#1}
		\caption{#3} \label{#4}
	\end{figure}
}
\newcommand{\addtwoimghere}[4]{
	\begin{figure}[H]
		\centering
		\begin{subfigure}[t]{0.45\textwidth}
			\includegraphics[width=\textwidth]{#1}

			%\subcaption{}\label{sub:2a}
		\end{subfigure}
		\begin{subfigure}[t]{0.45\textwidth}
			\centering
			\includegraphics[width=\textwidth]{#2}
			%	\subcaption{}\label{sub:2b}
		\end{subfigure}
		\caption{#3}\label{#4}

	\end{figure}
}


